%\documentclass{ncc}
%\documentclass{article}
\documentclass{scrartcl}

\usepackage{amsmath,amssymb,amsfonts} % Typical maths resource packages
\usepackage{graphics}                 % Packages to allow inclusion of graphics
\usepackage{color}                    % For creating coloured text and background
\usepackage{hyperref}                 % For creating hyperlinks in
                                % cross references 

\usepackage{algorithm}
\usepackage{algorithmic}

\usepackage[T2A]{fontenc}
\usepackage[utf8]{inputenc}
\usepackage[russian,english]{babel}
\usepackage{listings}
\lstloadlanguages {[LaTeX]TeX, Octave}
\lstset {language=[LaTeX]TeX,
  extendedchars=true ,escapechar=|}

\begin{document}
\section{Blah-blah}
\label{sec:1}

\begin{itemize}
\item Finance is about value (not money)
\item Value creation is about two key components: time and uncertainty
\item (Almost) anything can be valued: project, human capital etc.
\end{itemize}

Assumptions:
\begin{itemize}
\item Competitive markets
\item Frictions small relative to power of most good ideas (the small changes
  should be sufficient to make the big changes)
\item Capital can flow (relatively) easily
\end{itemize}

\paragraph{Essence of Desicision-Making}
\begin{itemize}
\item (Virtually) every decision involves time and uncertainty
\item Important to understand the impact of just the passage of time on a
  decision
\item At first, assume ``no uncertainty'' to internalise the time value of money
\end{itemize}
Terminology:
\begin{itemize}
\item PV = Present Value (\$)
\item FV = Future Value (\$)
\item n = \# of Periods (\#)
\item r = Interest Rate (\%) > 0 (assumption) - applied to one time period
\end{itemize}
Time line: ``A dollar today is worth more then a dollar tomorrow'' - i.e. time
``by itself'' have some value - so {\bf do not compare money across the time!}

\section {Time Value for Money: PV and FV for Single Cash Flows}
\label{sec:2}
In general, {\bf Future Value} = Initial Payment + Accumulated Interest: $$FV =
P + r * P = (1+r) * P$$

P - Payment, r - rate. Then $(1 + R)$ - a {\it future value factor}

For two years: $$FV_2 = (1 + r)^2 * P$$, this is called ``Compounding'' percent.

To calculate FV after 10 years of 7\% year rate for \$500 deposit:

Using Excel: $FV(rate, nper, pmt, [pv], [type])$
\begin{itemize}
\item rate - now 0.07
\item nper - 10 (number of period)
\item pmt - 0 (see later)
\item PV - 500
\end{itemize}
Returns -983.58: negative (some dance about this negative here).

{\bf Present Value} - can be derived from FV by boring
$\frac{FV}{(1+r)^{nper}}$. So the future value becomes lesser now - so this
``phenomenon'' (that something becomes less valuable today comparatively to its
value in future) is called {\bf discounting}.

In Excel: $PV(rate, nper, pmt, [fv],[type])$; pmt is still 0; other are obvious.

\section {Time Value for Money: PV and FV for Multiple Cash Flows}
\label{sec:3}

\subsection{FV of Annuity - Concept}
\label{sec:3-2}

A special case of multiple payments: annuities (C for Cash Flow, PMT for
payment). Classic annuity: loan

\begin{tabular}{c|c|c|c}
  Year & Cash Flow & Years to end: n & Future Value \\
  \hline
  0 & 0 & 3 & 0 \\
  1 & C & 2 & $C(1 + r)^2$ \\
  2 & C & 1 & $C(1 + r)$ \\
  3 & C & 0 & C
\end{tabular}

FV of annuity: formula: $FV = C(1 + r)^2 + C(1+r) + C = \\
= C[(1+r)^2 + (1+r) + 1]$. Finally: $$FV_n = C[(1+r)^{n-1} + \dots + 1]$$

Example: I'm going to put into pension fund 10,000 every year running for 40
years. Rate is 8\%. How much money will it be at the end? (2.59 million)

In Excel, payment (10,000) is a ``pmt'' part of FV function (in out particular
case PV becomes 0).

Example 2: we want to guarantee 500,000 for retirement after 25 years. The rate
is 8\%. How much do I need to invest every year, starting at the end of this
year?

FV is know, so use in Excel: $PMT((rate, nper, pv, fv, hype)$:
\begin{itemize}
\item rate - 0.08
\item nper - number of periods, 25
\item pv - present value, 0
\item fv - future value, 500,000
\end{itemize}
Answer: 6,840

\subsubsection{Annuities: Present Values}

\begin{tabular}{c|c|c|c}
  Year & Cash Flow & Years to Discount: n & Present Value \\
  \hline
  0 & 0 & 0 & 0 \\
  1 & C & 1 & $C/(1 + r)$ \\
  2 & C & 2 & $C/(1 + r) + C/(1 + r)^2$ \\
  3 & C & 3 & $C/(1 + r) + C/(1 + r)^2 + C/(1 + r)^3$
\end{tabular}

Years to discount - because this is so many years the relative sum should be
discounted for today

In the end of $3^{rd}$ period: $PV_3 = \frac{C}{(1+r)} + \frac{C}{(1+r)^2}
+\frac{C}{(1+r)^3} = C \left[\frac{1}{(1+r)} + \frac{1}{(1+r)^2} +
  \frac{1}{(1+r)^3} \right]$

\paragraph{PV of an Annuity: Example 1}
\label{sec:3-6}

How much money do you need in the bank today so that you can spend 10,000 every
year for the next 25 years, starting at the end of this year. Suppose r=5\%.

Solution: PMT=C; and $PV = C [ \frac{1}{(1+r)} + \frac{1}{(1+r)^2} + \dots +
\frac{1}{(1+r)^{25}} ]$

In Excel: $=PV(0.05, 25, 10000)$; answer 140,939

\paragraph{PV of an Annuity: Example 2}
\label{sec:3-6-1}

You plan to attend a business school and you will be forced to take out 100,000
in a loan at 10\%. You need to figure out yearly payments, given that you'll
have 5 years to pay back the loan.

Here: PV = 100,000. We're looking for PMT / C for every year:

In Excel: $PMT(0.10, 5, 100,000)$; answer is 26.380 (!!!)

Validation: PV(0.1, 5, 26380

\paragraph{Example 2: Loan Amortisation} same story

\begin{tabular}{c|c|c|c|c}
  Year & Beginning Balance & Yearly Payment & Interest & Principal Repayment \\
  \hline
  1 & 100,000 & 26,380 & 10,000  & 16,380 \\
  2 & 83,620 & 26,380 & 8,362 & 18,018 \\
  3 & 65.602 & 26,380 & 6,560 & 19,820 \\
  4 & 45,783 & 26,380 & 4578 & 21802 \\
  5 & 23,982 & 26,380 & 2398 & 23982 
\end{tabular}

Next questions: how much do you owe to bank at the beginning of year. Calculate
PV for 3 (remaining) years with payment 26,380.

Another one: what is the sum you have to pay, staying at the beginning of year
have 5 years to pay back the loan? The answer is not the 5 * 26.380, it is
PV(0.1, 5, 26380) = 100,000

The idea is: it is impossible to make money {\it only} by borrowing-landing, so
bank charges additional sums when provides a loan.

\subsubsection{Compounding}
\label{sec:3-8}

Now change story a bit: what are the monthly payments for the same 100,000, 10\%
and 5 years. Plus, what is the ``real'' annual interest rate?

Time line changes, so 5 years become 60 months; rate r becomes $r_{month} =
\frac{0.10}{12}$.

The PV = 100,000; so calculate PMT for 60 months: $=P(0.1/12, 60, 100000)$;
becomes 2,124.7

Actual annual rate (Effective Annual Rate, EAR): calculate sums for one dollar:
$EAR=\left(1 + \frac{r}{k}\right)^k - 1$. Using k=12 and r=0.1, get 10.47\%
which $\gg 10\%$.

\subsubsection{Valuing Perpetuity}
\label{sec:3-9}

A {\bf perpetuity} as simply a set of equal payments that are paid forever, with
or without growth. Examples: Bonds, stocks, etc.

$PV = \frac{C}{r}$ - the PV is the same, so C is the same too. If, however, the
sum should grow over time, the growth rate g is used to measure/calculate the
growth: $PV = \frac{C}{r-g}$

\section{Decision Making}
\label{sec:4}

\subsection{Properties of a good decision criteria}
\label{sec:4-2}

\begin{itemize}
\item Makes sense (benefits exceeds costs) - including non-money values
\item Unit of measurement
\item Benchmark is obvious
\item Easy to communicate (oops)
\item Easy to compare different ideas/projects
\item Easy to calculate
\item Other (?)
\end{itemize}


\subsection{Decision Criteria: NPV}
\label{sec:4-2-1}


{\bf Net Present Value} (NPV) - differs from ``just'' Present Value as we
subtract the expenses (?)

Assume that interest rate is 10\%, what is NPV of the idea?

\begin{tabular}{c|c|c|c}
  Year & Cash Flow & Years to Discount: n & Present Value  \\
  \hline
  0 & -\$1,000 & 0 & -\$1,000  \\
  1 & \$1,320 & 1 & \$1,200 \\
  \hline
  \multicolumn{3}{l} {NPV =}  & \$200
\end{tabular}

% Video 4-3
\begin{itemize}
\item {\bf Cash flow } comes from idea (is an attribute of business idea);
  equals to profit (not income). Responsibility of ``you'' as initiator
\item ``r'' (10\%) comes from ``opportunity cost of investing''; it is a return
  from investing into similar business (competitor?)
\item Final number (NPV) - created value
\item Should you pursue this idea/project - yes, because you're in plus
\item Caution: if you do not have resources to do? Go to market; ``if the idea
  is good, the resources will come'' (?)
\end{itemize}

The essence: Value is always incremental to investment (relative, never
absolute!).

\begin{tabular}{c|c|c|c}
  Year & Cash Flow & Years to Discount: n & Present Value  \\
  \hline
  0 & -\$1,000 & 0 & -\$1,000  \\
  1 & \$1,320 & 1 & \$1,200 \\
  1 & \$1,452 & 2 & \$1,200 \\
  \hline
  \multicolumn{3}{l|} {NPV =}  & \$1400
\end{tabular}

In Excel: $NPV(rate, value1, value2, \dots)$ but value starts from the end of
year 1, so {\it do not give time 0 info there!}.

\subsection{NPV - Properties and Calculation}
\label{sec:4-4}

NPV starts with a negative number (investment): $$NPV = -I_0 + \frac{C_1}{1+r} +
\frac{C_2}{(1+r)^2} + \dots + \frac{C_n}{(1+r)^n}$$ where
\begin{itemize}
\item $I_0$ (or $C_0$) - investment cost of the project
\item $C_i$ - the cash flow in period $i$.
\end{itemize}

Properties of NPV:
\begin{itemize}
\item Makes sense? $B-C (TVM)$
\item Unit of measurement? \$ (1400 NPV)
\item Benchmark obvious? $NPV > 0$
\item Easiest to communicate - yes
\item Easy to compare - compare NPV values, so yes
\item Easy to calculate - {\it no}
\item Other? 
\end{itemize}

The main problem with NPV is that it provides very static view of future, and
does not works with flexibility in future.

\subsection{Decision Criteria: Payback}
\label{sec:4-5}

\subsubsection{Payback Period}
\label{sec:4-5-1}

What is the payback of the following idea:
\begin{tabular}{c|c|c}
  Year & Cash Flow & Years from Today \\
  \hline
  0 & -\$1,000 & 0  \\
  1 & \$300 & 1 \\
  2 & \$700 & 2 \\
  3 & \$2000 & 3 \\
  \hline
  \multicolumn{2}{l|} {NPV =}  & \$1400
\end{tabular}

Answer: 2 years (\$300 + \$700 = \$1000)

Better parameter is {\bf Payback period with discounting}.

Properties: \begin{itemize}
\item Makes sense: {\bf no}
\item Unit of measurement {\bf time} - this makes the whole story useless (oops)
\item Benchmark obvious? not too much
\item Easy to communicate? not too much
\item Easy to compare ideas? again, NPV is better
\item Easy to calculate? again, NPV is better
\end{itemize}

The problem with Payback is that it is hungry for ``fast money''. Is considered
as ``bad habit''.

\subsection{Decision Criteria: IRR}
\label{sec:4-6}

{\bf IRR} stands for {\bf Internal Rate of Return}

\begin{tabular}{c|c|c}
  Year & Cash Flow & Years from Today \\
  \hline
  0 & -\$100 & 0  \\
  1 & \$110 & 1 \\
%  \hline
%  \multicolumn{2}{l|} {NPV =}  & \$1400
\end{tabular}

Naive answer: 10\% (\$110 - \$100) / year, or $r = \frac{\text{Final Sum} -
  \text{Initial Sum}}{\text{Initial Sum}} =  \frac{\text{Profit}} {\text{Investment}}$

\paragraph{Intuition } What is the NPV of the idea if you use the IRR to
calculate it?$-100 + \frac{110}{1 + IRR} = 0$

This is why this parameter is ``internal'' - no reference to rate (as an
indicator of how other similar businesses are going on)

{\bf Is a good idea?} Not too much as there is no benchmark to compare.

%slides 4-7 - Graphical representation
Example: calculate IRR here:
\begin{tabular}{c|c|c}
  Year & Cash Flow & Years from Today \\
  \hline
  0 & -\$100 & 0  \\
  1 & \$0 & 1 \\
  2 & \$110 & 2 \\
%  \hline
%  \multicolumn{2}{l|} {IRR =}  & \$1400
\end{tabular}

Here IRR = 10\% over 2 years. How to solve it (to find per year)? A: ``Make NPV
zero'':$$NPV=0=-100+\frac0{1+IRR} + \frac{110}{(1+IRR)^2}$$

In Excel: $=IRR(values...) = IRR(-100, 0, 110)$

\subsubsection{IRR: a Practical Issues}
\label{sec:4-7}

\begin{tabular}{c|c|c}
  Year & Cash Flow & Years from Today \\
  \hline
  0 & -\$100 & 0  \\
  1 & \$230 & 1 \\
  2 & -\$132 & 2 \\
\end{tabular}

The joke is that IRR value can be both 10\% and 20\%. In fact, it is possible to
calculate so many values how many times the Cash Flow changes the sign:$$0=-100
+ \frac{230}{(1+IRR)} + \frac{(-132)}{(1+IRR)^2}$$ - square function, 2 roots
(see slides 4-8 for graphics).

The intuitive decision rule when comparing mutually exclusive projects, would be
to accept the project with the highest IRR. This rule is {\bf incorrect} - see
next examples.


\end{document}
