%\documentclass{ncc}
%\documentclass{article}
\documentclass{scrartcl}

\usepackage{amsmath,amssymb,amsfonts} % Typical maths resource packages
\usepackage{graphics}                 % Packages to allow inclusion of graphics
\usepackage{color}                    % For creating coloured text and background
\usepackage{hyperref}                 % For creating hyperlinks in
                                % cross references 

\usepackage{algorithm}
\usepackage{algorithmic}

\usepackage[T2A]{fontenc}
\usepackage[utf8]{inputenc}
\usepackage[russian,english]{babel}
\usepackage{listings}
\lstloadlanguages {[LaTeX]TeX, Octave}
\lstset {language=[LaTeX]TeX,
  extendedchars=true ,escapechar=|}

\newcommand{\example}{\subparagraph{Example:}} % Example
% \newcommand{\def}{\subparagraph{Definition:}} % Definition

\begin{document}
\part{Blah-blah}
\label{par:1}

\begin{itemize}
\item Finance is about value (not money)
\item Value creation is about two key components: time and uncertainty
\item (Almost) anything can be valued: project, human capital etc.
\end{itemize}

Assumptions:
\begin{itemize}
\item Competitive markets
\item Frictions small relative to power of most good ideas (the small changes
  should be sufficient to make the big changes)
\item Capital can flow (relatively) easily
\end{itemize}

\paragraph{Essence of Desicision-Making}
\begin{itemize}
\item (Virtually) every decision involves time and uncertainty
\item Important to understand the impact of just the passage of time on a
  decision
\item At first, assume ``no uncertainty'' to internalise the time value of money
\end{itemize}
Terminology:
\begin{itemize}
\item PV = Present Value (\$)
\item FV = Future Value (\$)
\item n = \# of Periods (\#)
\item r = Interest Rate (\%) > 0 (assumption) - applied to one time period
\end{itemize}
Time line: ``A dollar today is worth more then a dollar tomorrow'' - i.e. time
``by itself'' have some value - so {\bf do not compare money across the time!}

\part{Time Value for Money}
\label{part:2}

\section{PV and FV for Single Cash Flows}
\label{sec:2-1}
In general, {\bf Future Value} = Initial Payment + Accumulated Interest: $$FV =
P + r * P = (1+r) * P$$

P - Payment, r - rate. Then $(1 + R)$ - a {\it future value factor}

For two years: $$FV_2 = (1 + r)^2 * P$$, this is called ``Compounding'' percent.

To calculate FV after 10 years of 7\% year rate for \$500 deposit:

Using Excel: $FV(rate, nper, pmt, [pv], [type])$
\begin{itemize}
\item rate - now 0.07
\item nper - 10 (number of period)
\item pmt - 0 (see later)
\item PV - 500
\end{itemize}
Returns -983.58: negative (some dance about this negative here).

{\bf Present Value} - can be derived from FV by boring
$\frac{FV}{(1+r)^{nper}}$. So the future value becomes lesser now - so this
``phenomenon'' (that something becomes less valuable today comparatively to its
value in future) is called {\bf discounting}.

In Excel: $PV(rate, nper, pmt, [fv],[type])$; pmt is still 0; other are obvious.

\section { PV and FV for Multiple Cash Flows}
\label{sec:2-2}

\subsection{FV of Annuity - Concept}
\label{sec:3-2}

A special case of multiple payments: annuities (C for Cash Flow, PMT for
payment). Classic annuity: loan

\begin{tabular}{c|c|c|c}
  Year & Cash Flow & Years to end: n & Future Value \\
  \hline
  0 & 0 & 3 & 0 \\
  1 & C & 2 & $C(1 + r)^2$ \\
  2 & C & 1 & $C(1 + r)$ \\
  3 & C & 0 & C
\end{tabular}

FV of annuity: formula: $FV = C(1 + r)^2 + C(1+r) + C = \\
= C[(1+r)^2 + (1+r) + 1]$. Finally: $$FV_n = C[(1+r)^{n-1} + \dots + 1]$$

Example: I'm going to put into pension fund 10,000 every year running for 40
years. Rate is 8\%. How much money will it be at the end? (2.59 million)

In Excel, payment (10,000) is a ``pmt'' part of FV function (in out particular
case PV becomes 0).

Example 2: we want to guarantee 500,000 for retirement after 25 years. The rate
is 8\%. How much do I need to invest every year, starting at the end of this
year?

FV is know, so use in Excel: $PMT((rate, nper, pv, fv, hype)$:
\begin{itemize}
\item rate - 0.08
\item nper - number of periods, 25
\item pv - present value, 0
\item fv - future value, 500,000
\end{itemize}
Answer: 6,840

\subsubsection{Annuities: Present Values}

\begin{tabular}{c|c|c|c}
  Year & Cash Flow & Years to Discount: n & Present Value \\
  \hline
  0 & 0 & 0 & 0 \\
  1 & C & 1 & $C/(1 + r)$ \\
  2 & C & 2 & $C/(1 + r) + C/(1 + r)^2$ \\
  3 & C & 3 & $C/(1 + r) + C/(1 + r)^2 + C/(1 + r)^3$
\end{tabular}

Years to discount - because this is so many years the relative sum should be
discounted for today

In the end of $3^{rd}$ period: $PV_3 = \frac{C}{(1+r)} + \frac{C}{(1+r)^2}
+\frac{C}{(1+r)^3} = C \left[\frac{1}{(1+r)} + \frac{1}{(1+r)^2} +
  \frac{1}{(1+r)^3} \right]$

\paragraph{PV of an Annuity: Example 1}
\label{sec:3-6}

How much money do you need in the bank today so that you can spend 10,000 every
year for the next 25 years, starting at the end of this year. Suppose r=5\%.

Solution: PMT=C; and $PV = C [ \frac{1}{(1+r)} + \frac{1}{(1+r)^2} + \dots +
\frac{1}{(1+r)^{25}} ]$

In Excel: $=PV(0.05, 25, 10000)$; answer 140,939

\paragraph{PV of an Annuity: Example 2}
\label{sec:3-6-1}

You plan to attend a business school and you will be forced to take out 100,000
in a loan at 10\%. You need to figure out yearly payments, given that you'll
have 5 years to pay back the loan.

Here: PV = 100,000. We're looking for PMT / C for every year:

In Excel: $PMT(0.10, 5, 100,000)$; answer is 26.380 (!!!)

Validation: PV(0.1, 5, 26380

\paragraph{Example 2: Loan Amortisation} same story

\begin{tabular}{c|c|c|c|c}
  Year & Beginning Balance & Yearly Payment & Interest & Principal Repayment \\
  \hline
  1 & 100,000 & 26,380 & 10,000  & 16,380 \\
  2 & 83,620 & 26,380 & 8,362 & 18,018 \\
  3 & 65.602 & 26,380 & 6,560 & 19,820 \\
  4 & 45,783 & 26,380 & 4578 & 21802 \\
  5 & 23,982 & 26,380 & 2398 & 23982 
\end{tabular}

Next questions: how much do you owe to bank at the beginning of year. Calculate
PV for 3 (remaining) years with payment 26,380.

Another one: what is the sum you have to pay, staying at the beginning of year
have 5 years to pay back the loan? The answer is not the 5 * 26.380, it is
PV(0.1, 5, 26380) = 100,000

The idea is: it is impossible to make money {\it only} by borrowing-landing, so
bank charges additional sums when provides a loan.

\subsubsection{Compounding}
\label{sec:3-8}

Now change story a bit: what are the monthly payments for the same 100,000, 10\%
and 5 years. Plus, what is the ``real'' annual interest rate?

Time line changes, so 5 years become 60 months; rate r becomes $r_{month} =
\frac{0.10}{12}$.

The PV = 100,000; so calculate PMT for 60 months: $=P(0.1/12, 60, 100000)$;
becomes 2,124.7

Actual annual rate (Effective Annual Rate, EAR): calculate sums for one dollar:
$EAR=\left(1 + \frac{r}{k}\right)^k - 1$. Using k=12 and r=0.1, get 10.47\%
which $\gg 10\%$.

\subsubsection{Valuing Perpetuity}
\label{sec:3-9}

A {\bf perpetuity} as simply a set of equal payments that are paid forever, with
or without growth. Examples: Bonds, stocks, etc.

$PV = \frac{C}{r}$ - the PV is the same, so C is the same too. If, however, the
sum should grow over time, the growth rate g is used to measure/calculate the
growth: $PV = \frac{C}{r-g}$

\section{Decision Making}
\label{sec:4}

\subsection{Properties of a good decision criteria}
\label{sec:4-2}

\begin{itemize}
\item Makes sense (benefits exceeds costs) - including non-money values
\item Unit of measurement
\item Benchmark is obvious
\item Easy to communicate (oops)
\item Easy to compare different ideas/projects
\item Easy to calculate
\item Other (?)
\end{itemize}


\subsection{Decision Criteria: NPV}
\label{sec:4-2-1}


{\bf Net Present Value} (NPV) - differs from ``just'' Present Value as we
subtract the expenses (?)

Assume that interest rate is 10\%, what is NPV of the idea?

\begin{tabular}{c|c|c|c}
  Year & Cash Flow & Years to Discount: n & Present Value  \\
  \hline
  0 & -\$1,000 & 0 & -\$1,000  \\
  1 & \$1,320 & 1 & \$1,200 \\
  \hline
  \multicolumn{3}{l} {NPV =}  & \$200
\end{tabular}

% Video 4-3
\begin{itemize}
\item {\bf Cash flow } comes from idea (is an attribute of business idea);
  equals to profit (not income). Responsibility of ``you'' as initiator
\item ``r'' (10\%) comes from ``opportunity cost of investing''; it is a return
  from investing into similar business (competitor?)
\item Final number (NPV) - created value
\item Should you pursue this idea/project - yes, because you're in plus
\item Caution: if you do not have resources to do? Go to market; ``if the idea
  is good, the resources will come'' (?)
\end{itemize}

The essence: Value is always incremental to investment (relative, never
absolute!).

\begin{tabular}{c|c|c|c}
  Year & Cash Flow & Years to Discount: n & Present Value  \\
  \hline
  0 & -\$1,000 & 0 & -\$1,000  \\
  1 & \$1,320 & 1 & \$1,200 \\
  1 & \$1,452 & 2 & \$1,200 \\
  \hline
  \multicolumn{3}{l|} {NPV =}  & \$1400
\end{tabular}

In Excel: $NPV(rate, value1, value2, \dots)$ but value starts from the end of
year 1, so {\it do not give time 0 info there!}.

\subsection{NPV - Properties and Calculation}
\label{sec:4-4}

NPV starts with a negative number (investment): $$NPV = -I_0 + \frac{C_1}{1+r} +
\frac{C_2}{(1+r)^2} + \dots + \frac{C_n}{(1+r)^n}$$ where
\begin{itemize}
\item $I_0$ (or $C_0$) - investment cost of the project
\item $C_i$ - the cash flow in period $i$.
\end{itemize}

Properties of NPV:
\begin{itemize}
\item Makes sense? $B-C (TVM)$
\item Unit of measurement? \$ (1400 NPV)
\item Benchmark obvious? $NPV > 0$
\item Easiest to communicate - yes
\item Easy to compare - compare NPV values, so yes
\item Easy to calculate - {\it no}
\item Other?
\end{itemize}

The main problem with NPV is that it provides very static view of future, and
does not works with flexibility in future.

\subsection{Decision Criteria: Payback}
\label{sec:4-5}

\subsubsection{Payback Period}
\label{sec:4-5-1}

What is the payback of the following idea:
\begin{tabular}{c|c|c}
  Year & Cash Flow & Years from Today \\
  \hline
  0 & -\$1,000 & 0  \\
  1 & \$300 & 1 \\
  2 & \$700 & 2 \\
  3 & \$2000 & 3 \\
  \hline
  \multicolumn{2}{l|} {NPV =}  & \$1400
\end{tabular}

Answer: 2 years (\$300 + \$700 = \$1000)

Better parameter is {\bf Payback period with discounting}.

Properties: \begin{itemize}
\item Makes sense: {\bf no}
\item Unit of measurement {\bf time} - this makes the whole story useless (oops)
\item Benchmark obvious? not too much
\item Easy to communicate? not too much
\item Easy to compare ideas? again, NPV is better
\item Easy to calculate? again, NPV is better
\end{itemize}

The problem with Payback is that it is hungry for ``fast money''. Is considered
as ``bad habit''.

\subsection{Decision Criteria: IRR}
\label{sec:4-6}

{\bf IRR} stands for {\bf Internal Rate of Return}

\begin{tabular}{c|c|c}
  Year & Cash Flow & Years from Today \\
  \hline
  0 & -\$100 & 0  \\
  1 & \$110 & 1 \\
  % \hline \multicolumn{2}{l|} {NPV =} & \$1400
\end{tabular}

Naive answer: 10\% (\$110 - \$100) / year, or $r = \frac{\text{Final Sum} -
  \text{Initial Sum}}{\text{Initial Sum}} = \frac{\text{Profit}}
{\text{Investment}}$

\paragraph{Intuition } What is the NPV of the idea if you use the IRR to
calculate it?$-100 + \frac{110}{1 + IRR} = 0$

This is why this parameter is ``internal'' - no reference to rate (as an
indicator of how other similar businesses are going on)

{\bf Is a good idea?} Not too much as there is no benchmark to compare.

% slides 4-7 - Graphical representation
Example: calculate IRR here:
\begin{tabular}{c|c|c}
  Year & Cash Flow & Years from Today \\
  \hline
  0 & -\$100 & 0  \\
  1 & \$0 & 1 \\
  2 & \$110 & 2 \\
  % \hline \multicolumn{2}{l|} {IRR =} & \$1400
\end{tabular}

Here IRR = 10\% over 2 years. How to solve it (to find per year)? A: ``Make NPV
zero'':$$NPV=0=-100+\frac0{1+IRR} + \frac{110}{(1+IRR)^2}$$

In Excel: $=IRR(values...) = IRR(-100, 0, 110)$

\subsubsection{IRR: a Practical Issues}
\label{sec:4-7}

\begin{tabular}{c|c|c}
  Year & Cash Flow & Years from Today \\
  \hline
  0 & -\$100 & 0  \\
  1 & \$230 & 1 \\
  2 & -\$132 & 2 \\
\end{tabular}

The joke is that IRR value can be both 10\% and 20\%. In fact, it is possible to
calculate so many values how many times the Cash Flow changes the sign:$$0=-100
+ \frac{230}{(1+IRR)} + \frac{(-132)}{(1+IRR)^2}$$ - square function, 2 roots
(see slides 4-8 for graphics).

The intuitive decision rule when comparing mutually exclusive projects, would be
to accept the project with the highest R. This rule is {\bf incorrect} - see
next examples.

\subsection{IRR: Bias}
\label{sec:IRR:bias}
% video 5-2

As it was mentioned, an intuitive rule would be to accept the projects with the
highest IRR. This is incorrect - see these examples.

\begin{enumerate}
\item consider two projects, A and B, with the following cash flows. Which one
  is better?
  \begin{tabular}{|c|c|c|}
    Year & Project A & Project B \\
    \hline
    0 & -\$2,000 & -\$2,000 \\
    1 & \$400 & \$2,000 \\
    2 & \$2,400 & \$625 \\
    \hline
    IRR & 20\%  & 25\% \\
    NPV at 5\% & \$558  & \$472  \\
    NPV at 11\% & \$308    & \$309  \\
    NPV at 20\% & \$0   & \$101  \\
  \end{tabular}

  Naive approach - to pick Project 2 because of earlier income reflected in IRR.
  The problem is we're not comparing IRR to the outside market (rate $r$).
  Actual decision should be made based on rate (see figures for variants).
  % video 5-3

\item Consider two projects A and B again:
  \begin{tabular}{|c|c|c|}
    Year & Project A & Project B \\
    \hline
    0 & -\$5,000 & -\$50,000 \\
    1 & \$7,500 & \$62,500 \\
    \hline
    IRR & 50\%  & 25\% \\
  \end{tabular}

  Another problem with IRR - it ``likes'' the small investments. And again, the
  actual selection should depend on rate $r$.
\end{enumerate}

So, taking decision on IRR only will tend to select the small and short-term
projects.
% video 5-4
Per-ce, IRR does make (some) sense. Unit if measurement (\%) is quite universal,
but not that clear as value. This is why we tend to compare it with rate $r$
(which works as benchmark too). Also it is very communicate by itself (at a
``superficial'' level).

Conclusion: NPV works better, but IRR is in use wider.

\section{Cash Flows}
\label{sec:Cashflow}
% video 5-5

The NPV is defined as $$NPV = -I_0 + \frac{C_1}{1+r} + \frac{C_2}{(1+r)^2} +
\dots + \frac{C_n}{(1+r)^n}$$

where $I_0$ (or $C_0$) is the investment cost of the project and $C_i$ is the
cash flow in period $i$.

\subsubsection*{A snapshot of Project/Firm/You}
\begin{itemize}


\item \begin{tabular}{|c|c|} \hline
    Assets & Liabilities \\
    \hline
    Real Assets &  Equity \\
    & Debt \\
    \hline
  \end{tabular}
  Real assets - ``things'' to generate (create) value in form of cash flow. On
  the ``liability'' side we finance our idea.

\item Time line: for simplicity, let's set 10 years time.
\item All Value is relative (Law of one price)
\item There are two basic ingredients to conducting a valuation:
  \begin{itemize}
  \item Cash Flows: ``belongs'' to project
  \item Cost of Capital, $r$: ``belongs'' to market
  \end{itemize}

\item Sources: pro-for-ma Income Statement and Balance Sheets
\item Income Statement: Year's flows
\item Balance Sheet: A snapshot of assets/stocks.
\end{itemize}
% video 5-6

\subsubsection{Estimating Cash FLows for a Specific Year}

Cash Flows from Projects/Operations.
\begin{itemize}
\item {\bf Revenue} is $P \times Q$ - Price times (say, 5.0 M) Quantity. To
  understand expected price, we make the market analysis
\item {\bf Costs of Goods Sold} : $P \times Q$ (say, 2.0 M)
\item {\bf Selling, General \& Administration costs.}Shown separately because
  they may be non-incremental (it is hard to estimate them on per-project basis)
  - say, 0.5
\item {\bf Depreciation} - this is largely imaginary (not the real number),
  based on appreciation - say, 1 M
\end{itemize}

{\bf Operating Profits}: = Revenues - Costs of Goods Sold, - Selling, General \&
Admin. costs - Depreciation. For us, it is 1.5

Assuming Cash Taxes on Operating Profit as 33\%, calculate its value (becomes
0.5); {\bf Net Operating Profits After Tax} = 1.0 M

Next, we {\bf add} the Depreciation (the same we've subtracted before). The
whole story was done in order to lower the tax on Operating Profit value. So:
Net Operating Profits After Tax + Depreciation - Capital Expenditures

{\bf Capital Expenditures} come NOT from Income Statement but from Balance Sheet
(say 0).

Next item to subtract is {\bf Increases in Working Capital} (say 0.1)

So, {\bf Cash Flow from Operations}: + 1.9 M

Again the word {\bf Profit} means it is calculated for specific year!

% video 5-8
The whole story:

\begin{tabular}{cl|c}
  \hline
  & Revenues & +5 M \\
  - & Costs of Goods Sold & 2 M \\
  - & Selling, General \& Admin. costs & 0.5 M \\
  - & Depreciation & 1 M\\
  \hline
  = & Operating Profits &  1.5 M \\
  - & Cash Taxes on Operating Profit & 0.5 M \\
  \hline
  = & Net Operating Profit After Tax & 1.0 M \\
  + & Depreciation & 0.5 M \\
  - & Capital Expenditures & 0 \\
  - & Increases in Working Capital & 0.1 \\
  \hline
  = & Cash Flows from Operations & 1.9 M \\
\end{tabular}

\subsection{Cash Flows: Important Principle}
% video 5-8 03:10

\begin{enumerate}
\item Estimate all cash flows in an {\bf incremental} basis - i.e. draw two time
  lines: with project (values $B_1 \dots B_n$ and without project $A_0 \dots
  A_n$. Actual info is shown in $B_i - A_i$
\item Do not forget the importance of year 0 and the last year of the chosen
  timeline for the project Examples of {\bf Capital} items (coming from {\it
    Balance sheets}\;!):
  \begin{itemize}
  \item Capital Expenses
  \item Working Capital (cash for business + inventory + account receivables:
    things sold on credit - account payables (payments for credits)) Managing
    inventory (Working Capital) creates value by itself.
  \end{itemize}
\item Accounting issues
  \begin{itemize}
  \item Depreciation - because it is made up
  \item Similar non-cash items
  \end{itemize}
\item Do not mix financing with operations - Please stay on the assets side (C):
  ``care on things which generate values''; stay away from financing
  (liabilities).

\item Include the effects of inflation/deflation. Every item in the accounting
  statement / cash flow statement has a different inflation.
\item Do not compare projects with unequal lives. ``Cheaper machines are usually
  shorter-lived'', so never compare numbers for the costs of different lives. If
  needed - convert into annuity.
\end{enumerate}

% video 6-2
\part{Financing Bonds and Stocks}
\label{part:BondsAndStocks}

\paragraph{Project vs Firm: Timeline again}

Project does have the end: at the end, the whole story is ``undone'', machinery
sold, capital restored etc. The final cost (PV of all expected cashflows after
project end if it would continue - {\bf Terminal Value})

Example: for 10-year project, the last Cashflow was $C_{10} = 20M$. Consider
$C_{11} = C_{10}(1+g)$ - so-called perpetuity (here $g$ is a growth rate).

% Then $\frac{C_{11}}{R-g}$ as )

The firm continues its existence after the project ended - so-called ``ongoing
thing''.

\section{Bonds and Stocks}
\label{sec:BondsAndStocksDef}

All bonds are eventually loans, but actually value creation goes on
projects/ideas/entities etc; finance can not be value-creating per se.

% video 6-2; 11:20
There are two kinds of financing ideas: Bonds/Debt and Stocks/Equity.

The snapshot of Idea/Entity again:

\begin{tabular}{|c|c|} \hline Assets & Liabilities \\ \hline Real Assets -
  create cashflows $C$, require $r$ (cost of capital) & Stock / Equity / Shares
  \\ & Debt / Bonds \\ \hline \end{tabular}

% video 6-3
\subsection{Bonds}
\label{sec:Bonds}

Bonds binds the borrower to the lender in a contract - just like in a loan. It
has the explicit IOU (``I owe you''): a promise to pay money in the future in
return for the money borrowed. Typically bonds pay in two ways: periodically
(so-called coupons, or PMT) and face value (a fixed value at {\bf maturity} - a
moment when the bond contract is over). The deposit contract is essentially a
bond.

Governments issue the simplest type of IOU. Types:
\subsubsection { Zero-Coupon Bonds} (or {\bf Discount Bonds}). This bond has
only one payment, the face value, at maturity.

\example Treasury bill or strip in the US.
% video 6-3 07:00
The price is
$$P=(\text{Face Value}) \times PF(r, n) = \frac{\text{Face Value}}{(1+r)^n}$$
Having $r>0$ we get price $P<F$ - because there is no coupons to compensate a
decrease in value.
% video 6-3 09:00
\example Suppose a zero-coupon bond pays \$1,000 in exactly one year. The price
of value / price of the bond today if the interest rate on similar bonds is 2\%
is $P=\frac{FaceValue}{(1+r)^n} = \frac{\$1,000}{(1.02)^1}=\$980.39$

\paragraph{Yield to Maturity (YTM)} - a return built into the pricing of a bond,
``a return(value) until maturity time''. \example What is YTM of a 10-year
zero-coupon bond with face value of \$1,000 and a current price of \$744.09?
$$P=\frac{FaceValue}{(1+YTM)^n} = \frac{\$1,000}{(1+YTM)^{10}} = \$744.09 \to YTM=3\%$$

One can consider YTM as some sort of IRR.

It is considered that YTM and price are going simultaneously despite the fact
that bonds are sold by value.

What makes the YTM of a zero-coupon government bonds risk-free? Nothing should
happen from ``now'' until a maturity time - so nothing should happen with
government issuing this bond.

In USA, the compounding interval is {\bf six months}. The YTM of the 10-year
\$1,000 zero-coupon bond selling for \$744.09:
$$YTM = \left( \frac{\$1,000}{\$744.09}\right)^{\frac1{20}} - 1 = 1.489\%$$

% video 6-4
\paragraph{Yield Curve} is a relation between the maturity (i.e. the term of
contract) and the YTM of government bonds. Typically - slightly increasing (due
to increasing risk) graph.

\subsubsection{(Government) Coupon Bonds}
The most common type of bond is a coupon bonds. These bonds pay periodic coupons
and a larger face value at maturity. All payments are explicitly stated in the
IOU/contract.

\example A government bond has a 6\% coupon, a face value of \$1,000 and 10
years to maturity. What is the price of this bond given that similar bonds yield
an annual return of $r_a = 6\%$? What if similar bonds yield 4\% or 8\%?

As we have six-months terms, we have 20 periods of payments. So we get \$30
every six months (6\% of \$1000 per year but payed every six months) 19 times
and \$1,030 = \$30 + \$1,000 at the end of period 20.

So the whole price will be of two components: PMT flow (20 payments of \$30 with
$r_{6\;months} = r_a / 2 = 3\%$) and good old PV of \$1,000 in 20 periods (10
years) on $r_{6\;months}=3\%$ {\bf !!! Match the compounding periods !!!}
% video 6-5

In Excel, calculation will be: $P=PV(rate, nper, pmt, fv)= PV(0.03, 20, 30,
1000)$ For coupon (C) equal to rate r (6\%), we have P = \$1,000: the coupon
payments are equal to interest rate payments.

If rate r = 4\%, $P= PV(0.02, 20, 30, 1000)=\$1,163.51$; such situation (PV>Face
Value) is called ``trading at a premium'': coupon payments are higher then
market rate.

If rate r = 8\%, $P= PV(0.04, 20, 30, 1000)=\$864.10$; such situation (PV<Face
Value) is called ``selling at a discount'': coupon payments are lower then
market rate.

``Interest goes up - price goes down''. Zero coupons always sold at a discount;
non-zero coupons depend on coupon values C: (F - face value, P - current price)
\begin{itemize}
\item if $C/F > r \to P > F$
\item if $C/F < r \to P < F$
\item if $C/F = r \to P = F$
\end{itemize}

The graph Price(r) is a hyperbola (see video 6-5, time 07:45). Plus, long-term
bonds are more influenced with changes in rate r, and zero-coupon bonds are more
sensitive to such changes (as all your money are coming at the end of maturity
period).
% video 6-6

\subsubsection{Sources of Risk in Bonds}
\label{sec:SourcesOfRiskBonds}

The uncertainty concerning bond values/prices due to interest rates fluctuations
is known as the {\bf interest rate risk} of a bonds. There are two types of
interest rate risk:
\paragraph{Price risk:} the fact that you do not know how the price (and YTM,
both depend on each other and rate) will change during time until maturity time
(Price Up when YTM Down etc).

\paragraph{Coupon reinvestment risk - CRR:} 
usually coupon is put back on market - so-called ``coupon reinvestment'' - the
fact that YTM changes over time so re-investing coupons can be less productive
then expected (FV of remaining payments changes when rate changes).

\subsubsection{Corporate Bonds }
\label{sec:CorporateBonds}

Corporate bonds almost always pay coupons. They are subject to interest rate
risk.

In addition, they are also subject to {\bf default risk} (Rating agencies, S\&P,
Moody's etc. rate corporate bonds in the US)

Most ``bonds'' are contracts/loans issued by a bank.

\example Suppose you have two identical bonds (same maturities and some coupon
rates), but one is a government bond and the other is a corporate bond. Which
will sell for less?- the one which has more chance of default (usually the
corporate bond).
% video 6-7

\subsubsection{Market Data on Bonds}
\label{sec:MarketDataOnBonds}

wwww.finance.yahoo.com, see bonds section.

\subsection{Stock}
\label{sec:StockEquity}

Stock/Equity is another form of financing. It is different from bond as bond is
a contract while stock is not (and so has no end date and Face Value). The
question on the ``is stock a IOU, really'' is ``maybe'': you're giving money
without promise of some money in return (actually, the bonds (the ``contracts'')
are paid first, the owners of stocks are getting the ``leftovers''):
\begin{itemize}
\item dividend - like coupon in bonds world: money paid regularly
\item selling price (capital gain or loss)
\end{itemize}

So you give in expectation of getting but you do not have contract. Particularly
this works at the start of the business, when there is no information if there
will be cash-flow for contract debts (bonds).
% video 7-3

\subsubsection{Stock Pricing and Valuation}
\label{sec:StockPricingValuation}

\paragraph{Pricing a Stock}

Let $P_0$ be today's stock pricing. Then $P_1$ is the {\bf expecting} stock
price next year. Expected dividend at the end of the year is $DIV_1$.

Going one period at a time (as it is a long-lived asset). Then
$$P_0 = \frac{DIV_1 + P_1}{1+r} $$

Now - {\bf expected} $P_1$. Surprise! $P_1 = \frac{DIV_2 + P_2}{1+r}$ (we're
ignoring risk right now). So for $P_0$:

$$P_0 = \frac{DIV_1 + \frac{DIV_2 + P_2}{1+r}}{1+r} = \frac{DIV_1}{1+r} +
\frac{DIV_2}{(1+r)^2}+\frac{P_2}{(1+r)^2}=\dots$$

All figures are {\bf expectations}!

After similar story for stock being kept for n years:
$$P_0 = \frac{DIV_1 + \frac{DIV_2 + P_2}{1+r}}{1+r} = \sum \limits_{t=1}^n
\frac{DIV_t}{(1+r)^t} + \frac{P_n}{(1+r)^n}$$

After expanding to extremely large $n$: as $n \to \infty$, $P_n$ part goes to 0:

$$P_0 = \sum_{t=1}^\infty \frac{DIV_t}{(1+r)^t}$$

- when you sell stock you get the expected price of its dividends.

% video 7-4

\paragraph{Dividend Growth Stocks}

\example Suppose dividends expected to remain approximately constant {\bf
  Dividend Stock}. What is the price?

This is similar to regular payments (cash-flows): $P_0 = \frac{C}r =
\frac{DIV}{r}$ - as a limit for $n \to \inf$

I.e. if some company pays \$0.50 per share for the foreseeable future and the
return of a business is 10\% (something like over-regulated utility company),
the stock price should be $P_0 = 0.5 / 0.1 = \$5.00$ ({\bf perpetuity}).

But, assume that this company paid the dividends for 30 years. Then we have {\bf
  annuity}: $=PV(0.1, 30, 5.0) = \$4.71$.

The difference: $\$0.29 = PV(0.1, 0.50, year 31 \to \infty)$ - so perpetuity
gives a good estimation.

\example Suppose dividend are expected to grow at a rate of $g$ per year ({\bf
  Growth Stock}). What is the stock price?

$$P_0 = \frac{C_1}{r-g} = \frac{DIV_1}{r-g}$$

So for company which is expected to pay \$20 per share next year and the
dividends are expected to grow indefinitely at a rate of 5\% per year. Stocks of
similar firms are earning an expected rate of return of 15\%. The price is
$P_0=\frac{DIV_1}{r-g} = \frac{\$20}{0.15-0.05} = \$200$
% video 7-5

\subsubsection{Engine of Growth (Assume Perpetuities)}
\label{sec:EngineOfGrowth}

Definitions:
\begin{itemize}
\item {\bf ICPS - Invested Capital Per Share}; $ICPS = CAPITAL/SHARE$ - as
  stocks are traded ``per share''
\item {\bf ROI - Return Of Investment} > 0;ROI is a different word for ROR (Rate
  of Return).
\item {\bf EPS - Earnings Per Share}; $EPS = CFPS = ICPS \times ROI$ ({\bf CFPS
    - Cash Flow Per Share}, but sometimes these terms are used for different
  meanings)

  It is a Real Assets which generate CF, but {\bf cash flows is completely
    separated from the financing}.

\item {\bf RE - Retained Earnings}, $RE = CFPS - DIV$; a part of CF which was
  not paid out as dividend but retained to provide growth (the ``Engine of
  Growth'') . So reinvesting is characterised as $b=\frac{RE}{EPS}$.

  This money are going ``back'' as (re)investment in the same ideas, so
  increasing assets. This makes sense if you have ``great ideas'' (like
  Microsoft for the first few years - they did not pay any dividends, just
  re-invested in business growth).

  So getting dividends is not always a good thing - ``it depends''; as paid
  dividends mean the company does not grow ``in full potential''.

  So the {\bf growth rate} $g = b \times ROI$ - ``the engine of growth''.
\end{itemize}
% video 7-6
\subsubsection{Valuation}
\label{sec:StockPricingValuation}

The price of a firm's stock can be divided up into
$$P_0 = \frac{EPS}{r} + PVGO,$$
where $EPS$ - cash flow per share, $PVGO$ is a $PV$ of growing opportunities.

Why? Because $P_0 = PV(DIV_s)$ - price is a PV of future dividends. Consider two
cases:
\begin{itemize}
\item no growth $P_0 = \frac{EPS}r = \frac{DIV}r = \frac{EPS}r$
\item growth - no dividends paid: as seen before $P_0 = \frac{DIV_1}{r-g}$. So
  $PVGO = \frac{DIV_1}{r-g} - \frac{DIV_1}{r}$
\end{itemize}

So the growth opportunities reflect value.

\example Suppose you know that Macrosoft:
\begin{itemize}
\item is expecting to earn 10\% on existing assets
\item has \$60 of capital per share
\item the {\bf market capitalisation rate} (another word for $r$) is 12\%.
\item the company is not planing to grow
\end{itemize}
% video 7-7
Then shares should be traded at $P_0 = \frac{DIV}{r} = \frac{EPS}{r} = \frac{CPS
  \times 0.1}{0.12} = \frac{\$6}{0.12} = \$50$

\example Now suppose that company announces a growth policy that {\bf plows
  back} (or $RE$ when measured in \$, or $b$ when measured in \$) 70\% of its
earnings every year at the same ROI as its existing assets.
\begin{itemize}
\item what growth rate will this policy generate?
\item what will happen to share price?
\end{itemize}

Having $EPS = \$6$ (see above), growth rate $g = b \times ROI = 0.70 \times 0.10
= 0.07$. Then $P_0 = \frac{DIV_1}{(r-g)}=\frac{0.3 \times \$6}{0.12-0.07} =
\frac{\$1.80}{0.05} = \$36$ - compared to \$50. As $IRR < r$, we're throwing
money into negative growth (see next section).

% video 7-8
\subsubsection{Good or Bad Growth}
\label{sec:GoodOrBadGrowth}

Growth in earnings is generated by investing into ideas which make $ROI > r$.
Sometimes IRR is inflated by short-term management. So the ``good'' growth is
the growth in assets, not just cash flows.

\example Macrosoft again. Alternative growth policy that plows back ($RE$) 50\%
of its earning every year at an $ROI$ of 14\% (they have better ideas).
\begin{itemize}
\item what growth rate will this policy generate? $g = b \times ROI = 0.5 \times
  14\% = 7\%$
\item what will happen to share price? As $ROI > r$, price $P_0 =
  \frac{DIV_1}{(r-g)} = \frac{50\% (\$6)}{(0.12-0.07)} = \frac{\$3}{0.05} =
  \$60$ - 20\% increase over initial variant.
\end{itemize}
The value is in stock price, not in CF!
\begin{itemize}
\item Emphasis on growth, but it is a flow
\item Could be ``good'' or ``bad''
\item Same applies to the whole nation. GNP is a {\bf flow}!
\item Same applies to the whole world
\end{itemize}
% video 7-9

\subsubsection{Stock Market Data}
\label{sec:StockMarketData}

Like morningstar.com; but go to good old finance.yahoo.com

Means: Market Cap - $price \times NumberOfShares$; $EPS$ is {\bf not} CF per
share.

See ``Key Statistic'' section

% video 8-1
\subsection{Risk \& Return}
\label{sec:Risk}

\paragraph{Why Risk Return} There are two basic ingredients to conduct a
valuation
\begin{itemize}
\item Cash Flows: they belong to the company/idea/project.
\item Cost of Capital, $r$: belongs to market (or competition).
\end{itemize}

There are two main sources of financing any idea: Bonds/Debt and Shares/Equity.
Now we'll devote some time to the ``cost of the capital'' Different
ideas/projects have different risks. Most people are risk-averse, and hence the
returns on ideas should be different depending on their risk. Therefore, we need
to measure risk and understand a return of different ideas/projects/firms.

In our examples we're measuring ``oranges'' which is a name for
idea/project/firm that is launching a u-phone, u-pad etc. We've projected all
cash flows. How would we determine the cost of capital? Things to remember:
\begin{itemize}
\item all valuation is relative
\item law of one price (same business - same risk).
\end{itemize}

% video 8-3
In order to evaluate CF, we compare our assets with similar assets which are
present on the same or similar market - to find out their return on assets. I.e.
imagine we have a company (Apple) which produces something similar - so we can
go at marketplace with hope that such company:
\begin{itemize}
\item exists
\item is publicity traded
\end{itemize}
% video 8-3 04:22
We still do not know the Apple's assets, but we have info about their
liabilities (equity and debt).

Let's assume the Apple's debt is 0 (it has only equities). Then we can find the
Risk and Return values for Apple's equity, and this have to be identical to the
Apple's {\bf real assets risk/return} (this is under assumption that there is no
debt!!!). See Yahoo Finance for such information.

% video 8-4
All boils down to
\begin{itemize}
\item Measuring the risk of a "comparable"
\item Determining the return that would compensate investors for that risk
\end{itemize}
For simplicity, assume we have only equity (no debt) for now.

\paragraph{What is Risk}

Tomorrow can be Good (with a chance of $P_G$), Normal (with probability $P_N$)
or Bad ($P_B$).Obviously $P_G + P_N + P_B = 1$ (no hidden variants should be
left) - so-called {\bf States of Nature (SoN)}.

The Treasury Bill is risk-less because we believe their repayment will be
provided regardless of SoN. The identical bond issued by an average corporation
is considered more riskier because $P_B$ is now considered >0. There are two
fundamental sources of risk:
\begin{itemize}
\item Economy - wide/Macro/Systematic - like ``big oil shock'', interest rate
  changes (things which impacts almost everybody).
\item Specific/Idiosyncratic/Unique (technology-specific to your product, like
  the management of a company) - something specific to your company.
\end{itemize}

% video 8-5
\subsubsection{Statistics}
\label{sec:Statistics}

Introduced the normal distribution (probability vs value). The average is named
also ``Mean'', or ``Median'', or ``Mode''. If our distribution is of parameter
$y$, the mean (also called expectation) is labelled as $Mean = \bar y = E(y) =
\sum \limits_i^n P_i y_i = \frac{\sum y_i}n$ (assuming $y_i = 1/n$ - assumption
are equal for all values).
% video 8-5 08:06

Variance (uncertainty): $Variance = \sum p_i (y_i - \bar y)^2 = \sum
\sigma_i^2$. Standard deviation: $\sigma_i = \sqrt{\sigma^2}$.

Adverse to risk $\to$ portfolios (collection of things) $\to$ relationships.

% video 8-6
\paragraph{Measuring Relationships}
Assume two values: $y$ (like bushels of corn) and $x$ i.e. inches of rainfall) -
both normally distributed.

To find out relation, check ``how does $y$ change when $x$ below mean?''. We
calculate $Covariance = \sum P_i (y_i - \bar y)(x_i - \bar x)$
% video 8-6 06:00
- shows the tendency of ``corn behaviour''depending of ``raindrop behaviour''.
Or ``how does $x$ deviate from mean when $y$ deviates from its mean''?

Two problems:
\begin{itemize}
\item magnitude is not communicative: $\sigma_{yx} = \sum p_i (y_i - \bar y)(x_i
  - \bar x)$ - only sign makes sense
\item unit dependent
\end{itemize}
To fix it, we take $Corellation = P_{x, y} = \frac{\sigma_{yx}}{\sigma_y
  \sigma_y}$. It ranges in [-1 \dots 1].

% video 8-7
\paragraph{Regression}

Standard formula for linear regression: $y_i = \alpha + \beta x_i + \epsilon_i$.
It can be based on the ``scatter plot'' by drawing the line on graph.
Parameters:
\begin{itemize}
\item $\alpha$ - value of $y$ where $x=0$
\item $\beta$ - slope (if $x$ changes by 1\%, how much $y$ change?). $\beta =
  \frac{Cov(\delta x, \delta y)}{Var(\delta x)}$ - can be any number.
\item $\epsilon_i$ - a level of ignorance (on average it should be 0) about
  ``what drives why''.
\end{itemize}

% video 8-8
\subsubsection{Diversification}
\label{sec:Diversification}

Portfolio which is ``just'' large but not diversified (i.e. containing only
high-tech assets but not diversified across areas) does not make much sense - it
is still not diversified enough. The diversification itself makes sense only if
there is a positive relation between risk and return (and individual securities
are much more risky then portfolio): $\sigma_P \ll \frac1n \sum \sigma_i$. This
whole phenomenon (that the average risk of the components of the portfolio is
much higher then the combined risk of portfolio) is called {\bf
  diversification}. 

Risk aversion (almost) automatically means portfolio. Small stocks (stocks of
small companies) are more risky while more returning. Next - S\&P 500 (companies
from S\&P 500 list).

{\bf Strips} - zero-coupon bonds with term longer then 1 year.

{\bf Risk Return} - a difference between risky and ``non-risky'' portfolio. 

\paragraph{Diversification: 1 asset}
%video 9-2
Suppose we hold one asset/security in portfolio; let it be Google. The risk of
Google securities is $\sigma_{GOOG}^2$ (STDEV on the Google's returns).

The risk of whole portfolio: $\sigma^2_p (\text{or }\sigma_p) =
\sigma^2_{GOOG}\text{or }\sigma_{GOOG}$
%video 9-2 05:20
- bad idea, ``all eggs in the same basket''.

\paragraph{Diversification: 2  assets}

- say, Google and Yahoo. Risk of each is $\sigma^2_{GOOG}|\sigma_{GOOG}$ and $\sigma^2_{YAHO}|\sigma_{YAHO}$

The average risk is $\frac12 \sum (\sigma_{GOOG} + \sigma_{YAHO})$ 
- assuming equal is invested in both. 

But the risk of a portfolio is a bit more interesting: $\sigma_p = $
definitions:
\begin{itemize}
\item $\sigma_p$ - variance
\item $cov_{a, b} = \sigma_{a, b}$ - covariance (aha, labelled with $\sigma$ -
  but with two sub-indexes)
\item $\rho_{a, b} = \frac{ \sigma_{a, b}}{\sigma_a \times \sigma_b}$ -
  correlation: $-1 \leq \rho_{a,b} \leq 1$
\item $x_a = \frac{\text{amount of investment in a}}{\text{total investment}}$ -
  proportion in $a$.
\end{itemize}
Return of a portfolio:
 $$R_p = x_a R_a + x_b R_b$$ -  linear function on proportion and return.

So the risk of the two assets portfolio:
$$\sigma^2(R_p) = \sigma^2_p = x_a^2\sigma_a^2 + x_b^2\sigma_b^2 + 2 x_a x_b
\sigma_{ab} =  x_a^2\sigma_a^2 + x_b^2\sigma_b^2 + 2 x_a x_b
\rho_{ab}\sigma_a \sigma_b$$

%video 9-3 - blah-blah
If the correlation is perfect ($\rho_{ab} = 1$), the main risk comes from {\bf
  market} (not company-specific) - !!!

%video 9-4
The risk of the two-security portfolio is affected by four factors: two standard
deviations and two relations (correlations).  

For three assets: say, Google, Boeing and Merck (pharmaceutical company).
Measuring risk of each: $x_a^2 \sigma_a^2 + x_b^2 \sigma_b^2 + x_c^2\sigma_c^2 +
2x_a x_b \sigma_{ab} + 2x_b x_c \sigma_{bc} + 2x_a x_c \sigma_{ac}$, or 
$$\sigma^2(R_p) = x_a^2 \sigma_a^2 + x_b^2 \sigma_b^2 + x_c^2\sigma_c^2 + 2x_a
x_b\rho_{ab} \sigma_a\sigma_b + 2x_b x_c \rho_{bc} \sigma_b\sigma_c + 2x_a x_c
\rho_{ac} \sigma_a\sigma_c$$ 

Total amount of factors for 3-asset portfolio: $9 = 3^2$ (3 sigmas and 6
relations). 

\paragraph{Bottom line on diversification}
\begin{itemize}
\item Variance of an asset determines risk faced by you only if you hold one (or
  very few) assets: you face both systematic and specific risks of that/those
  asset(s).
\item Variance of any assets is unimportant when held in a large (assets > 30)
  portfolio
\item Specific risks are diversified (cancel each other); Only relations among
  the assets matter
\item An asset's contribution to a portfolio's risk is determined by its
  relations with all other assets in the portfolio
\item All relations are due to common/systematic/market (not specific!) affects
\end{itemize}

% video 9-5
Empirically, starting from (around) 30 assets in portfolio, increasing amount
further does not affect the risk seriously. However, most of us invest in mutual
funds or existing portfolio (professor recommends Vanguard). 

%video 9-5 09:30
Risk of an idea/project/asset has to measure its relations with other assets.
Using a large diversified market portfolio (like S\&P or Russel 2000) we can
capture (almost) all relations (so-called {\bf market model}:
$$\tilde r_i = \alpha_i + \beta_i\tilde r_m + \tilde \epsilon_i$$
\begin{itemize}
\item $\tilde r_i$ - return of $i$ equity
\item $\tilde r_m$ - return of market ``as a whole''(i.e. large portfolio like
  S\&P 500) 
\item $\beta_i$ - regression (the sensitivity) of $i$ asset to the entire
  market: $\beta_i = \frac{\sigma_{im}}{\sigma^2_m} = \frac{cov(R_iR_m)}
  {Var(R_m)^2} $- covariance of Return of our asset with the return of market
  normalised on return of market. 

It captures all relationships between our asset and all other assets on market.
Slang name - {\bf beta}; idea is that risk is just this - a correlation between
our asset and whole market.
\item $\alpha_i$ - ???
\item $\epsilon_i$ - captures all the unique (specific) staff
\end{itemize}

% video 9-6

\paragraph{Capital Asset Pricing Model (CAPM)}

The relationship between risk (beta) and return is linear, with the following
form:
$$r_i = r_f + (r_m - r_f)\beta_i$$
\begin{itemize}
\item $r_f$ - ``risk-free'' return (like in long-term government bonds). For it
  $\beta_f = 0$
\item $r_i$ - {\bf expected } rate of return on the equity of project/idea/firm
  $i$
\item $r_m$ - {\bf expected} rate of return on the ``market'' portfolio
\item $(r_m - r_f)$ is the average market risk premium
\item $\beta_i$ - a risk of comparable equity
\end{itemize}
- any return can be broken in two parts: risk-free and risk-specific. The second
one is average market risk premium (is considered about 7\% in USA usually, prof
considers real figure is about 5\%).

So two points: one for risk-free (beta is 0), one for ``market'' (beta is 1),
$r_i = r_m$. Draw straight line and enjoy.

The intuition is that it is based on intuitive and simple idea. But not all
assets fit this model.

% video 9-7
Up to now we based everything on assumption that debt is 0. Let's try to value
``Orange'' vs ``Apple'' (back to the Beginning). We need to minimise a
measurement error - so in real life we need more then one asset to compare
(using just ``Apple'' is not sufficient - use something else to cut off noise).

Use Yahoo Finance to measure cost of capital for ``Orange''. Get a good
reflection of market (say, S\&P 500). Get source for $r_f$ - like 10-year
treasury bonds (do {\bf not} look at past averages!).

Now accept risk premium (5\%) and go for our $\beta_i$ - it is calculated in
yahoo finance. Also one can go to ``Key statistic'' and look for Total Debt (if
not 0 - the equity and assets are not the same).  

\subparagraph{Putting it Together}

Debt is 0, so $\beta_{assets} = \beta_{equity}$. Looking at Apple (as comparable
to Orange), we get $\beta_a = 0.94$ (from Yahoo Finance; for more serious
analysis use more sources - like morningstar). 

Next - calculate CAPM: expected return 
$$R^{APPLE}_e = \underbrace{R_f}_{1.45\%} + \underbrace{[R_m - R_f]}_{7\%}
\underbrace{\beta_e^{APPLE}}_{0.94} = 8.5\%$$ 
Here we got return on assets $\to$ can calculate cost of capital (as Debt is 0).

% video 10-2
\subsubsection{Valuation}
\label{sec:StockPricingValuation}

\paragraph{Cost of Capital WACC}

All published betas (and the ones which can be used for CAPM  are {\bf betas of
  equity} (not debt!). But the world has Equity and Debt (a firm/project/idea
could be financed by equity and debt (or hybrid). These capture most important
aspects of financing and evaluation.

Leverage means that the firm uses debt in its ``capital structure''. In a world
with competitive markets and no frictions, financing (or the mix of equity and
debt) has no effect on the value of a idea/project/firm (Modigliani - Miller). 

Value is created by the idea (real asset) and its ability to generate cash
flows. The cost of capital. or the return on asset, is determined by the
inherent (market) risk (beta) of the business (real access), not by ow it is
financed. 

Under perfect capital markets. $E(R_a)$ - estimation of return on assets is just
the weighted average of the equity and debt cost of capital, or the {\bf weighted
average cost of capital} (WACC):

$$WACC = E(R_a) = \frac{D}{E_L + D}E(R_d) + \frac{E_L}{E_L + D}E(R^L_e)$$
where
\begin{itemize}
\item $E(R_a)$ -  required rate of return on debt (estimated)
\item $E(R_e^L)$ - required rate of return on the {\it levered} equity of the
  firm
\item $D$ - weight of debt
\item $E_L$ - equity (?)
\end{itemize}
here $E()$ means estimation and $L$ means ``leverage'' 
% video 10-2, 10:20

% video 10-3
\paragraph{Leverage Risk}
As return on assets is the same for the situation ``no debt'' and ``financing
includes debt'', the cost of capital changes (see formula for WACC). Then the
expected rate of return on equity of a levered firm increases in proportion to
the debt-equity ration ($D/E$), expressed in market values (re-focusing the WACC
formula):
$$E(R^L_e) = E(R_a) + \frac{D}{E_L} [ E(R_a) - E(R_d)]$$
- as you take on debt, the expected return on equity goes up - because you take
on more risk. Similarly, the {\bf Risk of Equity}:
$$\beta_e^L = \beta_a + \frac{D}{E_L}(\beta_a - \beta_d)$$
\begin{itemize}
\item $\beta_a$ - Business Risk
\item $\beta_a - \beta_d$ - Financial Risk
\item if $D=0$ (i.e. firm has no leverage), $\beta_a = \beta_e$
\item if $D/E$ is positive (i.e. firm is {\it levered}), then $\beta_e >>
  \beta_a$ due to financial risk
\end{itemize}
- ``whenever you consider return, you should consider risk''

\paragraph{Business Risk and Financial Risk}

{\bf Business risk} is the risk of the firm's assets; the riskiness inherent in
their operations. It is also the risk of the equity of an all equity firm.

{\bf Financial risk} is the {\bf additional risk} place on {\bf equity} as a
result of the firm's decision to use senior fixed-income securities - i.e.,
debt. So ``debt is a promise, and you pay it first'' - which makes equity more
risky. 

Payment to equity holders = (project cash flow) - (amount owed on fixed
borrowing). But WACC is always the same - just when $\frac{D_L}{E_L}$ rises,
expected return on equity ($e_E$) rises too - and so rises return on financing
($r_D$). As a result, WACC (which is {\it weighted average} cost of capital)
stays the same.  

%video 10-4
Return on equity $r_E$ will always be more or equal to return on assets $r_A$,
because $\beta_E > \beta_A$ 
\begin{itemize}
\item To evaluate investment project, we need to find the $\beta$ and return of
  the assets, i.e. the risk and return of the assets generating the cash flows
\item Since the $\beta$ of assets are not observable, we take the $\beta$ of
  equity and remove the effects of financial leverage of the firm ({\it
    unlever}) to get an estimate of beta asset.
\end{itemize}

%video 10-5
\subsubsection{Valuation Mega Example}
\label{sec:MegaExample}

\begin{itemize}
\item Online, Inc is a video gaming company with no debt. Recently their stock
  has been added to NASDAQ and it has an (equity) $\beta_e^V = 1.50$ - ``V''
  for Video Games
\item Online, Inc is considering investing in the software business, The
  business project will require an initial investment of \$50,000,000. If
  undertaken, the video gaming business will represent 25\% of Online, Inc.'s
  assets. 
\item  There is 50\% chance the project will generate an annual payoff of
  \$7,000,000 forever, a 40\% chance of an annual payoff of \$5,000,000 forever,
  and a 10\% chance that the project will fail and generate no cash flows.
\item Companies solely in the software business have an equity beta of 1.40.
  These firms have a debt/equity ration of 0.25 (on average) and the risk-less
  debt. Online, Inc. is forecasting that the average market risk premium is
  about 5\% and the risk-free rate on a long-term bond is 4.5\%

So $\beta_e^{SOFT} = 1.40$, but $\frac{D}E = 0.25$. As debt is risk-less,
$\beta_d = 0$. For CAPM: $R_f = 4.5\%$; $(R_m - R_f) = 5\%$.
\end{itemize}
%video 10-6
Example questions:
\begin{enumerate}
\item What is Online Inc.'s {\bf cost of capital} before undertaking the project?

Debt is 0, so 
$$WACC = R_a^V (= R_e) = R_f + (Rm-Rf)\beta_e = 4.5\% + 5\% \times 1.5 = 12\%$$
%video 10-8
\item What is IRR of the new project?

Use cash-flows: $I_0 = 50M$, but cash-flows are actually ``expected'':
$$E(C) = \frac12 (7M) + 0.4(5M) + 0.1 \times 0 = 5.5$$ 
- actually it is bouncing up-down, and never equals to 5.5! 

So, for perpetual:
$$IRR = \frac{C}{I_0} = 5.5/50 = 11\%$$
(to check: calculate present value $PV = C/IRR = 5.5/0.11 = 50M$, which is equal
to Investment - checked!

IRR in isolation does not much sense, but here it is quite simple (cash-flow
changed sign only once, and there is single project - no other projects to
compare, where IRR causes problems), so everything is fine for now.

%video 10-10
\item Should Online, Inc. take the new project?

The cost of capital Online ``standalone'' (as-is) is 12\%, the IRR if a new
project is 11\%. 

Answer: do not know. IRR looks less then present CoC ($R_a$), but {\bf IRR is
  calculated for different business then $R_0$!}. So we need to find a cost of
capital for software: $R_a^{SOFT}$ - to compare similar risks and returns. 

%video 10-12
\item what is the hurdle value (cost of capital) for the project?

So $\beta_a = \beta_e(\frac{E}{D+E}) + \beta_d(\frac{D}{D+E}) \to \beta_e =
\beta_a + \frac DE(\beta_a - \beta_d)$. 

We have data for: $\beta_e^S = 1.40 = \beta_a^S + 0.25(\beta_a^S - 0)$. Here
$\beta_d = 0$ as debt is riskless (usually not true, typically $0.1 < \beta_d <
0.4$). Then $\beta_a^S = 1.40/1.25 = 1.12$ - beta for the software business. 

So {\bf cost of capital}: $R_a^S = R_f + [R_m - R_f] \beta_a^S = 4.5\% + [5\%] 1.12
\approx 10.1\%$

% video 10-14
\item Should Online, Inc. take the new project (2)?

IRR is 11\%, cost of capital for software business $R_a^S = 10.1\%$ - so $IRR >
R_a$, answer is {\bf yes}.

Separate question is how difficult it is to come with idea/project which will
beat the companies already on the market. 

% video 10-15 
\item If Online, Inc has 1 million shares, what will happen to its stock price? 

The NPV of a decision is $NPV = -I_0 + PV(C_1, C_2 \dots) =-I_0 + \frac
C{R_a^S}= -50M + \frac{5.5M}{0.101} \approx -50M + 55M = +5M$

So Share price should go up by $S_{up} = \$5M/1M = \$5$

% video 10-17
\item What is Online Inc. new cost of capital? Is this good or bad news?

Having $R_a^V = 12\%, R_a^S = 10.1\%$ and $\beta_a^V = 1.50; \beta_a^S = 1.12$
(BTW, high risk - high return works here fine).

Then $R_a = 12\%(0.25) + 10.1 (0.75) = 3\% + 7.5\% = 10.5\% $ - as software
business makes 75\% of the whole business. So cost of capital dropped from 12\%
to 10.5\%, which is {\bf NOT} bad - because as return drops, so drops risk:
$\beta_a = 1.50 (0.25) + 1.12 (0.75) = $<less then initial 1.50>.

So the change is {\bf neither good nor bad}.
\end{enumerate}

% video 10-18
A lot of firms use their own cost of capital in evaluations. For the case,
General Electric (GE) uses its own value for COC to evaluate {\bf ALL} projects.
This is dangerous as lead to comparing returns only without assuming risks - so
you tend to use high-risk projects.

Two reasons why capital structure might affect firm value in the {\bf real
  world}: 
\begin{itemize}
\item Capital structure policy may result in the realisation of tax benefits due
  to the interest tax shields provided by debt at the corporate level


\item The capital structure policy adopted by a company's management may lead to
  costs incurred due to the presence of debt, especially if the firm is in
  financial distress.
\end{itemize}

The weighted average cost of capital (WACC) of a firm {\bf with corporate taxes}
becomes 
$$WACC = \frac D{V_L}(1-T_c)E(R_d) + \frac{E_L}{V_L} E(R_e^L) $$

where $T_c$ - corporate tax rate. Then $(1-T_c)$ makes serious influence on the
final figures (WACC goes down in comparison to the ``pure'' calculation) -
because Government decides to let you deduct the interest as expense so debt
becomes attractive (artificially!).

The tax-deductibles of interest on debt implies that firms should have 100\%
leverage, but the evidence shows that firms do not borrow even close to 100\%
for two potential reasons:
\begin{itemize}
\item Personal taxes may favour equity
\item Bankruptcy-related costs
\end{itemize}
 
\end{document}
