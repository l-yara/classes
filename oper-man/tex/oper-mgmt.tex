%\documentclass{ncc}
%\documentclass{article}
\documentclass{scrartcl}

\usepackage{amsmath,amssymb,amsfonts} % Typical maths resource packages
\usepackage{graphics}                 % Packages to allow inclusion of graphics
\usepackage{color}                    % For creating coloured text and background
\usepackage{hyperref}                 % For creating hyperlinks in
                                % cross references

\usepackage{algorithm}
\usepackage{algorithmic}

\usepackage[T2A]{fontenc}
\usepackage[utf8]{inputenc}
\usepackage[russian,english]{babel}
\usepackage{listings}
\lstdefinelanguage{scala}{
  morekeywords={abstract,case,catch,class,def,%
    do,else,extends,false,final,finally,%
    for,if,implicit,import,match,mixin,%
    new,null,object,override,package,%
    private,protected,requires,return,sealed,%
    super,this,throw,trait,true,try,%
    type,val,var,while,with,yield},
  otherkeywords={=>,<-,<\%,<:,>:,\#,@},
  sensitive=true,
  morecomment=[l]{//},
  morecomment=[n]{/*}{*/},
  morestring=[b]",
  morestring=[b]',
  morestring=[b]"""
}
%\lstset {language=scala}
\usepackage{color}
\definecolor{dkgreen}{rgb}{0,0.6,0}
\definecolor{gray}{rgb}{0.5,0.5,0.5}
\definecolor{mauve}{rgb}{0.58,0,0.82}

% Default settings for code listings
\lstset{frame=tb,
  language=scala,
  aboveskip=3mm,
  belowskip=3mm,
  showstringspaces=false,
  columns=flexible,
  basicstyle={\small\ttfamily},
  numbers=none,
  numberstyle=\tiny\color{gray},
  keywordstyle=\color{blue},
  commentstyle=\color{dkgreen},
  stringstyle=\color{mauve},
  frame=single,
  breaklines=true,
  breakatwhitespace=true
  tabsize=3
}


\newcommand{\example}{\subparagraph{Example:}} % Example
\newcommand{\term}[1]{\verb~#1~} % Definition
\newcommand{\video}[1]{} % Definition

\begin{document}
\section{ Ch 1}
\label{sec:01-01}
Processes: the three basic measures:
\begin{itemize}
\item {\bf Flow rate / throughput} number of flow units going through the process per unit of time {\em Flow Unit / time}
\item {\bf Flow Time} time it takes a flow unit to go from the beginning to the end of the process
\item {\bf Inventory} the number of flow units in the process at a given moment in time. Happens where supply mis-matches the demand (!).
\item {\bf Flow Unit} Customer, Sandwich, Money, Cheese etc.
\end{itemize}
Process management is about the processes which are done repeatedly (in contrast to Project management) \\
Vocabulary:
\begin{itemize}
\item {\bf Processing time} (Activity time): how long does worker spend on the task? in {\em time / unit}
\item {\bf Capacity} = 1/processing time: how many units can the worker make per unit of time if there are $m$ workers at the activity: $Capacity = m / processing time$
\item {\bf Bottleneck} process step with the {\em lowest capacity}
\item {\bf Process Capacity} capacity of the bottleneck
\item {\bf Flow Rate} $=min(Demand rate, Process Capacity)$
\item {\bf Utilisation} $=Flow Rate / Capacity$
\item {\bf Flow Time}: the amount of time it takes a flow to go through the process
\item {\bf Inventory}: the number of flow units in the system
\end{itemize}

\section{Module 3: Variety}
\label{sec:05}
\subsection{Reasons for  Variety}
\label{sec:05-01}
\begin{itemize}
\item {\bf Horizontal differentiation}: each customer has a utility (preference function) maximising, say, shirt size. The further you go away from that point, the lower is the utility.
\item {\bf Vertical differentiation}: customer differ in their valuation of quality (performance): screen resolution, mpg, processor speed and / or their ability to pay.
\item {\bf Taste based variety}: taste for food, music, artists. Rugged landscape, often vary over time.
\end{itemize}

\subsection{Capacity Calculations with Set-Ups}
\label{sec:05-02}
$$\mbox{Capacity given Batch Size} = \frac{\mbox{Batch Size}}{\mbox{Set-up time} + \mbox{Batch-size} * \mbox{Time per unit}}$$

\subsection{Mixed Model Production}
\label{sec:05-03}

The downside of Large Batches:
\begin{itemize}
\item lead to more inventory in process
\item needs to be balanced with the need for capacity
\end{itemize}
Smaller batches lead to smaller inventory - in extreme cases the batch size is 1. This is called {\bf Mixed model production}, or Heijunk (?) in Toyota system.
\begin{itemize}
\item Product A: demand is 100 units per hour
\item Product B: demand is 75 units per hour
\end{itemize}
Production line can produce 300 units per hour; 30 minutes to switch. How to set the batch size?
$$p = \frac1{300} \frac{h}{unit}; S = 0.5h$$
The whole batch will have two set-ups and periods for producing A and producing B; so total switch time will be
$$S_{total} = 1h$$
Target flow is
$$175 \frac{units}{h} = \frac{B}{S + B \times p} = \frac{B}{1 + B\times \frac1{300}}$$
Solving equation: $B = 420$ (total units). Applying ratio:
\begin{itemize}
\item Product A: $B_A = 420 \times \frac{100}{100+75} = 240$
\item Product B: $B_B = 420 \times \frac{75}{100+75} = 180$
\end{itemize}

\subsection{Choosing a Batch Size}
\label{sec:05-04}
The idea is to balance the batching flow with next bottleneck: wasting a bit of capacity to lower inventory is usually a good idea. But, if batching operator is a bottleneck, ``stopping for a set-up is stopping the whole line''.

\subsection{Demand Fragmentation and Pooling}
\label{sec:05-05}
If our prediction for demand of Something Model 1 is $\mu_1, \sigma_1$than coefficient of variation is $CV_1 = \frac{\sigma_1}{\mu_1}$. Similar story for Model 2.\\

If our concurrent can unite both models, and roughly $\mu_1 = \mu_2 = \mu; \sigma_1 = \sigma_2$, than:
\begin{itemize}
\item ``Combined'' $\mu_c = \mu_1 + \mu_2 = 2\mu$
\item ``Combined'' $\sigma_c = \sqrt{ \sigma_2^1 + \sigma_2^2 + 2 cov(\sigma_1, \sigma_2)} = \sqrt{2 \sigma^2} = \sqrt 2 \sigma$
\end{itemize}
In this case $CV = \frac1{\sqrt{2}} \frac\sigma\mu$ - by combining demand we reduce variability (so-called {\bf demand pooling}).\\

Opposite process - {\bf fragmenting demand}.

\subsection{Flexibility}
\label{sec:05-06}

The idea behind {\bf SMED} (Single Minute Exchange of Diet(?))  is to shorten the setup time to increase flexibility. Every setup is divided in two types of activity:
\begin{itemize}



\item external
\item internal
\end{itemize}

\subsection{Platforms and Modularity}
\label{sec:05-07}
Design for supply chain performance: to keep separation to as late moment as possible (video 5-7, 03:00) - so-called {\bf delayed differentiation}.

\subsection{Limits to Customer Choice}
\label{sec:05-08}

See Apple's iShit: goes in various memory sizes (yeah, two), but same appearance. Overall, there are two reasons to limit the customer choice:
\begin{itemize}
\item customer is overwhelmed from a multiple choice
\item ???
\end{itemize}

\section{Responsiveness}
\label{sec:sec06}

\subsection{Intro}
\label{sec:sec06-01}

``Waiting is another form of mismatch between supply and demand''. Usually a waiting means insufficient capacity, but ``variability does not averages itself out'' too - see graph at video 6-1, 08:00

\subsection{The Waiting Time Formula}
\label{sec:06-02}

Waiting occur even if the average utilisation is below 100\%. The demand process is somewhat ``random''; one can introduce the ``inter-arrival time'' $IA_n$ with average \\
Distribution is: $CV_a = \frac{St-Dev(inter-arrivalTimes)}{Average(inter-arrivalTimes)}$\\
- often Poisson distributed
\begin{itemize}
\item $CV_a = 1$
\item Constant hazard time (no memory)
\item Exponential inter-arrivals
\end{itemize}

\paragraph{Processing}

Also there is some variability in the processing time:\\
$CV_p = \frac{Ct-Dev(processingTimes)}{Average(processingTimes)}$ \\
this can have many distributions (depends strongly on standardisation - or rather shows the level of standardisation); often Beta of LogNormal.
% video 6-2; 04:20

\paragraph{The Waiting Time Formula}

- see page 10 of Module 4 slides. Expected (average) time in queue
$$\mbox{Time in queue} = \mbox{Activity Time} \times \frac{utilization}{1-utilization} \times \frac{CV_a^2 + CV_p^2}2 $$
\begin{itemize}
\item $\mbox{Activity Time}$ - service time factor (processing time)
\item $\frac{utilization}{1-utilization}$ - utilization factor
\item $\frac{CV_a^2 + CV_p^2}2$ - variability factor
\end{itemize}

\subsection{Waiting Time Models with Mulitple Servers}
\label{sec:06-03}

for $m$ parallel servers (page 14 of Module 4 slides):
$$\mbox{Time in queue} = \frac{\mbox{Activity time}}m \times \frac{utilization ^{\sqrt{2(m+1)} - 1}}{1 -utilization} \times \frac{CV_a^2 + CV_p^2}2$$

Utilisation:
$$ u = \frac{1/a}{m * 1/p} = \frac{p}{am}$$

Inventory (flow rate $1/a$):
\begin{itemize}
\item $I_q = \frac1a * T_q$: inventory in queue =  rate * time in the queue
\item $I_p = u \times m$ - inventory ``in process'': utilisation * number of servers
\item $I = I_p + I_q$ - total inventory
\end{itemize}

When we have seasonal data: slice the data by hour (day, season etc); assume demand in ``stationary'' within a slice

\subsection{Pooling}
\label{sec:06-04}

Pooling by itself does not change the utilisation but reduces the (variability-based) waiting time. Limitations: see page 25 in Module 4 slides.

\subsection{Sequencing}
\label{sec:06-05}

It is about deciding ''who should be served first'' - or re-shuffling the queue through some priority rules
% video 6-5; 03:00
Usual idea - SPT (Shortest Processing Time), but sometimes it is problematic to predict a ``true'' processing time

\subsection{Process Design}
\label{sec:06-06}


\subsection{Loss Models}
\label{sec:06-07}

What happens if customer can not (unwilling) to wait?
\begin{itemize}
\item Waiting problem: Utilisation has to be less than 100\%. Impact of variability is on Flow Time
\item Loss problem: Demand can be bigger than capacity. Impact of variability is on Flow Rate
\end{itemize}

So:
\begin{itemize}
\item $m$ - amount of resources (``bays'')
\item $a$ - Arrival time; $CV_a$ is set
\item $p$ - Processing time; $CV_p$ is set
\end{itemize}

The probability that all $m$ resources are utilised: $P_m$ - big math problem; can be simplified by using so-called Erlang Loss Table using $r=p / a$ and $m$ (page 41 of Module 4 slides).\\
With this $P_m(r)$ one can find:
\begin{itemize}
\item time per day that system has to deny access ($24h * P_m(r)$)
\item flow units lost = $1 / a * P_m(r)$
\end{itemize}

\section{Quality}
\label{sec:07}

\subsection{Reasons for Defects}
\label{sec:07-01}

Two dimensions: {\bf performance} (to what extent the product or service is meeting customer's expectation) and {\bf conformance} (whether the process is carried out in the way we intend it to be carried). This module is concentrated on conformance quality, and the variability is our ``root of all evil'' again.\\

If probability of the defect is $p$, than yield of the process or stage is $y = 1 - p$ - share of ``good''(produced according to specification) units produced. Plus, defects increase variability in flow.

\subsection{Quality and Flow}
\label{sec:07-02}
Defects noticed before the bottleneck are driven by the input prices. Defects which are happening after bottleneck - driven by ``sell'' price (opportunity cost). => Tests are a good idea.\\

\paragraph{Idea of Kanban System: Demand Pull}
Kanban cards are also called ``demand cards''. Amount of WIP (W in production) is determined by number of cards. - Pull instead of buffers.

\subsection{Six Sigma}
\label{sec:07-03}

parameters:
\begin{itemize}
\item LSL - lower specification level
\item USL - upper specification level

\item Capability: $= (USL-LSL) / (6 * \sigma)$, where $\sigma$ is a standard deviation. This shows the likelihood of the defect: the more forgiving is our $USL - LSL$ or lower $\sigma$ is, the higher probability is that our product will be according to spec,
\end{itemize}

PPM (Parts per million) - expected amount of defected parts out of one million produced,

\subsection{Control Charts}
\label{sec:07-04}

Part of Statistical Process Control.\\

{\bf Common Cause Variation } - a variation among the items in the same set (produced by the same process)\\
{\bf Assignable Cause Variation} - a variation between the items produced by different process.\\

\paragraph{Detecting Abnormal Variation in the Process}

- identifying assignable causes {\em from the real (not spec) data}. Control limits (do not mix with {\bf specification} ones):
\begin{itemize}
\item lower: $LCL = Mean - 3 \sigma$
\item lower: $UCL = Mean + 3 \sigma$
\end{itemize}
Distinguish between:
\begin{itemize}
\item common cause variation (within control limits)
\item assignable cause variation (outside control limits)
\end{itemize}
Statistical Process control - see diagram at page 22 of Module 5 slides.

\subsection{Jidoka}
\label{sec:07-05}

Big inventory grows up the ITAT (information turnaround time) => slow feedback and no learning. The time when defect is found (detected) is more interesting than the time when it was actually made (see before bottleneck/after bottleneck example).

\paragraph{Detecting Abnormal Variation in the Process at Toyota}

- Detect-Stop-Alert framework (page 26 of Module 5 slides).

\subsection{Problem Solving}
\label{sec:07-06}

Ishikawa Diagram (also rule of 5 why's) - see page 31 (left half) for example.\\
Pareto chart: marks out the assignable causes of a problem in the categories of the Ishikawa diagram; order root causes in decreasing order of frequency of occurrence. Than use 80-20 logic

\subsection{Toyota Production System}
\label{sec:07-07}

\begin{itemize}
\item Systematic elimination of waste
\item Operating system built around serving demand
\end{itemize}




\end{document}