\documentclass{scrartcl}

\usepackage{amsmath,amssymb,amsfonts} % Typical maths resource packages
\usepackage{graphics}                 % Packages to allow inclusion of graphics
\usepackage{color}                    % For creating coloured text and background
\usepackage{hyperref}                 % For creating hyperlinks in
% cross references

\usepackage{algorithm}
\usepackage{algorithmic}

\usepackage[T2A]{fontenc}
\usepackage[utf8]{inputenc}
\usepackage[russian,english]{babel}
\usepackage{listings}
\lstloadlanguages {[LaTeX]TeX, Octave}
\lstset {language=[LaTeX]TeX, extendedchars=true ,escapechar=|}

\begin{document}
\tableofcontents
\section {Definitions}
\label{sec:2-1}
% video 2-1
\begin{itemize}
\item {\bf DNA } - hereditary material (composed of nucleotides A, C, G, T)
\item {\bf Genes } - units of biological info, made of DNA
\item {\bf Chromosomes } - organised package of DNA, bearing genes along its
  length (``scaffolding for genes'')
\item {\bf Genome } - entire set of DNA instructions in a cell
\item {\bf Diploid } - having one gene copy from each of two parents
\end{itemize}
% video 2-3
\subsection{Scales of Organisations}
\label{sec:2-3}
DNA is organised in chromosomes; sections with specific instructions are
``genes''. There are three primary levels of genetic information:
\begin{itemize}
\item {\bf DNA } - replicates self, transmits to offspring
\item {\bf RNA } - intermediate, some forms directly functional
\item {\bf Protein } - structure and content affects phenotype.
\end{itemize}
The formation of messenger RNA from DNA is ``transcription'': RNA sequqnce in
complement of DNA (A-U, C-G, G-C, T-A)\\
Message RNA (mRNA) is being translated to string of amino acids; code for
translation is 3 bases each (triplet code): specific triplets ({\bf Codons })
are being translated in specific acids, like AUG -> Met, GAU->Asp etc. The chain
is finished with ``stop codon'': UAA or other (we know three stop codons now).
See standard genetic code table for details: $4^3 = 64$ entries.\\
Some areas of the genome:
% video 2-3 08:00
\begin{itemize}
\item {\bf Genes } - often have amino-acid-coding segments
\item {\bf Introns } - areas within genes spliced out of mRNA (do not affect
  amino acid sequence)
\item {\bf Intergenic regions } - areas between genes, often functional:
  regulate how much mRNA produced, and sometimes produce other RNAs.
\end{itemize}
% video 2-4
\subsection{Mitosis, Meiosis and Ploidy}
\label{sec:2-4} {\bf Ploidy } - a number of complete copies of genetic information itself. Most animals have two copies (diploid) in the cell.
\begin{itemize}
\item {\bf Mitosis } - produces two diploid daughter cells genetically identical to single parent diploid cell.\\
  When mutation occurs during mitosis, you get a {\bf genetic mosaic }. Mutation in mitosis control genes often lead to cancer.
\item {\bf Meiosis} - produces {\bf haploid (1N)} daughter cell with one copy of genes from a parent diploid cell.\\
  Fertilisation brings together gametes from the two parents. If got two copies of a chromosome from one parent (and a third copy from the other) - bad things happen, like Down syndrome (three copies of chromosome 21), Klinefelter syndrome etc.
\end{itemize}
% video 2-5
\subsection{Basic Single-Gene Inheritance}
\label{sec:2-5} {\bf Allele } - a single (one of) a number of alternative forms
of the same gene or some genetic locus (a group of gene); located at a specific
position on a
specific chromosome.\\
``Masking'' - dominance of the dominant gene over the recessive one. Organisms
having both dominant and recessive genes are called {\bf heterozygous };

\paragraph{Mendel's First Law: Three postulates}
\begin{itemize}
\item Unit factor in pairs (diploid): one allele from mom, one allele from dad.
\item Dominance/ recessivity (sometimes $F_1$ is intermediate)
\item Equal segregation in gametes: paired factors separate, and equally likely
  to transmit either one to offspring.
\end{itemize}
% video 2-6
\subsection{X-linked Inheritance and Independent Assortment}
\label{sec:2-6}
Inheritance sometimes is different based on sex of individual. Inferred from
this, the principles of sex linkage (X-linkage):
\begin{itemize}
\item Some genes are not present on two copies in all organisms:
  \begin{itemize}
  \item often males XY, females XX
  \item Not the same set of genes on the X and Y despite the pairing of these
    chromosomes (often few functional genes on Y)
  \end{itemize}
\item Different patterns of inheritance depending on who is mom/dad and whether
  kid is male or female.
\end{itemize}
X-linked inheritance can be studied in the same way - just insert ``Y'' instead of the second allele for dad. Girls will not get this (they get both XX), boys - do get. Working example - green colour blindness.

Consider two traits controlled by genes on different chromosomes: Ss and Tt. T and S alleles are inherited independently in $F_2$: TS, Ts, tS, ts. Two ways to approach:
\begin{itemize}
\item Multiply probabilities: 1/4 TT, 1/2 Tt. 1/4 tt; 1/4 SS, 1/2 Ss, 1/4 ss
\item Follow all gametes: TS can fertilise TS, Ts, tS, ts... - work out all 16
  possibilities
\end{itemize}
After filling up tables: 9 for TS, 3 for tS; 3 for Ts, and 1 for ss.


Rule:\\
For two {\bf independent } events, multiply the two probabilities for the joint probability: assuming 1/4 TT, 1/2 Tt. 1/4 tt; 1/4 SS, 1/2 Ss, 1/4 ss can decide that TTSS will be at 1/16, TTSs - in 1/8, .... ttss - again 1/16

% video 3-1
\section{Recombination}
\label{sec:3-1}
A gamete with a combination of alleles that did not come from the parents is a
{\bf recombinant } gamete. Means:
\begin{itemize}
\item Independent assortment
\item Crossing over
\end{itemize}
Often homologous chromosomes ``trade'' pieces with each other during meiosis - {\bf crossing over }. This happens when c. doubled and can form mixed (``recombinant'') chromosomes in gametes.\\
This is ``discovered by'' looking at alleles located near {\bf genetic markers}.

Neighbouring alleles (gene variants) tend to stay associated; the whole story allows bigger diversity in offspring genes (pages 16, 18 on week3-1 slides).

% video 3-2
\subsection{Calculating Recombination Distances Between 2 Genes}
\label{sec:3-2}
Chromosomes are linear, and neighbouring variants tend to stay associated
(``parental'' combination more common than ``recombinant'' combination in
gametes, ranging from full association [0\% recombination] to no association
[50\% combination]).

To distinguish separate alleles, so-called {\bf Genetic markers} are used. These
are the reference points in the genome with 2+ alleles (don't necessarily know
ahead of time where it is). Markers can be molecular of phenotype.

% video 3-2 04:35
\subsubsection{Use Crosses (Pedigree) and Inferred Genotypes to Make Map}
\label{sec:3-2-1}
\begin{itemize}
\item Identify ``recombinants'' and ``parentals''
\item If A and B are linked, what are fractions of all?
\item If A and B are un-linked, what are fractions of all?
\end{itemize}
Video 3-2, 05:00 -12:30 is a long example on crossing.

\paragraph{The fraction recombinant reflects the distance between the genes}

\begin{itemize}
\item \% recobinant is called ``map units'' (mu)
  \begin{itemize}
  \item in Drosophila, often called centiMorgans (cM)
  \item recombinant fraction ranges for 0\% to 50\% (=0-50 cM)
  \end{itemize}
\item Gives and idea of distance between genes
  \begin{itemize}
  \item Developed before we had ``genome sequences'' and known physical
    distances in base-pairs.
  \item In humans, average 1.3cM ~ 1 million bases.
  \end{itemize}
\item Can determine linear order of genes.
\end{itemize}

% video 3-3
\subsection{Generating a map using 3 or more genes}
\label{sec:3-3}
What if we have three genes?
\begin{itemize}
\item {\bf ABC/abc x abc/abc }
\item Parental phase is ABC or abc, but what is the linear order (is A next to B
  or A next to C)? (ABC, BAC or ACB?)
  \begin{itemize}
  \item ABC/abc - parental 479 Parental (AB)
  \item abc/abc - parental 473 Parental (ab)
  \item Abc/abc - recomb 15 Recombinant (Ab)
  \item aBC/abc - recomb 13 Recombinant (aB)
  \item ABc/abc - recomb 9 Parental-like (AB)
  \item abC/abc - recomb 9 Parental-like (ab)
  \item AbC/abc - recomb 1 Recombinant (Ab)
  \item aBc/abc - recomb 1 Recombinant (aB)
  \end{itemize}
\end{itemize}
Recombinations on A and B: 15 + 13 + 1 + 1 = 30 out of 1000 total, so
recombination distance between A and B is 3\% (3 cM).\\
For A and C: 15 + 13 + 9 + 9 = 46; 4.6 cM\\
For B and C: 9 + 9 + 1 + 1 = 20; 2 \%

So, final decision from these distances: A - B - C (A-C is the longest, and A-B
plus B-C is roughly equal to AC)\\
AB + BC do not add to AC because of ``double crossovers'' which counted as
``parental'' for A-C.

\paragraph{Tricks}

If all loci are linked:
\begin{itemize}
\item Largest two will be original, non-recombinant parentals
\item Smallest two will be ``double-crossover'' (and they are rarest class)

  Can identify which marker is in the middle in rarest class because alleles at
  two markers will be parental, while allele at third will be recombinant.
\end{itemize}
% video 3-3 09:45
Do remember that calculations near 50\% are quite inaccurate! (really everything
above 40\% should be treated rather as reference, not precise values).

% video 3-4
\subsection{Genetic Mapping}
\label{sec:3-4} {\bf Genetic mapping } is using associations between alleles at
multiple spots ({\bf loci}) in the genome to:
\begin{itemize}
\item determine if a gene is a relative to another genes
\item localise a gene causing a phenotype (disease genes)
\end{itemize}
Look at Human Genome project (~ 20,000 human genes identified in 3 bn bases fo
DNA).

Genetic mapping is a first approach, very much like Drosophila exmples: identify
location of ``disease mutation'' relative to other spots on the genome. To do
this, we need {\bf reference points} in the genome (like ``genetic
markers''): \\
Several nucleotide sites found to be variable; can genotype blood or tissue for
these {\bf SNP} (single nucleotide polymorphism) sites and use as
genetic markers - ``snips''.\\
Resources available:
\begin{itemize}
\item we now have the genome sequence mostly put together from end to end on
  each chromosome
\item we have computer predictions of what parts are likely ``genes'' encoding
  for protein
\item we also have data showing some bases that differ between a log of humans
  (``HapMap'' project)
\end{itemize}
Ultimately the essence of all genetic mapping comes down to seeing an {\bf
  association between genotype and phenotype}:
\begin{itemize}
\item Genotype at markers (AA vs Aa vs aa)
\item Phenotype may be disease or other trait: healthy vs. diseased.
\end{itemize}
Video 3-4, 08:50 - example

% video 3-5
\subsection{Mapping a simple genetic trait relative to genetic markers in a
  cross}
\label{sec:3-5}
We can use HapMap markers like Drosophila mutations to map simple diseases. \\
Score offsprings of ABC/abc $\times$ abc/abc:
\begin{itemize}
\item ``A'' and ``B'' can be SNPs with known locations
\item ``C'' is the disease gene (cc = diseased)
\end{itemize}
plus, determine whether C is between A and B, and approximately how far away -
this tells relative location of disease gene. Video 3-5, starting from 03:00 -
whole example of locating

% video 3-6
\subsection{Mapping a simple genetic trait relative to genetic markers in a
  population}
\label{sec:3-6}
Using population data collected over time: historically human generation takes
time about 20 years. \\
When looking at very close genes, we saw low probability of exchange in ONE
generation, but, over the thousands of years of history, there has been a lot of
recombination: most neighbouring genes shuffled, and even areas within genes are
sometimes shuffled.

Recombination is not homogeneous when look at a very fine scale. They tend to
occur in ``hotspots'', rest of genome have about 0 RF (recombination fractions).

Shuffling occurs between the ``windows'' every few thousand BP (base pairs, or
simply ``bases''); virtually no shuffling occurs within windows. This is said to
be in LD (``linkage disequilibrium''). These features can be used to find
disease genes: Snips (single nucleotide polymorphism sites - SNPs) are
associates inside such windows but dissociated between them (slide 3-6 page 7;
video 3-6 10:15).

\paragraph{Prediction:}

Disease gene mapping is associating a ``genotype'' (marker allele at one
location) with a ``phenotype'' (disease). If a marker is very close to the
disease -causing gene, individuals having one allele will be more likely to have
the disease than individuals having the other allele - the marker is in LD with
the disease gene.

This does NOT mean the marker gene or SNP causes the disease!

Simple calculations: human genome is about 3 bn bases with hotspots every
~50,000 bp. So we'd have about 60,000 windows - if things were so simple. The
problem is, these markers are not fully present in whole population - so we need
to use about 1 million markers for studying.

For reference: in single generation, the recombination rate is quite low: two
sides of hotspot probably are < 0.01 cM apart.

% video 4-1
\section{Mutations as the Origin of Genetic Variation}
\label{sec:04-1}
``Changes in code'' are  ``a change to the DNA sequence''.  {\bf Mutations are the ultimate source of all genetic variation on the planet}

Any source of change in the genome:
\begin{itemize}
\item Error in replication/meiosis leading to a change in a base
\item ???
\end{itemize}

Mutation is {\bf not:}
\begin{itemize}
\item mutation don't happen ``when we need them'': they are random
\item Many mutations are ``bad''
\item Some mutations ``don't matter'' AAA -> AAG - actually, same protein will be produced
\item Rarely (though it happens) a mutation is ``good''
\end{itemize}

{\bf SNP} single-nuclear polymorphic (see definition).\\
After mutation, the new variant {\em may} spread in the population.
A lot of mutations don't have an effect on fertility or life span, but: may still affect a phenotype (for example, we have a lots of genetic variation in human height).

%video 4-2
\subsubsection{Shortcomings of the Single Gene Model}
\label{sec:4-2}
Why do we see so many variations? Many answers:
\begin{itemize}
\item The simple single gene/ 2 allele model is insufficient: almost every phenotype is controlled by more than one gene. For (fictional) example, alleles at 6 genes control height.  We get a good old bell curve on distribution: despite Mendelian inheritance of each gene involved, as a result we get a continuous variation from many genes.

  Variation becomes apparent in F2 crosses (i.e. crosses between offspring of 2nd generation): pages 6 - 8 of slides pot.

\item Variable ``penetrance'': mutant forms may ``sometimes'' affect phenotype but not always do so: some individuals with the mutation have a normal or nearly-normal phenotype - say the mutation is ``not fully penetrant'' (usual story for cancer susceptibility).
  {\bf Example: } BRCA1 breast cancer susceptibility:
  \begin{itemize}
  \item In most population (BB or Bb), low (12\%) risk of breast cancer
  \item if have 2 mutant alleles (bb), 60\% risk of breast cancer
  \item therefore, BRCA1 breast cancer phenotype is {\bf not fully penetrant} - you CAN be ``bb'' and still have all non-cancerous cells is breast
  \end{itemize}
\item Interactions among genes: ``epistals''\\
  Effect of genotype at one gene modifies effect of genotype at another gene
  % video 4-2 10:00
  Example: ``white-pink'' Pea Flower Colour: based on two genes (C and P), givespink colour if any of these is homozygous on any gene (so-called pathway with two genes).
  Provides 9:7 phenotypic ratio (page 14 of slides 4-2) in contrast to Mendelian 1:4\\
  Another example: mouse coat colour (p 15 slide 4-2).
  Epistasis(?) happens:
  \begin{itemize}
  \item One possibility: 2 genes
    \begin{itemize}
    \item Gene 1 (B) is a ``switch'' that tunes on gene 2: On (B-) vs Off (bb)
    \item Gene 2 (A) affects deposition of coat colour. 2 variants: one causes deposit of black, other brown
    \end{itemize}
  \item AA vs aa cause deposition of black vs brown
    \begin{itemize}
    \item but, if bb, then no deposition at all:
    \item Genotype at B modifies effect of genotype at A (and sometimes eliminates its effect)
    \end{itemize}
  \end{itemize}
\item There can be >2 alleles at a locus!\\
  Classic example: ABO blood types:
  \begin{itemize}
  \item Gene located on chromosome 9 of human
  \item three alleles: A, B are dominant over O. A and B create specific antigens, O does not.
  \item Blood phenotypes: A, B, AB, or O, genotypes AA, AO, BB, BO, AB, OO
  \item If get transfusion, reject if receive blood with foreign antigen: O blood is best to donate; AB best to receive blood transfusion
  \end{itemize}
\item Environment, and interactions with it: environment can affect phenotypes (example: some people naturally tan easier than others)
\end{itemize}
% video 4-3
\subsection{Mutation Rates}
\label{sec:4-3}
All genetic variations started as mew mutations. Mutation rate is measured per generation:
\begin{itemize}
\item Depends on type of mutation:
  \begin{itemize}
  \item Specific base to specific base: A -> G
  \item Specific base to ``any different'' base: (A -> nonA (GTC)) - measured by
    \begin{itemize}
    \item Get sequences of whole genome after known number of generations (and taking average)
    \item Comparing sequences from existing species and estimating number of generations
    \end{itemize}
    - so we got $2.1 \times 10^{-8}$ mutations per base per generation in worms. Human genome size is $3.1 \times 10^9$ bases.
    Applying: 65 new mutations! (confirmed by studies: 63 new mutations on average; strong effect of father's age: the elder is father, the bigger in a number of mutations)
  \item Other types (inversions: part of chromosome reverts order, trans-locations, deletions)
  \item anything in gene that yields a phenotype (ignores a ``neutral'' changes)
  \end{itemize}
\end{itemize}
Once mutation arise:
\begin{itemize}
\item New mutations may spread or be lost (via natural selection or other forces)
\item While present, new variations will have a ``frequency'' in the population:
  \begin{itemize}
  \item Allele frequencies of blood types: 24\% A; 10\% B; 66\% O in USA
  \item Genotype frequencies of blood types: 6\% AA; 32\% AO; 5\% AB; 44\% OO etc.
  \end{itemize}
\end{itemize}
``Deleterious'' mutations:
\begin{itemize}
\item If chromosomal anomalies, may have immediate effect on offspring, often result in miscarriage (abortion)
\item A log of other bad mutations are either more subtle or are masked because recessive.
  Current estimates, $\approx 1-2$ per generation
\end{itemize}
% video 4-4
\subsection{Mapping Complex Traits}
\label{sec:4-4}
Two general approaches:
\begin{itemize}
\item Mapping differences in crosses / pedigrees
\item Mapping variations within population (e.g.GWAS - Genome Wide Association Study)
\end{itemize}
We often know the locations of the {\bf markers} being used ahead of time, but we {\bf don't} know the locations of the genes causing traits/diseases ahead of time.\\
We map complex traits to {\bf QTL} - Quantitative Trait Locus (a locus with allelic variation that influences a phenotype).
Its exact location is not known, but we infer its approximate location from association with marker genotypes.\\
Often detect may QTLs affecting single traits.
If there are many genes, and/or if the effect is ``complicated'', the association between markers near (``linked to'') gene(s) causing different phenotypes may not be very strong.

\paragraph{Recombination between marker and gene affecting phenotype.}
\begin{itemize}
\item If you have NO recombination between marker and trait gene, then marker genotype predicts trait phenotype very well (maybe perfectly)
\item if you have A LITTLE recombination, marker genotype may still be ``associated'' with trait
\item if LOTS of recombination, no association
\end{itemize}
Fine localisation requires examining associations of multiple linked markers with trait - ideally we need to identify ``stronger'' associations in some markers than others. \\
Associations will be stronger when there's less recombination between a marker and a causative gene (``QTL''):
\begin{itemize}
\item if look at many linked markers, you should be able to pinpoint the location of a QTL by where the association is strongest (OR where you'd predict it to be strongest
\item Can follow ``trajectory of association strength'' to infer location of the QTL
\end{itemize}
%video 4-5
\subsection{Mapping complex trait in crosses}
\label{sec:4-5}
Association is depicted using ``LOD'' plot: use statistic to translate observed and expected associations and identify ``likely'' position of gene(s) affecting trait of interest.
High number (>3) means likely effect.\\
Big difference from simple traits is that now we have markers in multiple regions of genome associated with trait.

\paragraph{We do not know what makes the QTL effect}
\begin{itemize}
\item May be multiple genes that are close together or a single gene
\item {\bf Important:} QTL mapping (by cross/pedigree or by population association study) is not conclusive - it is an hypothesis of where one or more genes affecting the trait.
\end{itemize}

% video 4-6
\subsection{Genome-wide association studies (GWAS)}
\label{sec:4-6}
Let's say you map a gene causing a diabetes in one family to 18p22 (strong effect next to a marker).
You map in another family having diabetes and you see no effect of 18p22 region.\\
So, another approach is to map in population by association studies (e.g.GWAS): similar story like the one for simple traits.
(If a marker is very close to the disease-causing gene, individuals having one allele will be more likely to have the disease than individuals having the other allele).

\paragraph{Testing many (thousands) markers}
\begin{itemize}
\item Some have an allele associated with disease:
  \begin{itemize}
  \item AA: prob disease: 10\%
  \item aa: prob disease: 1\%
  \item Chi-square or other statistic says unlikely difference by chance
  \end{itemize}
\item Many have no association
  \begin{itemize}
  \item BB: prob disease 1.5 \%
  \item bb: prob disease 1.6 \%
  \end{itemize}
\end{itemize}
Problem: multiple comparisons. \\
From lab, chi-square test gave probability of getting that association ``by chance'': when test 1000000 SNPs for association with disease of interest, it is very likely to get an association that looks strong by change but no real biology.
As a result, instead of using 5\% probability as a cut-off, often use 0.00001\% probability.
But this requires a VERY strong association - so we can reject a ``good'' association.

\paragraph{Pedigree/cross vs Population}
\begin{itemize}
\item Pedigree/cross analyses find genetic factors causing differences {\bf in that family} - typically, high power for small number of genes\\
  - because you're limited to mapping genes with effects in that cross, BUT the alleles you're mapping are abundant (25\% for $Zz \times zz$)
\item Population association studies try to find genetic factors causing variation {\bf across the population} - typically, lower power but can find more genes and localises more precisely.\\
  - you can theoretically map alleles at any gene causing effect, BUT you may have different very rare variants causing the same effect/disease
\end{itemize}
GWAS in populations work for mapping {\bf ``common disease variants''}, but:
\begin{itemize}
\item are most disease variants common?
\item how often do different rare mutations cause the same disease?
\end{itemize}
Also, can see difference among ethnic groups for associations: sometimes see a marker allele associated with disease in one ethnic group but not another.

% video 5-1
\section{Heritability}
\label{sec:05-0}
\subsection{Contribution of Genes vs Environment}
\label{sec:05-1}
The ``Genes vs Environment'' is not a pure dichotomy, while one may contribute more then other:
\begin{itemize}
\item eye colour (excluding contacts)
\item HIV status
\end{itemize}
The idea is to separate one from another:
\subsubsection{Resemblance between relatives}
\label{sec:5-1-0533}
\begin{itemize}
\item Constant environment, varying genetic relations\\
  Example: two types of twins:
  \begin{itemize}
  \item Monozygotic (``identical'') - genetically exactly the same (always same sex, for example); shared placenta
  \item Dizygotic(``fraternal'') - genetically like any brother/sister; separate placentas.
  \end{itemize}
  If there's genetic component to some trait, {\bf monozygotic} twins should be more similar than {\bf dizygotic} twins.\\
  Some correlations found:
  \begin{itemize}
  \item For IQ: monozygotic twins 0.85; dizygotic twins 0.42
  \item For Gastroesophageal re-flux disease: monozygotic twins 0.29; dizygotic twins: 0.13
  \end{itemize}
  This does not show the absence of Environment component (after all, twins usually share very similar environment), but does show the presence of genetic component
\item Constant genetic relations, varying  environment\\
  Monozygotic twins reared together should have higher correlation in traits than monozygotic twins reared apart if there's environmental component
  % video 5-1, 11:30
  \begin{itemize}
  \item For BMI (Body Mass Index): Monozygotic together 0.74, apart 0.70\\
    - surprisingly small environmental impact
  \item For Verbal ability: Monozygotic together 0.76, apart 0.51\\
    - very strong environment impact\\
    Dizygotic together: 0.43 (genetic component is still strong)
  \end{itemize}
\end{itemize}
\subsubsection{``Common Garden'' Experiments}
\label{sec:5-1-}
- studies of {\em Potentilla} in 1930 by Clausen, Keck \& Hiesey: study plants of the same specie from different environments (low-, medium- and high- altitude: plants looked ``a bit'' differently).
Grow plants obtained from different places in the same environment.
Prediction:
\begin{itemize}
\item If form difference is all {\bf environmental}, plants would all look the same
\item if part of the form difference is {\bf genetic}, plants would look different
\end{itemize}
- same concept as Mono- vs Dizygotic twin studies (constant environment, unrelated plants)\\
Results:\\
form different apparent even when grown in the same environment $\Rightarrow$ form difference {\bf has} a genetic component.\\
It does NOT prove that there is no environment component!
\subsubsection{"Reciprocal Transplant" Experiment}
\label{sec:5-1-1}
Grow plant forms obtained from a single environment (assuming the genetics is fairly constant) in several different environments\\
Prediction:
\begin{itemize}
\item if part of the form difference is {\bf environmental}, plans would all look different
\item if form difference is all genetic, plants would still look the same
\end{itemize}
- same concept as twins reared together vs apart: same genetic makeup, varying environment.\\
Results:\\
Form difference for same type grown in different environments $\Rightarrow$ form difference has an environmental component.

% video 5-2
\subsection{How Much do Each Contribute}
\label{sec:5-2}
The concept of {\bf Heritability} - based on stats.

\paragraph{What causes the variance in traits like height?}
(you see the variance in the {\bf phenotype}: some of the variance is genetic, some is environmental.\\
Simple idea: $V_P = V_G + V_E$:
\begin{itemize}
\item $V_P$ - phenotypic variance
\item $V_G$ - genetic variance
\item $V_E$ - environmental variance
\end{itemize}
Example: genetic variation becomes apparent in $F_2$ of crosses: 6 genes for ``height'':
\begin{itemize}
\item start with AA BB CC DD EE FF (6' tall people) having kids with aa bb cc dd ee ff (5' tall people)
\item offspring (F1) all heterozygous (5' 6'' tall): Aa Bb Cc Dd Ee Ff - we can find environmental variance ($V_E$), as no genetic variance is here (all ppl have the same genotype on these genes, despite heterozygous)
\item in F2: Aa Bb Cc Dd Ee Ff $\times$ Aa Bb Cc Dd Ee Ff: if un-linked, MANY possibilities.
  We can find $V_P = V_G + V_E$
\end{itemize}

{\bf Heritability:} fraction of total phenotypic variance that's genetic: ($V_G/V_P$), or ($\frac{V_G}{(V_G + V_E)}$)\\
- ranges from 0 (no genetic) to 1 (all genetic)\\
% video 5-3
This will not always work for people, so better (easier) to use

\paragraph{Parent-Offspring Correlation}
If we assume all variation is genetic (and assume no dominance), then offspring should be exactly the average of his two parents.
Slope of the dependency line ``height of offspring depending on average height of parents'' shows the strength of correlation - {\bf estimates heritability}.\\
% video 5-2 04:00
But story is not that simple:
\begin{itemize}
\item Environment is not ``constant'': estimates will be different in different places because $V_E$ different
\item amounts of genetic variation also not constant in different families or populations
\end{itemize}
``It is not a reference, it is a starting point''

% video 5-4
\subsection{Breeders Equation}
\label{sec:5-4}
Heritability can be calculated by response to artificial selection:\\
Start with 5 feet tall (corn)\\
Select 7 feet tall plants, breed:
\begin{itemize}
\item if offspring average 5 feet, heritability = 0.0
\item if offspring average 5.5 feet, heritability = 0.25
\item if offspring average 6 feet, heritability = 0.5
\item if offspring average 6.5 feet, heritability = 0.75
\item if offspring average 7 feet, heritability = 1.0
\end{itemize}
Another way to think about it: Heritability = Response ($C_G$)/Selection($V_G + V_E$) - a fraction of phenotypic variation that is genetic.

\paragraph{Artificial vs Natural Selection}
Formulations:
\begin{itemize}
\item {\bf Artificial selection:} breeder chooses desirable traits and has organisms with the most extreme desirable traits breed
\item {\bf Natural selection:} particular traits facilitate survival/reproduction, and organisms with the most extreme such traits have more offspring.
\end{itemize}

% video 5-5

\subsection{Population Growth}
\label{sec:5-5}
Natural selection is noncontroversial (?) and inevitable. Conditions:
\begin{itemize}
\item Phenotypic variance ($V_P$ is not zero)
\item Inheritance of the variation (heritability is not zero)
\item Variation affects survival or reproduction
\end{itemize}
Population can be modelled with a ``stable'' rate of increase:
\begin{itemize}
\item birth rate (\# births per thousand per year)
\item death rate (\# deaths per thousand per year)
\item birth rate - death rate = rate of increase ({\bf r})
\item USA example: r = 14/1000 - 6/1000 = 0.008 (population grows naturally by 0.8\% per year)
\end{itemize}
Standard rate of population growth: $\frac{dN}{dt} = r \times N$ \\
So, at time $t$: $N_t = N_0 e^{rt}$, and population doubling time is
$$\ln(2) = \ln(e^{rt}) \Rightarrow 0.693 = rt \Rightarrow t = \frac{0.693}r$$
For USA: 86.6 years.
% video 5-6
\subsection{Carrying Capacity}
\label{sec:5-6}

{\bf Carrying Capacity (K)} - the total number of individuals that can be supported within a population

\section{Hardy-Weinberg Equilibrium}
\label{sec:06-1}
\subsection{Allele and Genotype Frequencies}
\label{sec:06-1-1}
Calculating allele frequencies: either directly (calculate every allele), or generalised:
\begin{itemize}
\item freq(A) = freq(AA) + 1/2 freq(Aa)
\item freq(a) = freq(aa) + 1/2 freq(Aa)
\end{itemize}
- everything adds up to 1.\\
One can use ``joint probability'' multiplication to determine genotype frequencies in offspring. Example:
\begin{itemize}
\item Initial:
  \begin{itemize}
  \item 60\% sperm ``A''; 60\% of eggs ``A''
  \item 40\% sperm ``a''; 40\% of eggs ``a''
  \end{itemize}
\item Probability of ``AA'' offspring: ``A'' sperm fertilises ``A'' egg: $0.6 \times 0.6 = 0.36 = 36\%$
\item But. there is two ways to make ``Aa'' zygote:
  \begin{itemize}
  \item ``A'' sperm + ``a'' egg: $0.6 \times 0.4 = 0.24$
  \item ``a'' sperm + ``A'' egg: $0.6 \times 0.4 = 0.24$
  \end{itemize}
  - together 0.48
\item for ``aa'' - nothing interesting: $0.4 \times 0.4 = 0.16$
\end{itemize}
- together, add up to 1.\\
Now, calculate the allele frequencies:\\
\begin{itemize}
\item AA; 0.36 -> all A
\item Aa: 0.48 -> half A
\item freq(A) = $0.36 + \frac12(0.48) = 0.6$
\item freq(a) = 0.4 $(=0.24 + \frac12(.048)$
\end{itemize}
- so the whole story self-perpetuates.

% video 6-2
\subsection{The Hardy Weinberg Equilibrium (HW)}
\label{sec:06-2}
Up to 1902, people thought the {\bf dominant} alleles would intrinsically increase in a population some assumed rare alleles would always be lost eventually.\\
1908: Hardy \& Weinberg independently showed both assumption not true: allele and genotype frequencies stay stable {\em when some assumptions are made}\\
Notation:
\begin{itemize}
\item Frequency of A = p
\item Frequency of a = q; \\
  p + q = 1, and:
\item Frequency of AA = $p^2$
\item Frequency of Aa = $2pq$
\item Frequency of aa = $q2$\\
  again: $p^2 + 2pq + q^2 = 1$
\end{itemize}
(see graph on p.6 of slides 06-02), if:
\begin{itemize}
\item you can ALWAYS know genotype frequencies from genotype counts
\item you can ALWAYS know allele frequencies from genotype frequencies
\item but you CANNOT always know genotype frequencies from allele frequencies
\end{itemize}
So HW allows prediction of genotype frequencies from allele frequencies {\bf under certain conditions: }
\begin{itemize}
\item random mating (multiplying probabilities rule)
\item No selection / migration /mutation at that locus
\item infinite population size - no ``genetic drift''
\end{itemize}
- provides a good ``null hypothesis''; by seeing {\bf how } natural population deviate from the HW expected genotype frequencies we infer what evolutionary forces are operating.\\
Example;
\begin{itemize}
\item AA 245
\item Aa 210
\item aa 45\\
  Now calculate:
\item ``true'' genotype frequencies:
  \begin{itemize}
  \item total: 245 + 210 + 45 = 500
  \item AA: 245/500 = 0.49
  \item Aa: 210/500 = 0.42
  \item aa: 45/500 = 0.09
  \end{itemize}
\item ``true'' allele frequencies:
  \begin{itemize}
  \item p(A) = $0.49 + \frac12 0.42 = 0.7$
  \item p(a) = $0.09 + \frac12 0.42 = 0.3$
  \end{itemize}
\item HW ``expected'' genotype freq
  \begin{itemize}
  \item $p^2 = 0.7^2 = 0.49$ - corresponds to ``true''
  \item $2pq = 2(0.7)(0.3) = 0.42$ - corresponds to ``true''
  \item $q^2 = 0.3^2 = 0.09$ - corresponds to ``true''
  \end{itemize} - so population is ``in'' HW
\end{itemize}
% video 6-3
\subsection{Deviation from Hardy Weinberg Equilibrium: Wahlund Effect}
\label{sec:6-3}
Example: HW can be violated as a result of ``non-random mating'' - so-called {\bf Wahlund effect} (i.e. under-representation of heterozygotes relative to HW)\\
So, the first step in genome-wide association studies (GWAS) of genetic diseases is usually to {\bf test for HW}.
(actually, GWAS assumes first that HW holds - otherwise we'll get errors - false links between SNPs and traits) - which happens quite often.

% video 6-4
\subsection{Differences Between Populations: Origins and Quantifying}
\label{sec:06-4}
Quantifying:
\begin{itemize}
\item Simplest - {\bf all individuals differ: }
  \begin{itemize}
  \item ``fixed difference'': Population 1: all AA; Population 2: all aa\\
    - happens but not very common within a species, and generally not true among modern human ethnic groups
  \end{itemize}
\item More common - {\bf frequency differences of alleles \& genotypes: }
  \begin{itemize}
  \item Population 1: p(A) = 0.7
  \item Population 2: p(A) = 0.5
  \end{itemize}
\end{itemize}
-so measure differences between {\bf populations }, NOT {\bf individuals } (can not be applied to individual)

\paragraph{Deviation from HW allows you to quantify allele freq differences!}

Assume two populations at HW:
\begin{itemize}
\item if sample {\em each by itself} - see HW
\item if sample {\em both together} - see deviation from HW (like Wahlund effect - lack of heterozygous ppl).
\end{itemize}

{\bf How big the deviation is from HW } when sampling both together quantifies difference in allele frequencies:\\
Measure: $F_{ST} = \frac{\text{HW predicted 2pq} - \text{\% observed hetx}}{\text{HW predicted 2pq}}$ (ranges from 0 to 1):
\begin{itemize}
\item 0: no allele frequency differences
\item $0 < F_{ST} < 1$: allele frequencies differ somewhat
\item 1: ``fixed'' difference between populations
\end{itemize}
In words: $F_{ST}$ is the \% heterozygous of randomly chosen alleles within populations (observed) relative to that expected in the entire species (2pq)\\
\begin{itemize}
\item measures {\bf difference in allele frequencies }
  \begin{itemize}
  \item if identical allele frequencies, $F_{ST} = 0$
  \item if fixed for different alleles, $F_{ST} = 1$
  \end{itemize}
\end{itemize}
In {\bf human } population, $F_{ST}$ is quite small, because some $F_{ST}$ assumptions violated in humans:
\begin{itemize}
\item supposed to be applied to genes experiencing little/no natural selection
\item susceptible to differences (and historical changes) in populatio size among groups
\end{itemize}
But probably the biggest effect is caused by Gene flow

% video 6-5
\subsection{Differences Between Populations: Effects of Gene Flow}
\label{sec:06-5}
Gene flow (migration) makes populations' allele frequencies converge; prevents (and ``undoes'') divergence.\\
Happens by:
\begin{itemize}
\item Organisms (or gametes) move to new location and reproduce there
\item Math for it assumes it's ``random'' with respect to genotype: particular genotypes are not more/less likely to migrate
\end{itemize}
There are a number or models of gene flow:
\begin{itemize}
\item Content - island model
  \begin{itemize}
  \item huge effect of continent on island, but negligible effect of island on continent
  \end{itemize}
  Example: Continent allele frequency p = 0.5; island allele frequency = 0.9; migration rate = 1\% (on each generation, 1 \% of island inhabitants migrates from the continent to island).\\
  In 500 generations, island frequency value converges to continent value.
  (Influence island -> continent is negligible)
\item Island model
  \begin{itemize}
  \item Multiple populations affecting each others' allele frequencies
  \item Stepping-stone model
  \end{itemize}
  Example: 4 islands exchanging genes (same 1\% migration rate) with each other.
  Initial frequencies are like p=0.9, p=0.65, p=0.35, p=0.1
  In 500 generations, the frequency values converge on mean value.
\end{itemize}

\paragraph{Relevant Variables}
- What affects the speed of convergence (or how fast the allele frequencies become similar):
\begin{itemize}
\item Migration rate (how many migrants move)
\item How different the allele frequencies are
\end{itemize}

% video 6-6
\subsection{Inbreeding}
\label{sec:06-6}
Inbreeding: breeding between closely related individuals.
This is often caused by limited capacity for dispersal, and {\em changes distribution of genotypes}.\\

\paragraph{An extreme form of inbreeding: self - fertilisation}
- page 7 of slides 06-11.\\
Every generation, heterozygote fraction goes down, and feeds alleles into homozygotes.\\
- creates ``pure breeding'' lines.\\
So quantifying is quite simple:\\
- inbreeding (even if not self-fertilisation) reduces \% heterozygotes.
So reduction in \% heterozygotes from HW expected quantifies inbreeding.\\
So {\bf Wright's inbreeding coefficient } $F = \frac{\text{HW predicted 2pq} - \text{\% observed hetz}}{\text{HW predicted 2pq}}$:
\begin{itemize}
\item 0: at HW expectations for \% heterozygotes
\item 0 < F < 1: somewhat fewer heterozygotes that predicted
\item 1: no heterozygotes
\end{itemize}

\paragraph{Inbreeding F vs $F_{ST}$}
\begin{itemize}
\item {\bf Inbreeding F } looks at the individuals within one population
\item {\bf $F_{ST}$ } quantifies difference between populations
\end{itemize}
By itself, inbreeding only changes the distribution of alleles among genotypes, does not make any alleles ``go away''\\
So-called ``inbreeding depression'' requires {\bf natural selection } as well as inbreeding (if mate two relatives, they are likely to have same recessive mutation + moe likely to produce homozygous offspring.

\section{Natural Selection}
\label{sec:07}
\subsection{Fundamentals}
\label{sec:07-01}
Requirements for evolution by natural selection:
\begin{itemize}
\item Variation in traits
\item Heritability of traits
\item Trait variants affect survival / reproduction
\end{itemize}
Two approaches to studying:
\begin{itemize}
\item {\bf Quantitative Traits} - in context of heritability: {\bf Heritability = Response / Selection}
  \begin{itemize}
  \item Genetic component of variation dictates selection's response
  \item Response often from change in allele frequencies at {\bf multiple loci}
  \end{itemize}
\item {\bf Single Locus / gene}
\end{itemize}

%video 7-1, 05:13
\subsection{Single Gene Selection}
\label{sec:07-01-01}
The single gene selection (alleles at individual loci):
\begin{itemize}
\item affects abundance of particular genotypes
\item affects frequencies of alleles in population
\item {\bf Dominance of alleles matters for population}
\end{itemize}

\subsubsection{Strong selection in humans: Single loci leads to:}
\label{sec:07-01-02}
\begin{itemize}
\item Spontaneous bad mutations are {\em common}
\item Half of pregnancies never detected because spontaneously abort very early
\item Half of spontaneous abortions result from genetic problems - i.e 25 \% of all human fertilisation immediately eliminated by natural selection
\end{itemize}

\subsubsection{Weak(er) selection in humans: Single Loci}
\label{sec:07-01-03}
\begin{itemize}
\item Historically, all human adult lactose intolerant.
\item Estimate ~ 5\% fewer kids if lactose intolerant
\item New mutation arose - now most people ``lactose persistent'' (lactose tolerant) as adults.\\
  - effect of 5\% more kids - can simulate with AlleleA1 software:
  \begin{itemize}
  \item Fitness of ``AA'' (intolerant): 0.95
  \item Fitness of ``Aa'' and ``aa'' (tolerant) is 1.00
  \item Time: 5000 years\\
    - see graph at page 10 of slides 07-12-PopGen3-1
  \end{itemize}
\end{itemize}
Relative fitness: example
\begin{itemize}
  \item BB genotypes produce on average 3.2 surviving offspring
  \item Bb genotypes produce on average 3.0 surviving offspring
  \item bb genotypes produce on average 2.4 surviving offspring
\end{itemize}
Most fit genotype: BB, call it ``100\%'', fitness = w(BB) = 1.00. Others are percentages of maximum: \\
\begin{itemize}
  \item w(Bb) = 3.0/3.2 = 0.94 ( 6\% less fit than BB)
  \item w(bb) = 2.4/3.2 = 0.75 (near 25\% less fit than BB)
\end{itemize}
(then looking at different variations on Hardy - Weinberg)

\subsection{Types at a Single Loci}
\label{07-02}
\subsubsection{Selection and Dominance}
\label{07-02-01}
Imagine, we have two alleles: M and N.
\begin{itemize}
\item let all adult ``NN'' die:\\
w(MM) = 1.00; w(MN) = 1.00; w(NN) = 0\\
M is {\bf dominant}, and N is {\bf detrimental} (bad)
\item different story:\\
w(MM) = 1.00; w(MN) = 0; w(NN) = 0\\
N is dominant (and is still ``bad'')
\item now more interesting:\\
w(MM) = 1.00; w(MN) = 0.5; w(NN) = 0\\
- ``no dominance''
\end{itemize}
- See pages 6, 7, 8 of lecture slides 07-12-PopGen3-2 for graphs of Dominant / detrimental and recessive detrimental\\
\paragraph{Types of selection on different locus}

\begin{itemize}
\item Directional selection: one allele eventually replaces the other
\item Heterozygote advantage (also called ``over-dominance''): the heterozygote genotype is the most fit, like:\\
w(AA) = 0.85; w(Aa) = 1.00; w(aa) = 0.05\\
One allele {\bf does not} replace the other but alleles go to some ``equilibrium'' frequencies.\\
Prediction of equilibrium frequency for ``a`` if:\\
$$q(a) = \frac{1 - w(AA)}{(1-w(aa)) + (1-w(AA))} $$
(normalised to w(Aa) - ???)

\item Heterozygote disadvantage (AKA ``underdominance''): the heterozygote is the {\bf least} fit genotype:\\
w(AA) = 1.0; w(Aa) = 0.2; w(aa) = 0.5\\
Here we have unstable equilibrium (like 0.272727): if start below this value, go to loss for one of alleles. If started above it - lose the another allele (see pages 22, 23, 24 for slides 3-2).
And only if started {\bf exactly} on the equilibrium point - you stay there (pure theoretical case)
\item Frequency dependent selection: sometimes it is better to be ``rare''
\end{itemize}

\subsection{Types acting on Traits}
\label{sec:07-03}
Natural selection can also be studied in the context of phenotypes; don't necessarily know underlying genes to infer type of selection operating.
\begin{itemize}
\item Directional selection: individuals at one end of distribution favoured (i.e. ``big'' or ``small''). Cause change of {\bf mean} of population over time
\item Stabilising selection (phenotypes): individuals in {\bf middle} of distribution favoured. No change in mean but loss of extremes.
\item Disruptive selection (phenotypes): individuals at {\bf both ends} of distribution favoured. No change in mean but loss of intermediate phenotypes.
\end{itemize}
So natural selection preferentially reduces / eliminates ``bad'' genotypes. Average fitness of all individuals {\bf remaining in population} after selection goes up - since ``bad alleles'' are removed, simple directional selection gives long - term improvement to population.

\paragraph{Fisher's fundamental theorem of natural selection}
The rate of increase in fitness is equal to the genetic variance in fitness.

\section{Genetic Drift: Sampling Error over single generations}
\label{sec:07-04}
\subsection{Basics}
\label{sec:07-04-01}
{\bf Genetic Drift} is basically a ``sampling error'' in allele frequency over a single generation. Natural selection is {\bf predictable}: some genotypes have a higher fitness => higher fitness leads to more offspring => ``good'' genotypes become ``over-represented''.\\
But not all evolutionary changes are predictable.
Frequently a small (not-perfectly-representative) sample of gametes form the next generation => allele and genotype frequencies change.\\
Effect compounds over time: it compounds and relates to the population size. Greater changes occur in the allele frequency if the sample (population) is smaller.
\begin{itemize}
\item Variance in allele frequency due to one generation of drift: $=(pq)/(2N)$
\item Standard deviation is (slight over-) estimate of average allele in allele frequency change in one generation: $=\sqrt{(pq)/(2N)}$ (square root of variance).
{\bf Use this for estimation of expected changes in frequency}
\end{itemize}
So, finalising:
\begin{enumerate}
\item Drift is strongest in small populations
\item Drift is neither predictable in direction in one generation nor exactly replicable in degree (under exact same conditions, get different results from genetic drift)
\item drift can cause big changes in allele frequency over time (see next sub-section)
\end{enumerate}

% video 7-5
\subsection{Sampling error over many generations}
\label{sec:07-05}
Even starting from p(A) = 0.5; p(a0) = 0.5, we can end up with p(A) = 0 or 1 due to drift.
Once you get there, there will be {\bf no variation} left (one allele is gone), so can't ``drift back''.\\
On the long run, it is possible to predict the probability of a ``long-term'' outcome: in one generation, rough equally likely for allele frequency p(A) to go up or down\\
But long-term ``loss'' or ``fixation'' of allele is more predictable:
\begin{itemize}
\item if p(A) = 0.5, equally likely
\item if p(A) < 0.5, more likely that allele to be lost
\item if p(A) > 0.5, more likely that allele to be fixed
\end{itemize}
{\bf Probability of eventual fixation of A equals P(A)!} - drift eventually leads to allele fixation or loss in every population.
If the population size is small (i.e. drift is strong), genetic drift {\bf can} sometimes counteract weak selection to spread or fix ``bad'' allele.

\paragraph{Founder Effect}
- a strong genetic drift when a new population is established by a very small number of individuals from a larger population.
Often associated with colonising islands.
Sometimes causes spread (or fixation) of even detrimental alleles since drift is strong.

% video 7-6
\subsection{Rate of neutral molecular evolution}
\label{sec:07-06}
New mutations arise at some rate; mutations in some parts of genome have no effect on fitness => might spread or be lost by drift.\\
The idea is, can we predict the {\bf rate} at which they arise and spread to fixation?\\
Introduce:\\
Mutations arise at rate $\mu$ (``mutations per year'' or ``mutations per generation''), example: $\mu = 1 \times 10^{-9}$ mutations per year per basepair studied.\\
In bigger populations, more likely to get a mutation: rate $2N\mu$ of {\bf getting} new mutations (2 - because every organism has 2 chromosoms).\\
Mutation must then also fix by drift: probability of new mutation arising $\times$ probability of new mutation fixing:
$$= 2 N \mu \times 1/(2N) = \mu $$
So:
\begin{itemize}
\item large populations have more chance mutation will arise, but smaller chance it will fix
\item rate of neutral molecular evolution does not depend on population size!
\end{itemize}

\paragraph{Application of this calculation}
Mutation rate for human pseudo-genes is roughly $1 \times 10^{-9}$ mutations/year/bp.
If we want to know divergence time between humans and mouse lemurs, we sequence a pseudo-gene and see 150 base differences in 1000 bp between human and mouse lemur.
Pseudo-genes are the ones that are no longer functional - so mutations there are neutral:\\
$1 \times 10^{-9}$ mutations / bp / year => $1 \times 10^{-6}$ mutations in 1000 bp / year => $10^6$ years / 1 mutation / 1000 bp\\
Seeing 150 mutations: 150 mutations $\times 10^6$ years / mutation = $1.5 \times 10^8$ years total divergence\\
But, {\bf two} branches: $7.5 \times 10^7$ years to ancestor (75 million years)
See www.timetree.org for more samples and calculations

% video 8-1
\section{Evolutionary Advantage of Sex and Recombination}
\label{sec:08}
\subsection{Sex and Recombination}
\label{sec:08-01}
Asexual reproduction vs sexual reproduction:
\begin{itemize}
\item Asexual reproduction: fission (single cell splits into two copies), budding (little ``baby'' hydras coming off the adult hydra), parthenogenesis (offspring from unfertilised eggs): produces ``clones'', genetically identical
\item Sexual reproduction: union of genetic material from two distinct gametes: genetic material then shuffled in their gametes by recombination.
\end{itemize}
Despite a lot of benefits of asexual reproduction, the benefits from recombination are too serious:
\begin{enumerate}
\item Recombination makes {\bf combination of alleles } across two or more loci that may be advantageous - especially important with {\em epistasis} (interaction between loci) favoring a specific combination of alleles at the two loci
\item Recombination helps get rid of bad mutations to create mutation-free offspring. Without recombination, mutations accumulate => the whole population gets worse (sicker) and worse every generation (process called ``Muller's ratchet'' - there is no way to clean the mutated genes).
\end{enumerate}
{\bf Recombination accelerates adaptation:}
\begin{itemize}
\item Recombination can produce advantageous combinations of alleles
\item Recombination can accelerate adaptation
\item Recombination allows the population to ``unload'' itself from bad mutations (stopping the ``ratchet'')
\item Recombination may be {\em partially} helpful in variable environments
\end{itemize}

% video 8-2
\subsection{Recombination, Selective Sweeps and Hitchhiking}
\label{08-02}
This is about how recombination affects variations in {\bf neutral} (having no effect on fitness) sequences.\\
Molecular variation is being measured by $\pi$ - average number of pairwise mismatch (analogous to 2pq in Hardy- Weinberg): see pages 4 - 9 of slides 1-2 for calculation example.
$\pi$ {\em per site} is greater when there are more bases differing among individuals.
% video 8-2 05:30
``Selective sweep'' is spread of an advantageous allele throughout population by natural selection and associated loss of variation near it.\\
``Hitchhiking'' is spread of other nearby alleles along with the advantageous one because of linkage (lack of recombination) - see pp 11 - 15.\\
Recombination allows the neutral alleles to maintain variation.
Summarising:
\begin{itemize}
\item Adaptive alleles can ``sweep'' (spreading over the whole population)
\item SNP/marker alleles {\bf near} adaptive allele (with {\bf zero or low recombination}) hitchhike - $\pi$ typically reduced
\item SNP/marker alleles {\bf far} from adaptive allele (with {\bf high recombination}) don't hitchhike - $\pi$ not typically reduced
\end{itemize}

% video 8-3
\subsection{Signatures of Past Natural Selection in the Genome}
\label{sec:08-03}
Recombination rates not constant across genome (``crossover hot-spots''; some regions have more hotspots than others); often lower near centromere.
There is a {\bf positive } association between recombination rate and $\pi$: high cM/Mb => high $\pi$.\\
{\bf But} in species with mostly asexual reproduction (yeast) - no significant positive or negative relationship.
\paragraph{Can we levarage this?}

What if we see very low $\pi$ in a small region of high recombination?
There is a number of possible explanation, one of which is a recent selective sweep.
This can be used to find genes that experienced recent selection.\\
Alternative hypothesis: background selection (ta-dam).

% video 08-04
\subsection{Background Selection}
\label{sec:08-04}
The idea that beneficial mutations around the genome cause the association of recombination rate to $\pi$ is popular, but what if beneficial mutations are very rare?
Than we won't see a strong overall trend: many low recombination regions may still have high $\pi$.\\
The alternative  hypothesis:
\begin{itemize}
\item ``Bad'' mutation are {\bf very} common
\item When a bad mutation happens, it dooms the alleles near it to {\bf eventual death}
\item The sixe of the doomed window varies with recombination rate
\end{itemize}
- so-called ``background selection'': a lot more variability is retained because bad mutations can be lost without ``dragging away'' so much of the rest of the chromosomes - pp 5 - 10 of slides 1-4.\\
Background selection prediction depends on high rate of bad mutation.
Regions of high recombination (close hotspots) will preserve more variation (high $\pi$) - exactly the same prediction as hitchhiking.

\paragraph{The big debate}

\begin{itemize}
\item Could be {\bf Selective sweeps \& hitchhiking}: new advantageous variants arise frequently, and their fixation takes away variation.
Neutral theory says these are very rare.
\item Could be {\bf Background selection}: new bad mutations arise very frequently, and their elimination takes away variation.
Neutral theory says no problem with this.
\end{itemize}
- an area of active research today.

% video 08-05
\subsection{Challenges in Searching for Signatures of Natural Selection at Individual Genes}
\label{sec:08-05}
The common question is: do particular gene evolves primarily as a result of natural selection or genetic drift.
And how many differences we see between species were driven by selection vs drift?
Provokingly, what gene changes made us ``human''?
(As nucleotide variations exists within species and between species).
Or how much of the variation observed within and between species is ``neutral'' (evolved via drift) vs. ``selected'':
\begin{itemize}
\item {\bf Neutralists} - most nucleotide variation within and between species is neutral
\item {\bf Selectionists} - very little nucleotide variation is neutral - most variation is selected
\end{itemize}
Human genome sequenced - 2003; chimpanzee genome sequenced - 2005.
Similarity: 98.77\% in nucleotides.
We (evolutionary biologists) want to know what specific gene changes were irrelevant (drift) vs important (selection): differ at about 47 million bases.

% video 08-05 03:10
\subsubsection{Neutral Theory of Molecular Evolution}
\label{sec:08-05}
{\bf Most} mutations that get abundant and eventually get fixed have {\bf no effect on fitness}: arise via mutation and spread via genetic drift (not via selection).\\
Corollary: most nucleotide differences between species spread by drift instead of selection.\\
This {\bf doesn't} say that all mutations are neutral; it acknowledges rare adaptive mutations and common bad ones, but the most have little or no effect.\\
But not all sequence differences matter: base change in pseudo-genes and introns (often) have no effect on phenotype; {\bf Codon third-position} changes often don't change amino acid (synonymous); second position always changes (non-synonymous).\\
Still synonymous sites usually accumulate differences faster: they happen less often but far more likely to spread because not selected against.\\
Mutation rates not the same in all genes: some genes get more mutations than others, irrespective of selection => if genes get more mutations, then likely to accumulate more differences (both synonymous and nonsynonymous) between species, not telling you much about selection or ``importance''.\\
The fix: scale using number of synonymous changes (as synonymous differences accumulate neutrally) - can use them to scale for mutation rate differences.
Ratio of non-synonymous to synonymous differences estimates non-neutral changes relative to neutral changes.

% video 08-06
\subsection{Searching for Natural Selection on Individual Genes}
\label{sec:08-06}
\subsubsection{ dN/dS}
\label{sec:08-06-1}

\begin{itemize}
\item dN = number of non-synonymous changes {\bf per non-synonymous site}
\item dS = number of synonymous changes {\bf per synonymous site}
\end{itemize}
video 08-06, starting from 03:00 - long mechanic (pp 4 - 8 and 11-13 of slides 2-2).\\
If some changes do affect aminoacid and some do not - divide and add to totals as shown in p 10 of slides 2-2.
\paragraph{Meaning of dN/dS value}

- estimates how much ``non-neutral'' (nonsynonymous) evolution has happened relative to ``neutral'' (synonymous) evolution:
\begin{itemize}
\item if gene is evolving neutrally, than dN/dS = 1 (no selection on nonsynonymous changes (null hypothesis))
\item if a gene has dN/dS < 1, then changes are being ``constrained'': non-synonymous changes selected against
\item if a gene has dN/dS > 1, then changing rapidly (multiple non-synonymous changes favoured by natural selection)
\end{itemize}
- ``try this'' starting from 09:00 on 08-06: for Aspm gene (affecting brain size in humans) the real value dN/dS $\approx 0.9$.
This does not mean this gene likely to evolve {\em neutrally} (aminoacid changes ``don't matter''); this means that both constraint and rapid evolution may operate within a gene at different bases => this muddies a single ``generalisation'' about a gene as a whole.
Returning:
\begin{itemize}
\item if dN/dS = 1, then ``cannot reject'' neutrality: unlikely any protein-coding gene would be evolving neutrally
\item if dN/dS < 1, then lots of ``constraint'': {\em most} aminoacid changes {\bf disfavoured}
\item if dN/dS > 1, then selection driving rapid change: {\em multiple } aminoacid changes {\bf favoured}
\end{itemize}
- so we need another test, as dN/dS can be too conservative for finding adaptive amino-acid changes (gives many false negatives)

% video 08-07
\subsection{McDonald-Kreitman Test}
\label{sec:08-07}
dN/dS can be too conservative for finding adaptive aminoacid changes (many false negatives: mixing constraint with favoured changes and neutralisation)\\
Neutral theory predicts that ratio of nonsynonymous to synonymous changes should be constant through time: ratio observed among individuals {\em within} species should be equal to ration observed {\em between} species.

\paragraph{Neutral Theory Assumption}

most non-deleterious non-synonymous mutations are neutral:
\begin{itemize}
\item Non-deleterious (neutral) non-synonymous mutations behave just like synonymous mutations
\item arise at a relatively constant rate (assumption)
\item fix with same probability as synonymous: 1/(2N)
\end{itemize}
McDonald-Kreitman test is test if NS:S ratio is constant within and between species: contrast {\em present} (within specie) and {\em historical} (between species).
Align sets of sequences and identify if {\bf variable} nucleotide sites have:
\begin{itemize}
\item Nonsynonymous differences {\em between} species (A)
\item Nonsynonymous differences {\em within} species (B)
\item Synonymous differences {\em between} species (C)
\item Synonymous differences {\em within} species (D)
\end{itemize}
03:10 - ``try it'', pp 5-12 of slides 2-3\\
\begin{itemize}
\item if all non-synonymous differences are neutral, expect A/C = B/D (``can't reject neutrality'')
\item if some non-synonymous differences between species were {\bf advantageous \& selected}, expect A/C > B/D.
This means increased nonsynonymous changes between species - ``positive selection''
\item if maladaptive non-synonymous differences persist within species, expect A/C < B/D - decreased nonsynonymous changes between species - ``negative selection''\\
If there is proportionally {\bf more} non-synonymous variable sites within species than between species?
This means we have variation that persists but doesn't fix - hard to eliminate unfit recessive allele.
Diseases tend to be recessive; mutations causing disease stick around but don't fix.
\end{itemize}
% video 08-07 08:58
Continue human - chimp genomes comparing:
\begin{itemize}
\item Positively selected genes (304): immunity protein genes, gamete formation genes, sensory perception genes.
\item Negatively selected genes (813): many involved in cytoskeleton formation: associated with diseases like muscular distrophy, congenial deafness, cardiovascular disease.
\end{itemize}
One more to try - page 21 of slides 2-3.\\
Many other metrics also exists, like Tajama's (looking at frequencies of alleles to test for recent selection or recent changes in population size).\\
Two general approaches to studying:
\begin{itemize}
\item Scan whole genome using these metrics and look for possible selection, then try to interoret
\item Look at ``candidate genes'' (i.e. brain size, speech) and look for signature of selection.
\end{itemize}

%video 9-1

\section{Optimality}
\label{sec:9-1}

{\bf Animal behaviour} integrates a number of aspects.
{\bf Optimality theory} - achieving maximal effect for minimum cost.
Not always work as many traits / behaviours are not perfectly adopted:
\begin{itemize}
\item failure of appropriate mutations to occur
\item Single genes cause multiple phenotypic effects (pleiotropy): one allele good for X trait but bad for Y trait
\item Insufficient time/changing environment
\item etc...
\end{itemize}
- Optimality predictions must be tested \& judged, not presumed
\section{Optimality}
\label{9-1}

\subsection{Adaptive Feeding}
\label{sec:9-1-0}
The considerations going into optimal feeding theory:
\begin{itemize}
\item Plus: calories from food (energy)
\item Minus: energy for getting food: energy used searching, energy used handling, energy used eating \& digesting
\item Minus: time involved getting food (time for searching, handling, eating \& digesting)
\end{itemize}
% video 9-1 11:30
- see 18:05, 18:35 for links
% video 9-2

\subsection{Communication}
\label{sec:9-2-0}
Communication - signal sent from one individual to another with effect of altering recipient's behaviour.\\
Types:
\begin{itemize}
\item Foraging
\item Alarm calls
\item Sexual behaviours
\item Treat
\end{itemize}
Dishonest signalling (attempts to manipulate others) like give warning call of predators so others flee while you get food or fool opponent into thinking you're stronger that you are difficult to maintain:
\begin{itemize}
\item Selection favours ignoring them
\item as ``ignoring'' spreads, no longer elicit responses
\end{itemize}
So conditions for dishonest signals:
\begin{itemize}
\item Cost for not responding high
\item Rare relative to dishonest signal
\end{itemize}
example - firefly at video 9-2, 05:00

\subsection{Parental Care}
\label{sec:09-02-1}
Females usually do most of the parental care:
\begin{itemize}
\item females more likely to provide food
\item females more likely to defend kids
\item Why?
  \begin{itemize}
  \item Paternity certainty often lower than maternity
  \item Males can ``sneak off'' briefly to mate more often in short time-period
  \item Females invest more in eggs, so worth more to continue to invest
  \end{itemize}
\end{itemize}
% video 9-3

\subsection{Keeping Species Separated (Speciation)}
\label{sec:09-03}
- also species formation.\\
Evolution has two fundamental processes:
\begin{itemize}
\item change with a lineage
\item formation of new lineages (associated with split of existing lineage)
\end{itemize}
The species are defined as ``gene pools'': groups of interbreeding natural populations that do not {\em exchange genes} with other such groups - so-called {\bf Biological Species Concept}.\\
{\bf Barrier traits} - separate gene pools in two ways:
\begin{enumerate}
\item Interbreeding doesn't happen at all
\item Interbreeding does not result in gene exchange for other reasons:
  \begin{itemize}
  \item Sperm does not fertilise eggs of other species
  \item hybrids die early in life
  \item Hybrids live but are sterile
  \end{itemize}
\end{enumerate}
Hybridising species can exchange some genes, but not any (``not others'').
Some ``hybrid zones'' have persisted for thousand of years.

% video 09-04
\subsubsection{Speciation: Effect of Geography}
\label{sec:09-04}
Models of species formation:
\begin{itemize}
\item Geographic isolation (one population is being separated, two sub-populations evolve separated: random mutations, genetic drift, different environment on two sides => different gene forms are favoured => sub-populations evolve into separate species) - i.e. the gene flow is cut for some time
\item Geographic isolation but regain contact before speciation: changes happen within populations on opposite sides, come back into contact but now a little different => continued divergence and formation of barrier traits (conception of reinforcement)
\item No geographical isolation: one population, partitioning into distinct types, interbreeding reduced => continued divergence and formation of barrier traits.
The split requires {\bf strong} natural selection: distinct niches, filled by types in which intermediates (or switchers) are less fit (trade-offs in adaptation).
\end{itemize}
In a middle of video: long story about ``Co-occurring species show higher mating discrimination than geographically separated ones (variation among species)''.

% video 9-5

\subsection{Genetic Control}
\label{sec:09-05}

The genetic basis of species formation\\
Recall: genetic of species formation = genetics of barrier traits.

\paragraph{Why are hybrids sterile?}

\begin{itemize}
\item Hybrids have only the alleles of their parent sppecies
\item No gene ``functions'' to cause sterility; more likely disruption of a normal function
\item likely, {\em interactions between allele(s) from one species with allele(s) from the other}
\item can map sterility within the genome through QTL mapping (?)
\end{itemize}

Single gene speciation is very difficult (if heterozygotes Aa sterile and starting population is AA, then mutation to ``a'' will be cut off) - under-dominance, or ``selection favours loss of rare allele''.

Workaround: epistasis (interaction) between two or more loci.(pages 8-9 of slides 09-17/1-3 for diagram).\\

When one hybrid sex is sterile or inviable, it tends to be the XY sex (Haldane's rule, 1921, one of the most consistent rules in evol. biology) - sort of ``created by recessivity of genes on X''.
Explanation: 2-locus epistasis involving the X-chromosomes and autosomes causes hybrid sterility (and many other hybrid problems).

Darwin: Hybrid sterility ``is not a specially endowed quality but is incidental on other acquired differences''.
It is cause by a hybrid's ``organisation having been disturbed by two organisation having been compounded into one''
\end{document}