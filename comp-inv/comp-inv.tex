%\documentclass{ncc}
%\documentclass{article}
\documentclass{scrartcl}

\usepackage{amsmath,amssymb,amsfonts} % Typical maths resource packages
\usepackage{graphics}                 % Packages to allow inclusion of graphics
\usepackage{color}                    % For creating coloured text and background
\usepackage{hyperref}                 % For creating hyperlinks in
                                % cross references 

\usepackage{algorithm}
\usepackage{algorithmic}

\usepackage[T2A]{fontenc}
\usepackage[utf8]{inputenc}
\usepackage[russian,english]{babel}
\usepackage{listings}
\lstdefinelanguage{scala}{
  morekeywords={abstract,case,catch,class,def,%
    do,else,extends,false,final,finally,%
    for,if,implicit,import,match,mixin,%
    new,null,object,override,package,%
    private,protected,requires,return,sealed,%
    super,this,throw,trait,true,try,%
    type,val,var,while,with,yield},
  otherkeywords={=>,<-,<\%,<:,>:,\#,@},
  sensitive=true,
  morecomment=[l]{//},
  morecomment=[n]{/*}{*/},
  morestring=[b]",
  morestring=[b]',
  morestring=[b]"""
}
\usepackage{color}
\definecolor{dkgreen}{rgb}{0,0.6,0}
\definecolor{gray}{rgb}{0.5,0.5,0.5}
\definecolor{mauve}{rgb}{0.58,0,0.82}
  
% Default settings for code listings
\lstset{frame=tb,
  language=scala,
  aboveskip=3mm,
  belowskip=3mm,
  showstringspaces=false,
  columns=flexible,
  basicstyle={\small\ttfamily},
  numbers=none,
  numberstyle=\tiny\color{gray},
  keywordstyle=\color{blue},
  commentstyle=\color{dkgreen},
  stringstyle=\color{mauve},
  frame=single,
  breaklines=true,
  breakatwhitespace=true
  tabsize=3
}

\newcommand{\example}{\subparagraph{Example:}} % Example
\newcommand{\term}[1]{\verb~#1~} % Term

\begin{document}
\part{Part I}
\label{part:Intro}
They promised the Part II with more geeky staff!
Course info - wiki.quantsoftware.org
Books: ``Active Portfolio Management''(for deeper look); ``All about Hedge Funds''.

% video 2-2
\section{Portfolio Management Basics}
\label{sec:Basics}
\subsection{Portfolio Management at a Glance}
\label{sec:PortfolioManagement}
\term{ETF} - Electronically Traded Fund.
There are (basically) two possible incentives for Portfolio Manager / Hedge
Fund:
\begin{itemize}
\item By \term{Expense ratio}
  \begin{itemize}
  \item Used mostly by mutual funds and ETFs
  \item The percentage is less then 1\% of all the funds managed
  \end{itemize}
When managing such fund/ETF, the main goal becomes attracting investments.
\item ``Two and twenty'' - the standard structure for hedge fund. Includes 2\%
  of the total assets plus 20\% of profits.
\end{itemize}

\paragraph{Attracting investors}
Know your audience:
\begin{itemize}
\item Individuals (probably the smallest component)
\item Institutions:
  \begin{itemize}
  \item Harvard Foundation
  \item CalPERS
  \end{itemize}
\item Funds of Funds
\end{itemize}

Attracting investors:
\begin{itemize}
\item Must have a track record
\item Very compelling story and back test - provide a good simulation result 
\item Fit in a ``pigeon hole''? (i.e. fit the expectation of investor on a
  particular type of assets)
\end{itemize}

Two main types of fund goals:
\begin{itemize}
\item Reference to a benchmark (pigeon hole)
\item Absolute return
\end{itemize}

% video 2-3
\subsection{Common Metrics for Hedge Funds}
\label{sec:CommonMetricsHedgeFunds}
Common Metrics:
\begin{itemize}
\item Annual Return - in percent
$$ metric = value[end] / value[start] - 1$$
- can be yearly, daily etc.
\item Risk: standard deviation of return (volatility)
For particular day $i$:
$$daily\_rets[i] = value[i]/value[i-1]-1$$
$$std\_metric = stdev(daily\_rets)$$
\item Risk: Draw down (D-down): ``when a market(benchmark index) goes down - how
  much does our fund goes down?'' (proportion of the overall fund size)  
- being calculated as ``Max Draw down'': 
\item Reward / Risk: Sharpe Ratio:  
\item Reward / Risk: Sortino Ratio: only counts volatility when it is downward
\item Jensen's Alpha
\end{itemize}

% video 2-4

\subsection{Sharpe Ratio}
\label{sec:SharpeRatio}

Used to differentiate similar portfolios (imagine their returns metrics are
similar). The Sharpe Ratio is calculated from daily returns and standard
deviation metrics:
$$daily_rets[i] = (value[i]/value[i-1]) - 1$$
$$std\_metric = stdev(daily\_rets)$$
Properties:
\begin{itemize}
\item most ``important'' measure of asset performance (the higher - the better) 
\item how well does the return of an asset compensate the investors for the risk
  taken?
\item when comparing two assets with the same return, higher Sharper ratio gives
  more return for the same risk
\end{itemize}

$$S = \frac{E[R-R_f]}{\sigma} = \frac{E[R-R_f]}{\sqrt{var[R-R_f]}} $$
here:
\begin{itemize}
\item $E[R - R_f]$ - expected return:
  \begin{itemize}
  \item  $R_f$ - a ``risk-free return'' (return for low-risk asset); quite often
    it is ignored
  \item $R$ - return itself
  \end{itemize}
\item $var[R-R_f]$ - the volatility, or the standard deviation of day-to-day
  changes in return
\end{itemize}
a bit simplified version: $k *
\frac{average\_daily\_return}{std(daily\_return)}$, or 
$$ metric = k * mean(daily\_rets)/stdev(daily\_rets)$$
where $k = \sqrt{250}$ for daily returns (250 - amount of working days in a
trading year). For monthly returns, use $k = \sqrt{12}$

% video 2-5
\subsubsection{Homework \& Demo with Excel}
\label{sec:Excel}
Getting data: yahoo finance, Stock ticker -> Historical Prices, ``Download as
Spreadsheet''. 
Columns: Open, High, Low, Close, Volume, Adj(Adjusted) Close. We work with {\bf
  Adj Close}. Sort by Date ascending (it is generated in descending order).

Then calculate:
\begin{itemize}
\item $Total\_return=final\_value/starting\_value$ (in \%)
\item for the first day = 0;$Daily_returns[i] = return[i]/return[i-1] - 1$ 
\item Sharpe ratio: first get $=AVERAGE(daily_return)$, then
  $=STDEV(daily_returns)$. Finally,
  $Sharpe\_ratio=sqrt(250)*[avg\_daily\_ret]/[stdev\_daily\_ret]$ 
\end{itemize}

% video 2-6
\subsection{Homework 1}
\label{sec:Homework1}

\begin{itemize}
\item Find online broker to ``paper trade'' (google/yahoo finance)
\item Invest \$1M in 4 equities.
\item Access portfolio for 2011:
  \begin{itemize}
  \item Annual Return
  \item Average daily return
  \item Cumulative return: $price[i]/starting\_price$ 
  \item Investment: $original\_investment * cumulative\_return$
  \item STDEV of daily return
  \item Sharpe Ration
  \end{itemize}
\item Compare with benchmark: SPY
\item Submit:
  \begin{itemize}
  \item .pdf printout of the spreadsheet
  \item Screen-shot of portfolio online
  \end{itemize}
\end{itemize}
(see trick with relative investment values at video 2-6, 07:10).
The highest Sharpe ration comes up to 4 - try to beat it (ho-ho).

% video 3-1
\section{Mechanics of the Market}
\label{sec:MechanicsOfMarkets}
Types of orders at the Exchange:
\begin{itemize}
\item Symbol (name of the equity)
\item Buy / Sell
\item Market order (take whatever price is available)/ Limit (max/minimum price
  for buying/selling respectively order)
\item Shares (how many?)
\item Price (if Limit order)
\end{itemize}

\subsection{The Order Book}
\label{sec:OrderBook}
The \term{Order Book} - something like a blackboard at the exchange where
players post their Limit orders where:
\begin{itemize}
\item \term{Ask} price - ``ready to sell by this price''
\item \term{Bid} price - ``ready to buy by this price''
\end{itemize}
The gap between maximum Bid and minimum Ask prices is called \term{Spread}.
Usually for very liquid (?) assets this is quite small.  
The order book remains static until one of the event occurs:
\begin{itemize}
\item One of the buying parties goes ahead and order buying by the Asking price
  - so-called \term{Crossing the Spread}
\item Market order comes in
\end{itemize}

% video 3-2
If there a lot more shares to sell then to buy - that means that the price is
going to go down (the high-frequency trading is about that).

\subsection{How Brokers Connected to Exchange}
\label{sec:BrokerConnectionToExchange}

Quite often the order from Trader (you) does not even reach the Exchange - it is
being fulfilled withing the Broker. This can make them some coins:
\begin{itemize}
\item every order placed on exchange costs broker some money - they can save it
  a bit
\item sometimes broker can make some coins on margin between Bid and Ask price
  for the same symbol from different traders.
\end{itemize}

Sometimes brokers pass orders not directly to \term{Stock Exchange(SE)} but to
the \term{Market Makers(MM)}, and they fulfil them internally. 

The SE itself accepts only very basic orders - like Buy/Sell Market/Limit. There
are a lot more complex types of orders which may occur, usually through the
broker:
\begin{itemize}
\item Sell short (to open): bet against stock
\item Buy (to close)
\item More complex orders (stop limit)
\end{itemize}

\subsubsection{Mechanics of Short Selling}
\label{sec:MechanicsOfShortSelling}
Shorting the stock - when you believe it is going to go down:
\begin{itemize}
\item Borrow the shares
\item Sell them. Now we have:
  \begin{itemize}
  \item Asset: Cash
  \item Liability: Shares owed
  \end{itemize}
\item Share price drops. Buy the shares at lower price (close the transaction). 
\item Return shares. Profit (C).
\end{itemize}
There is a theoretical limit of what you can make: if the price of a stock goes
all the way to 0, you get the whole price of your sell. However, if the market
price goes up, there is no limit for your loss.

%video 3-3
\subsection{Hedge finds and Arbitrage}
\label{sec:HedgeFundsAndArbitrage}

\subsubsection{Order Book Observation}
\label{sec:OrderBookObservation}
Classical: if there are much more players trying to sell some stock hen trying
to buy, there is a good chance that the price will go down - it makes sense to
stay in short. To catch-up with such imbalances, often Hedge Funds rent the
co-located rack space on SE for observing the order books.

\subsubsection{Arbitrage}
\label{sec:HedgeFundsAndArbitrage}
Imagine there are two SE trading the same shares. The hedge fund can execute
transaction between these SE can make money on such differences.

% video 3-4
\subsection{Computational Components of a Hedge Fund}
\label{sec:HFAnatomy}
We're working on slides 3-4. 
\paragraph{Page 2 (Slower) Quant Shop.}
Trading algorithm is connected to order book, Live Portfolio, Historical Price
Data; is parametrised (at the moment of opening) with the Target Portfolio. 
Trading algorithm works to bring the Live Portfolio to the Target one. The trick
is not to disturb market too much - so orders are issues slowly. (\term{Fat
  Finger} - a trader who worked with large amounts of equities so ``prices going
mad'').

Step back: how do we get the Target Portfolio?
\paragraph{Page 3 Quant Shop}
The centre is the Portfolio Optimiser algorithm, fed by data from forecasts
(sometimes insider), Current Portfolio, Historical Price Data, and Risk
Constraints. 

Nice, but the forecast comes from (sometimes) people like ``Cramer on TV'', or 
\paragraph{Page 4: Forecasting Algorithm}
Often built of top of some Machine Learning systems, fed with some information
feeds (like News) and Historical Price Data.

Some news providers are providing the news optimised for forecasting: in a
format ``Equity symbol, bad news/good news''. See TR (Thomas Reuters?) for
detail. 

% video 3-5 (part 4-1)
\subsection{Company Value}
\label{sec:CompanValue}
The ``traditional'' ways are:
\begin{itemize}
\item Market cap (shares * share price)
\item Future dividends 

So, imagine the \term{Notional Company} which prints \$1/year each year,
forever. The company worth is figured out by a Current Value of a the future
\$1. 
Assume that discount rate is 0.95 (it is lower then the banking rate, but we
trust bank more). So the \$1 in next year costs 0.95 today. The \$1 in two years
costs even less: $9.95^2$. Next year - $0.95^3$, etc. 

So the present value will be $$\sum_{i=1}^\infty 0.95^i=\sum_{i=1}^\infty
dividend * \gamma^i = dividend * \frac 1{(1-\gamma)} = \frac{\$1}{1-0.95}=\$20$$

Various information (like CEO effectiveness, cost of raw materials) informs
traders about company's capacity to make money in the future.
\item Book value
\end{itemize}

% video 3-6 (part 4-2)

\end{document}