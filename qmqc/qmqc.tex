%\documentclass{ncc}
%\documentclass{article}
\documentclass{scrartcl}

 % Typical maths resource packages
\usepackage{amsmath,amssymb,amsfonts,amscd}
 % Packages to allow inclusion of graphics
\usepackage{graphics}                
% For creating coloured text and background
\usepackage{color}                    
% For creating hyperlinks in cross  references 
\usepackage{hyperref}                 

\usepackage{algorithm, algorithmic}
\usepackage{textcomp}

\usepackage[T2A]{fontenc}
\usepackage[utf8]{inputenc}
\usepackage[russian,english]{babel}
\usepackage{listings}
\lstloadlanguages {[LaTeX]TeX, Octave}
\lstset {language=[LaTeX]TeX,
  extendedchars=true ,escapechar=|}

\newcommand{\cplx}{\mathbb{C}} % Complex numbers space
\newcommand{\bigR}{\mathbb{R}} % Real numbers space
\newcommand{\hilbert}{\mathbb{H}} % Hilbert space
\newcommand{\ket}[1]{\left| #1 \right>} % for Dirac kets
\newcommand{\bra}[1]{\left< #1 \right|} % for Dirac bras
\newcommand{\braket}[2]{\left< #1 | #2 \right>} % for Dirac brakets

\begin{document}

\section{Axioms}
\label{sec:2, week 1}

\subsection {Superposition Principle}
\label{sec:2-1}

\begin{itemize}
\item{\bf Superposition axiom} Suppose we have a k-level quantum system
  \begin{itemize}
  \item k distinguishable or classical states of the system
  \item Possible classical states $\ket{0},\ket{1}, \dots , \ket{k-1}$
  \end{itemize}
\item {\bf Superposition principle:} if a system can be in one of the k states,
  it can also be in any linear superposition of k states: \[\alpha_0 \ket{0} +
  \alpha_1 \ket{1} + \dots + \alpha_{k-1} \ket{k-1} \] where $\alpha_i \in
  \cplx$ - complex ``probability amplitude'' for each of these
  states: \[\sum \limits_{i=0}^{k-1} |\alpha|^2=1\]

  Examples for k = 3:

  $\frac 1{\sqrt{3}} \ket0 + \frac1{\sqrt{3}} \ket1 + \frac1{\sqrt{3}}\ket2$ or
  $\frac1{2}\ket0 - \frac1 {\sqrt{2}}\ket1 + (\frac1{\sqrt{2}} +
  \frac{i}{\sqrt{2}}) \ket2$
\end{itemize}

\subsection{Measurement axiom}
\begin{itemize}
\item Suppose our system is in state $$\ket\psi = \alpha_0\ket0 + \alpha_1\ket1
  + \ldots + \alpha_{k-1} \ket{k-1} $$
\item Measure: outcome is one of the k classical states
  \begin{itemize}
  \item Observe j with probability $|\alpha_j|^2$.
  \item New state = $\ket j$.
  \end{itemize}
\end{itemize}

\paragraph{Qubit}

Two-level systems (k=2) are called {\bf qubits}. These can be presented as
system where electron can be either in ``ground'' state $\ket0$ or in excited
state $\ket1$. The ``common'' state is a superposition $\alpha_0 \ket0 +
\alpha_1 \ket1$

\subsubsection{Geometrical Interpretation}
\label{sec:2-2}

The notation of state (linear superposition) can be presented also:
\[\alpha_0 \ket0 + \alpha_1 \ket1 + \dots +
\alpha_{k-1} \ket{k-1} = \begin{pmatrix} \alpha_0 \\ \alpha_1 \\ \vdots \\
  \alpha_{k-1} \end{pmatrix}\] As so we can present the state as a vector in
some space of k-measures where axes are vectors representing $i^{th}$ states:
$\begin{pmatrix} 1 \\ 0 \\ 0 \end{pmatrix}$ represents the system in ground
state ($\ket0$), $\begin{pmatrix} 0 \\ 1 \\ 0 \end{pmatrix}$ represents $1^{st}$
excited state ($\ket1$) etc.

So, the quantum state of a k-level system is a unit vector (which means it's
size is equal to 1) in k-dimensional complex Gilbert space: $\vec u \in
\cplx^k$

For qubits: $\ket\psi = \alpha_0 \ket0 + \alpha \ket1$ can be represented in two
- dimensional space as a vector of length 1 and angle $\theta$. Projections of
this vector are the probabilities of the system to be in the relevant state
after measurement:
\[ Pr[0] = \alpha_0^2 = \cos^2 \theta; Pr[1] = \alpha_1^2 = sin^2 \theta\]

\subsubsection{Measurement in an arbitrary basis}
\label{sec:2-3}

For given vector $\psi$ representing a state of a qubit we can use arbitrary
basis (let's call it u, it should consist of axis $\ket u$ and
$\ket{u^{\perp}}$) and do our measurements in this new basis u:
\begin{itemize}
\item $\ket u = \beta_0\ket0 + \beta_1\ket1$
\item $\ket {u^{\perp}} = -\beta_1\ket0 + \beta_0\ket1$
\end{itemize}

In this new basis, the bow angle will be $\theta'$ and probabilities for
outcomes of measurement will be:
\begin{itemize}
\item $u$ with probability $\cos^2 \theta' \leftrightarrow (\ket u, |\psi
  \rangle)^2 $
\item $u^{\perp}$ with probability $\sin^2 \theta' \leftrightarrow
  (|u^{\perp}\rangle, |\psi \rangle)^2$
\end{itemize}

\paragraph{Example } Hadamard basis (so - called ``sign basis''). A basis turned
$-\frac\pi4$:
\begin{itemize}
\item $\ket+ = \frac1{\sqrt{2}} \ket0 + \frac1{\sqrt{2}} \ket1$
\item $\ket- = \frac1{\sqrt{2}} \ket0 - \frac1{\sqrt{2}} \ket1$
\end{itemize}

Curiosity: can we distinguish $\ket+$ from $\ket-$ through a measurement in the
standard basis? The problem is that for state $\ket+$ probabilities for $\ket0$
and $\ket1$ are both $\frac1{2}$ - and so for $\ket-$. The idea is to used ``yet
another'' basis - namely, switch to sign basis in this case.

So measure (for example) $\ket\psi = \frac1{2} \ket0 + \frac{\sqrt{3}}{2}\ket1$
in $\ket+ / \ket-$ basis can be found by good old linear algebra:
\begin{enumerate}
\item Compute inner product:
  \[ Pr[+] = |(\ket\psi, \ket+)|^2 = \left|\left(\frac1{2}\ket0 +
      \frac{\sqrt{3}}{2}\ket1, \frac1{\sqrt{2}}\ket0 + \frac1{\sqrt{2}}\ket1
    \right)\right|^2 =\]
  \[ = \left| \frac1{2 \sqrt{2}} + \frac{\sqrt{3}}{2\sqrt{2}} \right|^2 =
  \frac{(1 + \sqrt{3})} {8} = \frac{4 + 2 \sqrt{3}}8 \]

  $ Pr[-] = |(\ket\psi, \ket-)|^2$, or simpler $\dots = 1 - Pr[+] = \frac{4 - 2
    \sqrt{3}}{8}$
\item Change of basis: rewrite state in new basis $\ket\psi = \beta_0 \ket+ +
  \beta_1 \ket-$ so that $Pr[+] = |\beta_0|^2, Pr[-] = |\beta_1|^2$. Then,
  accepting that
  \begin{itemize}
  \item $\ket+ = \frac1{\sqrt{2}} \ket0 + \frac1{\sqrt{2}} \ket1$
  \item $\ket- = \frac1{\sqrt{2}} \ket0 - \frac1{\sqrt{2}} \ket1$
  \end{itemize}
  put these into formula, after simplifications:
  \[ Pr[+] = | \frac1{2 \sqrt{2}} + \frac{\sqrt{3}}{2\sqrt{2}} |^2 = \frac{4 + 2
    \sqrt{3}}{8} \]
  \[ Pr[-] = | \frac1{2 \sqrt{2}} - \frac{\sqrt{3}}{2\sqrt{2}} |^2 = \frac{4 - 2
    \sqrt{3}}{8} \]
\end{enumerate}

\subsection{Uncertainity principle}
\label{sec:2-4}

\paragraph{Heisenberg}forever: One can never know with perfect accuracy both of
those two important factors which determine the movement of one of the smallest
particles - its position and its velocity.

Similar story for qubits (hydrogen atom) :
\begin{itemize}
\item it has energy (Bit) : $\ket0$ or $\ket1$
\item it has sign : $\ket+$ or $\ket-$
\end{itemize}

It is impossible to know both bit and sign of qubit, because if we have value
for, say, bit, this will provide maximum uncertainty for sign (0.5 - 0.5
probability) and vice verse: if we know sign (and thus fix electron in either
state) we get uncertainly for the any other basis.

Thus, these two basis are incompatible with each other (as well as any two
different basis).

To quantify this we define a {\bf spread} of a quantum state.
\[\ket\psi = \alpha_0 \ket0 + \alpha_1 \ket1 = \beta_0 \ket+ + \beta_1
\ket- \]
\begin{itemize}
\item The spread in standard basis $:= S(|\psi|) = |\alpha_0| + |\alpha_1|$
\item The spread in sign basis $:= \hat{S}(|\psi|) = |\beta_0| + |\beta_1|$
\end{itemize}
For fixed states:
\begin{itemize}
\item In standard basis:
  \begin{itemize}
  \item $S(\ket0) = 1 + 0 = 1$
  \item $S(\ket+) = \frac1{\sqrt{2}} + \frac1{\sqrt{2}} = \sqrt{2}$
  \end{itemize}
\item In sign basis:
  \begin{itemize}
  \item $\hat{S}(\ket0) = \frac1{\sqrt{2}} + \frac1{\sqrt{2}} = \sqrt{2}$
  \item $\hat{S}(\ket+) = 1 + 0 = 1$
  \end{itemize}
\end{itemize}

Uncertainty principle for bit and sign: $S(|\psi\rangle) \hat{S}(|\psi\rangle)
\geq \sqrt{2}$ for any $|\psi\rangle$

\section{Two Qubits \& Entanglement}
\label{sec:week 2}
Two qubit system is classically represent with two bits of information: $00, 01,
10, 11$, so quantum state is a superposition over all four classical
possibilities: $$|\psi \rangle = \alpha_{00}|00\rangle + \alpha_{01}|01\rangle +
\alpha_{10}|10\rangle + \alpha_{11}|11\rangle $$, and $$|\alpha_{00}|^2 +
|\alpha_{01}|^2 + |\alpha_{10}|^2 + |\alpha_{11}|^2 = 1$$
 
Observer j with probability $|\alpha_j|^2$. New state = $|j\rangle$

\subsubsection{Partial Measurement}
\label{sec:3-2}

What is the result of measurement {\bf just one} qubit? I.e. the probability to
see the first qubit as 0 is $$P[0] = |\alpha_{00}|^2 + |\alpha_{01}|^2$$

But what about the new state? Just discard the states which are not possible any
more, and re-normalise probabilities: $$New State = \frac{\alpha_{00}|00\rangle
  + \alpha_{01}|01\rangle }{\sqrt{|\alpha_{00}|^2 + \alpha_{01}|^2} } $$

\subsection{Entanglement}
\label{sec:3-3}

Imagine a composite system: qubit 1 with state $\alpha_{0}\ket0 +
\alpha_{1}\ket1$ and qubit 2 with state $\beta_{0}\ket0 + \beta_{1}\ket1$. The
composite state will be \[ (\alpha_{0}\ket0 + \alpha_{1}\ket1) (\beta_{0}\ket0
+\beta_{1}\ket1) = \alpha_0 \beta_0 |00\rangle + \alpha_0 \beta_1 |01\rangle +
\alpha_1\beta_0 |10\rangle + \alpha_1 \beta_1 |11\rangle \]

Now back: Given the state of the composite system, determine the state of each
qubit. In theory, we have 4 variables ($\alpha_0, \alpha_1, \beta_0, \beta_1$),
4 products - so we can calculate. The problem is - not all composite states are
possible to ``de-compose''; such ``en-decomposing'' states are called {\bf
  entangled}.

For such states, each qubit does not have its own separate state - it is not in
some definite state. All we can say is that the whose system is in some
superposition quantum state.

Canonical example - {\bf Bell State}
$$|\psi\rangle = \frac1{\sqrt{2}} |00\rangle + \frac1{\sqrt{2}}|11\rangle $$
- the formula implies that $\alpha_0, \beta_0, \alpha_1, \beta_1$ are non-zero
which contradicts the fact $\alpha_0 \beta_1 = 0; \alpha_1 \beta_0 = 0$.

\paragraph{Measuring the Bell State}

Measure first qubit:
\begin{itemize}
\item see 0 with probability $\frac1{2}$. New state = $|00\rangle$
\item see 1 with probability $\frac1{2}$. New state = $|11\rangle$
\end{itemize}
The trick is: let's separate qubits by a big distance. The measure first of them
- let's say, we get 0. Then measure the second one (we are expecting to get 0
with probability 1). As such, we have coincidence in particle states -
regardless of the distance between particles (something like if two particles
share a coin flip before they are separated).

\subsection{EPR Paradox}
\label{sec:3-4}

Einstein, Podolsky, Rosen (1935). Was introduced to show problems with QM
theory. The idea is: we can rewrite the state of a Bell State using the sign
basis:
$$|\psi\rangle = \frac1{\sqrt{2}} |00\rangle + \frac1{\sqrt{2}}|11\rangle =
\frac1{\sqrt{2}} |++\rangle + \frac1{\sqrt{2}}|--\rangle $$

Then, assuming the trick from previous paragraph:
\begin{itemize}
\item Faster then light communication?
\item If when two particles were together they flipped two coins to coordinate
  how they would answer on bit and sign measurement, then they violate
  uncertainty principle (measure first qubit in bit basis and second qubit in
  sign basis; know both bit and sign for first qubit contradicts uncertainty
  principle).
\end{itemize}
Variants for solution:
\paragraph{Local Hidden Variable Theory:} The two particles carry with them all
the information necessary to locally decide the outcome of any future
measurements.

- contradicted by Bell's experiment

\paragraph{Quantum Mechanics:} As soon as the first qubit is measured, say in
the bit basis, the entanglement between the two qubits is destroyed. The new
state is either $|00\rangle$ or $|11\rangle$, each of which is unentangled.
Measuring the second qubit in the sign basis no longer reveals any information
about the first qubit. (Does not constitute faster than light communication,
since no information is communicated).


\section{Bell's Experiment}
\label{sec:4}
John Bell devised an experiment with one of two outcomes:
\begin{enumerate}
\item Nature is inconsistent with quantum mechanics but might be better
  explained by some local hidden variable theory
\item Nature is consistent with quantum mechanics but inconsistent with any
  local hidden variable theory
\end{enumerate}
The Bell experiment relies on properties of entangled qubits. Performed numerous
times, outcome is second (consistent with quantum mechanic).
 
\subsection{Experiment Setup}
\label{sec:4-1}
Two boxes situated far apart (light should not have enough time to travel
between during experiment => time of experiment should be short): first one gets
input x (0 or 1), second box gets input y (also 0 or 1). Both boxes are set up
to output a bit: a for first box, b for second.

Experiment: if both boxes are give 1 input, we expect both boxes to output the
different values ($a \neq b$). Otherwise, want output of the boxes to be the
same The idea is that
\begin{itemize}
\item if boxes are described by local hidden variables, the experiment should be
  successful with probability $\leq \frac{3}{4} = 0.75$

  Proof: Fail on any of the 4 inputs implies success probability $\leq
  \frac{3}{4}$ Suppose output on x=y=0 is a=b=0. By locality, output on x=0, y=1
  and x=1, y=0 must also be a=b=0.

  Finally by locality, output on x=y=1 must also be a=b=0 => contradiction.

\item But, if boxes share a Bell state, then experiment can succeed with
  probability as high as 0.85. See next section for proof.
\end{itemize}

\subsection{Properties of Bell state}
\label{sec:4-2}

\subsubsection{Rotational invariance of Bell State}
\label{sec:4-2-1}

The idea is to measure state of both qubits in some basis $(u, u^{\perp})$.
Initially, the probabilities are the same: $P[u] = \frac1{2}, P[u^{\perp}] =
\frac1{2}$. But, if measured state of qubit 1 is u, then probability of qubit 2
to be in state u is P[u] = 1.

Now add yet one more basis: $(v, v^{\perp})$ - rotated from $(u, u^{\perp})$ by
angle $\theta$. If qubit 1 is in state $u$, the probability for qubit 2 to be in
state $v$ is $\cos^2 \theta$.

Proof: The state of a qubit can be expressed as: $$|\psi\rangle =
\frac1{\sqrt{2}} |00\rangle + \frac1{\sqrt{2}}|11\rangle = \frac1{\sqrt{2}}
|uu\rangle + \frac1{\sqrt{2}}|u^{\perp}u^{\perp}\rangle$$ - through $\ket u = a
\ket0 + b \ket1;|u^{\perp}\rangle = -b \ket0 + a \ket1; a, b \in R; a^2 + b^2 =
1$ and re-calculating

For complex a and b, use different form of Bell's state: $$ |\psi\rangle =
\frac1{\sqrt{2}} | 01 \rangle - \frac1{\sqrt{2}} | 10 \rangle $$, then $$\ket u
= \alpha \ket0 + \beta \ket1; |u^{\perp}\rangle = -\beta^* \ket0 + \alpha^*
\ket1;$$ $ \alpha, \beta \in C; |\alpha|^2 + |\beta|^2 = 1$ - complex
conjugates.

\subsubsection{Bell's Experiment}
\label{sec:4-3}

In box 1, we measure the state of a qubit in a basis depending on the input
value (select basis U if input is 1 and some other if 0). In box 2 - select yet
another basis V if input is 1 and other is input is 0).

If y = 1 (second box is given 1), then P[matching output] = $cos^2 \theta$ where
$\theta$ is the angle between basis U and V (see previous subsection).

The trick is to select relevant basis's in a manner to make such outcome
maximally probable. One of the variants is: Box 1: basis $U_0$ (to be used if
x=0) is the ``normal'' basis; basis $U_1$ will be turned $\pi / 4$

Box 2: Basis $V_0$ (for y = 0): turned by $\pi /8$; basis $V_1$ (for y =1)
turned by $ - \pi / 8$

So, the probabilities will be:
\begin{tabular}{cc|c}
  \textbf{ x=} & \textbf{ y = } & \textbf{P(match)} \\
  \hline
  0 & 0  & $cos^2 \pi/8$ \\
  0 & 1  & $cos^2 \pi/8$ \\
  1 & 0  & $cos^2 \pi/8$ \\
  1 & 1  & $cos^2 3\pi/8 = sin^2 \pi/8$ \\
\end{tabular}

In last case, P(not match) = $cos^2 \pi / 8$; so for the whole experiment
$$P[success] = cos^2 \pi / 8 \approx 0.85$$

Returning to original formulations:
\begin{itemize}
\item Success probability $\leq 3/4 \Rightarrow$ Nature is inconsistent with
  quantum mechanics but might be better explained by some local hidden variable
  theory
\item Success probability $> 3/4 \Rightarrow$ Nature is consistent with quantum
  mechanics and inconsistent with any local hidden variable theory
\end{itemize}
The Bell experiment has been performed numerous time; the result have always
been consistent with quantum mechanics. Einstein wasted near 20 years of hos
life.

\subsubsection{Certifying Randomness}
\label{sec:5-1}

{\bf Task:} Construct a physical random generator whose output can be certified
to be random. The Bell experiment is used to certify the ``true randomness'' of
the outputs.

The idea is to check if Bell's experiment is successful with probability at
least 0.85. Details - see papers mentioned at (5-1 10:10)

\subsection{Evolution of a Qubit}
\label{sec:5-2}

Allowable states of k-level system: unit vector in a k-dimensional complex
vector space (called a Hilbert space):

\[\ket\psi = \alpha_0\ket0 + \dots + \alpha_{k-1}\ket{k-1} =
\begin{pmatrix} \alpha_0 \\ \vdots \\ \alpha_{k-1} \end{pmatrix} \in
\cplx^k\]


\subsection{Axioms of Quantum Mechanics}
\label{sec:5-2}
... re-formulation
\begin{enumerate}
\item {\bf Superposition principle: } allowable states of k-level system: unit
  vector in a k-dimensional complex vector space (called a Hilbert space)

\item {\bf Measurement}

  \begin{itemize}
  \item A measurement is specified by choosing an orthonormal basis.
  \item The probability of each outcome is the square of the length of the
    projection onto the corresponding basis vector.
  \item The state collapses to the observed basis vector
  \item If $\ket{u_0}, \ket{u_1}, \dots, \ket{u_{k-1}}$ was the chosen basis and
    $\ket\psi = \alpha \ket{u_0} + \alpha \ket{u_1} + \dots + \alpha_{k-1} |
    u_{k-1}\rangle$, then qubit is in the state $u_i$ with probability
    $|\alpha_i|^2 = |(|\psi \rangle,|u_i\rangle)|^2$. New state is
    $|u_i\rangle$.
  \end{itemize}


\item {\bf Unitary evolution}

  Quantum systems evolve by rotating of the Hibert space (so-called rigid body
  rotation, meaning that the angles between vectors are preserved). As rotation
  of the space is a linear transformation, it can be represented by a matrix.

  Example: turning basis by angle $\theta$. Coordinate vectors: $\ket0 \to \cos
  \theta \ket0 + \sin \theta \ket1;\ket0 \to - \sin \theta \ket0 + \cos \theta
  \ket1$, so

\[\bigR_\theta = \left( \begin{array}{cc}\cos\theta &-\sin\theta \\ \sin
    \theta & \cos\theta\end{array}  \right)\]

Later on, \[ \bigR_{- \theta} = \left( \begin{array}{cc} \cos \theta & \sin
    \theta \\ -\sin \theta & \cos \theta \end{array} \right) =
\bigR_\theta^T \] and \[ \bigR_\theta \bigR_{-\theta} =
\mathbf{I}; \bigR_\theta \bigR_{\theta}^T = \bigR_\theta^T
\bigR_{\theta} = \mathbf{I} \]
\end{enumerate}

\subsection{Unitary Transformation}
\label{sec:5-3}

Unitary transformation is a rotation of a qubit state vector in a complex space.
The rotation matrix is: $U = \begin{pmatrix} a & c \\ b & d \end{pmatrix}; a, b,
c, d \in \cplx$.

If our state is represented be real (not complex) parameters, then
transformation can be expressed as rotation by angle $\theta$: $R_\theta =
\begin{pmatrix} \cos \theta & -\sin\theta \\ \sin\theta &
  \cos\theta \end{pmatrix}$

Examples:
\begin{itemize}
\item $\ket\psi \to U\ket\psi$
\item for $\ket0$ vector: $ \dbinom{a}{b} = \ket0 \to \dbinom ab = a\ket0 + b
  \ket1$
\item for $\ket1$ vector: $\ket1 \to = c\ket0 + d \ket1$
\end{itemize}

By definition, a {\bf conjugate transpose} matrix $U^\dag$ is obtained from $U$
by taking the transpose and then taking the {\it complex conjugate} (that is,
negating their imaginary parts bu not their real parts) of each entry. In
addition, if for given matrix $A$ it's inverse matrix $A^{-1} = A^\dag$, that is
$A \times A^\dag = I$, then matrix $A$ is called {\bf unitary matrix}.

The main idea is that angles are preserved, and orthogonal vectors remains
orthogonal: $|a|^2 + |b|^2 = 1 = |c|^2 + |d|^2$, and $a^*c + b^*d = 0$

Another property - {\bf Linearity}, i.e. if
\begin{gather*}
  U\ket0 = \ket{\phi_0} = a\ket0 + b \ket1 \\
  U\ket1 = \ket{\phi_1} = c\ket0 + d \ket1 \end{gather*} then
\[\begin{split}
  U(\alpha \ket0 + \beta \ket1) &= \alpha\ket{\phi_0} + \beta\ket{\phi_1} \\ &=
  \alpha (a\ket0 + b \ket1) + \beta (c\ket0 + d\ket1) \\ &= (\alpha a + \beta b)
  \ket0 + (\alpha b + \beta d) \ket1 \end{split}\]

\subsubsection{Single Qubit Gates}
\label{sec:5-4}

\paragraph{Quantum gates} transforms qubits passed to them. Examples:
\begin{itemize}
\item ``Bit flip'': $\alpha_0 \ket0 + \alpha_1 \ket1 \to \alpha_1 \ket0 +
  \alpha_0 \ket1 $ Transformation matrix:
  \begin{gather*}
    \mathbf{X} = \left( \begin{array}{cc} 0 & 1 \\ 1 & 0 \end{array} \right);
    \mathbf{X} \ket0 = \ket1; \mathbf{X} \ket1 = \ket0; \\
    \mbox{ and } \mathbf{X}^T = \left( \begin{array}{cc} 0 & 1 \\ 1 &
        0 \end{array} \right) = \mathbf{X}; \mathbf{X}^2 = \mathbf(I)
  \end{gather*}

  Can be considered as a reflection (rotation) of vector over axis tilted at
  45\textdegree.

\item ``Phase flip'': $\alpha_0 \ket0 + \alpha_1 \ket1 \to \alpha_0 \ket0 -
  \alpha_1 \ket1 $ Transformation matrix:
  \begin{gather*}
    \mathbf{Z} = \left( \begin{array}{cc} 1 & 0 \\ 0 & -1 \end{array} \right);
    \mathbf{Z} \ket0 = \ket0; \mathbf{Z} \ket1 = - \ket1; \\
    \mbox{ again } \mathbf{Z}^T = \mathbf{Z}; \mathbf{Z}^2 = \mathbf(I)
  \end{gather*}

  Reflection over $\ket0$ axis. BTW, \begin{gather*} \mathbf{Z} \ket+ = \ket-;
    \\ \mathbf{Z} \ket- = \ket+
  \end{gather*} - Phase flip swaps the axis of the sign basis.

\item ``Hadamard transform''

  \begin{gather*}
    \mathbf{H} = \left( \begin{array}{cc} \frac1{\sqrt{2}} & \frac1{\sqrt{2}} \\
        \frac1{\sqrt{2}} & - \frac1{\sqrt{2}}
      \end{array} \right);\\
    \mathbf{H} \ket0 = \ket+ = \frac1{\sqrt{2}} \ket0 +
    \frac1{\sqrt{2}} \ket1 \\
    \mathbf{H} \ket1 = \ket- = \frac1{\sqrt{2}} \ket0 -
    \frac1{\sqrt{2}} \ket1 \\
    \mbox{ again }  \mathbf{H}^T = \mathbf{H}; \mathbf{H}^2 = \mathbf(I) \\
    \mathbf{H} \ket+ = \ket0; \mathbf{H} \ket- = \ket1\\
  \end{gather*}

  Reflection over $\pi / 8 $ axis.
\end{itemize}

Now we have a bit of tricks: if we do not have a Z gate, we can emulate one with
the chain $\mathbf{H} \to \mathbf{X} \to \mathbf{H}$, in other words,
$\mathbf{Z} = \mathbf{H} \mathbf{X} \mathbf{H}$. Big picture:

$$\begin{array}{ccccc}
  &\ket0 & \stackrel{X}{---} & \ket1 & \\
  & | & & | &  \\
  H & | & & | & H \\
  & | & & | & \\
  &\ket+ &  \stackrel{Z}{---} & \ket- & \\
\end{array}$$

\subsubsection{No cloning theorem}
\label{sec:5-5}

It is impossible to create a qubit with exactly the same state as some arbitrary
state.

\subsubsection{Two qubit gate}
\label{sec:5-5-1}

Transformation U transforms the two=qubit state $|\psi \rangle$ into another
two-qubit state $|\phi\rangle$. Such transformation is set by 4 x 4 matrix (U)
that $\mathbf{U} |\psi_1 \rangle = |\phi_1 \rangle; \mathbf{U} | \psi_2 \rangle
= | \phi_2 \rangle$.

Then, if $|\psi \rangle = |\psi_1 \rangle + | \psi_2 \rangle$, then
$\mathbf{U}|\psi \rangle = |\phi_1 \rangle + |\phi_2 \rangle$

Now try to build a gate (circuit ?) to copy a quantum bit:
\subsubsection{Quantum circuit for copying a quantum bit}
\label{sec:5-5-2}

Input: unknown state $|\psi\rangle = a \ket0 + b \ket1$. Put this and ``clean
qubit'' $\ket0$, expected output should be: \begin{equation}(a \ket0 + b
  \ket1)(a \ket0 + b \ket1) = a^2 |00\rangle + ab |01 \rangle + ab |10 \rangle +
  b^2|11\rangle \label{copy-eq} \end{equation}

Cases:
\begin{itemize}
\item $|\psi \rangle = \ket0$, expected output $|00\rangle$
\item $|\psi \rangle = \ket1$, expected output $|11\rangle$
\item Then, by linearity, if input is $|\psi \rangle = a \ket0 + b \ket1$ then
  output should be $a |00\rangle + b |11\rangle$
\end{itemize}
The last must be equal to (\ref{copy-eq}), so it must be ab=0 which is possible
only if a=0 or b=0.

So, {\bf the fact that you can clone $\ket0 \mbox{ or } \ket1$ prevents you from
  copying the ``normal'' state } or

\paragraph{No cloning theorem} It is impossible to clone an unknown quantum
state.

Or, even stronger, it is not possible to construct a two qubit of the same
states from the one qubit in this state and one arbitrary one.

\section{Quantum circuits and teleportation}
\label{sec:6}

\subsection{CNOT gate + circuits}
\label{sec:6-1,2}

One-qubit gate U can be specified either by matrix U or by
\begin{gather*} \ket0 \to \dots \\ \ket1 \to \dots \end{gather*} - because it is
a linear transformation.

For two-qubit gates: $$\alpha_{00} |00\rangle + \alpha_{01} |01\rangle +
\alpha_{[10} |10\rangle + \alpha_{11} |11\rangle \to \alpha_{00}' |00\rangle +
\alpha_{01}' |01\rangle + \alpha_{10}' |10\rangle + \alpha_{11}' |11\rangle$$

So that U is the 4-dimensional unitary ($UU^\dag = U^\dag U = I$) matrix.

\subsubsection{CNOT gate}
\label{sec:6-1-1}

The first bit (a AKA control) goes unchanged: $a \to a$; the second one (b AKA
Target bit) is being flipped if a is 1 and NOT flipped otherwise: $b \to b
\oplus a$.

So: $\begin{array}{ccc}
  |00\rangle & \to & |00\rangle \\
  |01\rangle & \to & |01\rangle \\
  |10\rangle & \to & |11\rangle \\
  |11\rangle & \to & |10\rangle \\
\end{array}$, and superposition is being mapped as: $\begin{array}{c}
  \alpha_{00} |00\rangle + \alpha_{01} |01\rangle + \alpha_{[10} |10\rangle +
  \alpha_{11} |11\rangle \\
  \downarrow \mbox{ CNOT } \\
  \alpha_{00} |00\rangle + \alpha_{01} |01\rangle + \alpha_{11} |10\rangle +
  \alpha_{10} |11\rangle 
\end{array}$

The matrix looks even more interesting: $\mathbf{CNOT} = \left(
  \begin{array}{cccc}
    1 & 0 & 0 & 0 \\ 0 & 1 & 0 & 0 \\ 0 & 0 & 0 & 1 \\ 0 & 0 & 1 & 0
  \end{array}
\right)$,

and $\mbox{CNOT CNOT}^\dag = I$

\subsubsection{Quantum circuit}
\label{sec:6-1-2}

Let's play with Hadamard gate $\binom{H\ket0=\ket+=\frac1{\sqrt2}\ket0 +
  \frac1{\sqrt2}\ket1}{H\ket1=\ket-=\frac1{\sqrt2}\ket0 - \frac1{\sqrt2}\ket1}$
on qubit 1 followed by CNOT on both. Transformations:
\begin{equation*} \begin{split} \ket{00} \to \mbox{Step 1: } & (\frac1{\sqrt{2}}
    \ket0 + \frac1{\sqrt{2}} \ket1 ) ( \ket0 ) =
    \frac1{\sqrt{2}} \ket{00} + \frac1{\sqrt{2}} \ket{10} \\
    \to \mbox{Step 2: } & \mbox{flip $2^{nd}$ qubit when the $1^{st}$ is 1: getting Bell state: }\\
    \to &\frac1{\sqrt{2}}(\ket{00} + \ket{11}) \\
  \end{split}\end{equation*}

By playing the input states, we can get all four Bell states:
\begin{align*}
  \ket{00} \to \ket{\phi^+} = \frac1{\sqrt{2}} (\ket{00} + \ket{11})\\
  \ket{10} \to \ket{\phi^-} = \frac1{\sqrt{2}} (\ket{00} - \ket{11})\\
  \ket{01} \to \ket{\psi^+} = \frac1{\sqrt{2}} (\ket{01} + \ket{10})\\
  \ket{11} \to \ket{\psi^-} = \frac1{\sqrt{2}} (\ket{01} - \ket{10})
\end{align*}
- so-called Bell basis states

As Hadamard and CNOT are their own inversions, the same input can go ``from
right to left'': feed the circuit with, say, $\ket{\phi^+}$ and get $\ket{00}$
at the left. So it is possible to restore the initial state - measure qubits in
bit basis after passing it through ``CNOT -> Hadamard'' circuit.

\subsection{Teleportation}
\label{sec:6-3}

It is impossible to clone quantum information but it is possible to teleport a
quantum state to another location.

Settings: Alice has a qubit is state $\ket\psi = \alpha\ket0 + \beta\ket1$,which
she wants to transfer. In addition, Alice and Bob create a pair in Bell state:
$\ket\psi = \frac1{\sqrt{2}} \ket{00} + \frac1{\sqrt{2}}\ket{11}$

Then Alice measures both her qubit $\psi$ and her qubit from the Bell pair with
outcome, of two bits: $b_1, b_2$ - one of (00, 01, 10, 11). This iformation is
passed to Bob.

Bob performs some (depending on $b_1, b_2$) unitary transformation on his part
of Bell state - in such a way that outcome is $\alpha\ket0 + \beta\ket1$ - so
the qubit destroyed by Anna during her measurement ``suddenly'' re-appears on
the Bob's end. So we have a ``teleportation'' of quantum state.

% point 6-3 08:26

\subsubsection{Mechanics: Assume CNOT}
\label{sec:6-3-1}

Imagine we can create an entanglement pair out from Alice's $\alpha\ket0 +
\beta\ket1$ and Bob's ``pure'' $\ket0$, and put it through CNOT gate. The
outcome will be $\alpha\ket{00} + \beta\ket{11}$.

Then Alice measures her qubit {\it in sign basis}. The outcome for the pair will
be:
\[\begin{split}
  \alpha \ket{00} + \beta \ket{11}) &= \alpha\left(\frac1{\sqrt{2}}\ket+ +
    \frac1{\sqrt{2}}\ket-\right)(\ket0) +
  \beta\left(\frac1{\sqrt{2}}\ket+ - \frac1{\sqrt{2}}\ket-\right)(\ket0)  \\
  &= \frac1{\sqrt{2}}\ket+(\alpha\ket0 + \beta\ket1) +
  \frac1{\sqrt{2}}\ket-(\alpha\ket0 - \beta\ket1)
\end{split}\] So:
\begin{itemize}
\item if Alice measured $+$, Bob gets $\alpha\ket0 + \beta\ket1$ - job done
\item if Alice measured $-$, Bob gets $\alpha\ket0 - \beta\ket1$ - need to apply
  the ``phase flip'' Z gate.
\end{itemize}
- now we have a challenge to create the entangled state $\alpha\ket{00} +
\beta\ket{11}$ without quantum communication between Alice and Bob.

\subsubsection{Mechanics: Implementation}
\label{sec:6-4}

So task is to create a $\alpha\ket{00} + \beta\ket{11}$ state. Suppose we have
two qubits sharing a Bell state. Can we use it to effectively apply CNOT
remotely?
\begin{itemize}
\item {\bf Step 1 (initial):} Alice has her source (or ``unknown'') qubit
  $\alpha\ket0 + \beta\ket1$; Alice and Bob share the Bell state
  $\left(\frac1{\sqrt{2}}\ket{00} + \frac1{\sqrt{2}}\ket{11}\right)$. Common
  state: \begin{gather*}(\alpha\ket0 +
    \beta\ket1)\left(\frac1{\sqrt{2}}\ket{00} + \frac1{\sqrt{2}}\ket{11}\right) = \\
    = \frac\alpha{\sqrt{2}}\ket{000} + \frac\alpha{\sqrt{2}}\ket{011}
    +\frac\beta{\sqrt{2}}\ket{100} + \frac\beta{\sqrt{2}}\ket{111}
  \end{gather*}

\item {\bf Step 2: } Alice applies CNOT gate to her qubits (from qubit 1 to
  qubit 2): in states when qubit 1 is 0, nothing happens, when qubit 1 is 1, the
  qubit 2 flips: $$\frac\alpha{\sqrt{2}}\ket{000} +
  \frac\alpha{\sqrt{2}}\ket{011} +\frac\beta{\sqrt{2}}\ket{110} +
  \frac\beta{\sqrt{2}}\ket{101}$$

\item {\bf Step 3:} Alice measures qubit 2. Possible state outcomes:
  \begin{itemize}
  \item $0 \to \alpha\ket{00} + \beta\ket{11}$ (selected 1st and 4th parts of
    Step 2). {\bf Done:} Alice and Bob share two qubits with the same state
    $\alpha\ket0 + \beta\ket1$.
  \item $1 \to \alpha\ket{01} + \beta\ket{10}$ (selected 2nd and 3rd parts of
    Step 2). Then Bob performs a Bit flip ($X$) on his qubit, and we get
    $\alpha\ket{00} + \beta\ket{11}$ again.
  \end{itemize}

\item Alice performs measurement on unknown qubit (qubit 1) in Hadamard basis
  (i.e. applies $H$, then measures), reports outcome to Bob. If outcome is $+$,
  Bob does nothing, if it is $-$ - Bob applies ``phase flip'' - $Z$ to his
  qubit.
\end{itemize}
For the diagram of the whole protocol - see slide 6-4, page 6. Result: both
Alice's qubits are measured (``destroyed''), but her unknown state qubit is
``materialised'' on the Bob side.

\subsection{Computation by Teleportation}
\label{sec:6-5}

Order of application: Apply $ZX$ means apply first $X$, then $Z$!

I general case, it would not work to apply some arbitrary gate at the Bob's side
before teleportation - because at most cases $U \times X \neq X \times U$ (see
slides 6-5, notice U-gate at the Bob's side). For details - see
http://arxiv.org/pdf/quant-ph/9908010v1.pdf)

\section{Bra-ket notation, Eigenvectors, Tensor products}
\label{chap:7}

Euler's formula: $e^{ix} = \cos x + i \sin x$

 

Ket notation:$$\ket\psi = \begin{pmatrix} \alpha_0 \\\alpha_1 \\ \vdots \\
  \alpha_{k-1} \end{pmatrix} = \alpha_0\ket0 + \alpha_1\ket1 + \dots +
\alpha_{k-1}\ket{k-1}$$

Corresponding to this vector (called ``ket'') there exists ``bra'' vector:
$$(\alpha_0^* \enskip \alpha_1^* \enskip \dots \enskip \alpha_{k-1}^*) =
\bra\psi $$, where $\alpha_i^*$ - complex conjugates for $\alpha_i$.

Suppose we have another vector state $\ket\phi = \beta_0\ket0 + \beta_1\ket1 +
\dots + \beta_{k-1}\ket{k-1}$. The inner product of $\ket\psi$ and $\ket\phi$
will be: $$(\alpha_0^* \enskip \alpha_2^* \enskip \dots \enskip \alpha_{k-1}^*)
\times \begin{pmatrix} \beta_0 \\ \beta_1 \\ \vdots \\ \beta_{k-1} \end{pmatrix}
= \sum_j\alpha_J^* \beta_J^* = \braket\psi\phi $$

The length of a vector: $|\ket\psi| = \sqrt{ \braket\psi\psi} = \sqrt{\sum
  \alpha_j^* \alpha_j} = \sqrt{\vphantom{\sum\alpha_j^*} \sum|\alpha_j|^2}$

Another story: inner product is equal to the $\cos\theta$ between unit vectors
in complex space: $\cos \theta = \frac{|\braket\psi\phi|}{|\ket\psi||\ket\phi|};
0\leq \theta \leq \pi/2$
% video 7-2

\paragraph{Projection Matrix} P onto $\ket\phi$ - i.e. the matrix which for
given vector $\ket\psi$ provides a projection of this vector to
$\ket\phi= \begin{pmatrix}\beta_0 \\ \vdots \\ \beta_{k-1} \end{pmatrix}$
(vector!)

Such projection matrix will be $P=\begin{pmatrix} \beta_0 \\ \vdots \\
  \beta_{k-1}\end{pmatrix} (\beta_0^* \dots \beta_{k-1}^*)$ or, in bra-ket
notation, $$P=\ket\phi \bra\phi$$

Then, when applied to vector $\ket\psi = (\ket\phi\bra\phi)\ket\psi$. Further
down: $\ket\psi = (\ket\phi\bra\phi)\ket\psi = \ket\phi \times
\underbrace{\bra\phi\ket\psi}_{\text{inner product}} = \braket\phi\psi \ket\phi$
- i.e. the vector $\ket\phi$ scaled by the inner product.

Another story: $P^2 = P$ (projection of already projected vector equals to this
projection). In out notation: $P^2 = \ket\phi \bra\phi \ket\phi \bra\phi$,
assuming that middle part is inner product of the unit vector, and so equals to
1: $\ldots = \ket\phi\bra\phi = P$

\paragraph{U unitary} - by definition $$U U^\dag = U^\dag U = I$$

and rows are orthonormal (as are the columns): if $U_{.j}$ = $j^{th}$ column of
U then the vectors represented by these columns/rows are orthogonal in space of
measurement k = ``U preserves angles'' (or, rather, inner products): if
$U\ket\phi = \ket{\phi'}, U\ket\psi = \ket{\psi'}$ then
$\braket\phi\psi=\braket{\phi'}{\psi'}$

Proof: if $U\ket\phi = \ket{\phi'}, U\ket\psi = \ket{\psi'}$ then $\bra{\phi'}
=\bra\phi U^\dag$ then $\braket{\phi'}{\psi'} = \braket{\bra\phi U^\dag}{U
  \ket\psi}$. With $U U^\dag = I$ - done.

\subsection{Hermitian Matrices, eigenvectors}
\label{sec:7-3}

\paragraph{Def} Matrix A is {\bf Hermitian} $\Leftrightarrow A=A^\dag$. It has
to be square, entries on the diagonal has to be real, and entries not on the
diagonal has to be conjugates relatively to diagonal.

If A is Hermitian and Real, it is symmetric.

\paragraph{Def} Vector $\ket\phi$ is {\bf eigenvector} of matrix A if $A\ket\phi
= \lambda\ket\phi (\lambda \in \cplx)$

\paragraph{Spectral Theorem: } if A is Hermitian $\Rightarrow$ A has an
orthonormal set of eigenvectors ($\ket{\phi_0}, \dots, \ket{\phi_{k-1}}$) with
real evalues ($\lambda_0, \dots, \lambda_{k-1}$).

I.e. matrix A maps unit sphere to ellipsoid of k principle axis.
% video 7-3,time 05:40

For example, for {\bf X operator}: $X = \begin{pmatrix} 0 & 1 \\ 1&
  0 \end{pmatrix}$; obviously Hermitian. $\ket+,\ket-$ are eigenvectors for it:
\begin{itemize}
\item $X\ket+ = \begin{pmatrix} 0 & 1 \\ 1 &
    0 \end{pmatrix}\binom{\frac1{\sqrt2}}{\frac1{\sqrt2}} =
  \binom{\frac1{\sqrt2}}{\frac1{\sqrt2}}; \; \lambda_+ = 1 $
\item $X\ket- = \begin{pmatrix} 0 & 1 \\ 1 &
    0 \end{pmatrix}\binom{\frac1{\sqrt2}}{-\frac1{\sqrt2}} =
  \binom{-\frac1{\sqrt2}}{\frac1{\sqrt2}}; \; \lambda_- = -1 $
\end{itemize}

Another example: {\bf Hadamard transform}: $H = \begin{pmatrix} \frac1{\sqrt2} &
  \frac1{\sqrt2} \\ \frac1{\sqrt2}& -\frac1{\sqrt2} \end{pmatrix}$;

For arbitrary vector $\ket\phi$ that of $H \ket\phi = \lambda\ket\phi$. Than we
have: $$(H-\lambda I)\ket\phi = 0 \text{ i.e. determinant } \det(H-\lambda I) =
0$$
Then $$\det \begin{pmatrix} \frac1{\sqrt2} - \lambda & \frac1{\sqrt2} \\
  \frac1{\sqrt2} & -\frac1{\sqrt2} - \lambda \end{pmatrix} =
\left(\frac1{\sqrt2} - \lambda \right)\left( -\frac1{\sqrt2} - \lambda \right) -
\frac1{\sqrt2} \frac1{\sqrt2} = -\frac12 + \lambda^2 -\frac12 = 0$$

So $\lambda = \pm 1 $; we can select any of these value and find out the
relevant eigenvector.
% video 7-3, 11:30

So Hermitian matrix A has orthonormal set of eigenvectors $\ket{\phi_0} \dots
\ket{\phi_{k-1}}$ with real evalues $\lambda_0 \dots \lambda_{k-1}$.

If we'll write our matrix A in the $\ket{\phi_0} \dots \ket{\phi_{k-1}}$ basis,
it will be just the diagonal matrix $\Lambda = \begin{pmatrix} \lambda_0 &&&0 \\
  & \lambda_1 && \\ & & \ddots & \\ 0 &&& \lambda_{k-1} \end{pmatrix}$. Then $A
= U^\dag \Lambda U$ where $U= \begin{pmatrix} \vdots & \vdots &&\vdots \\
  \ket{\phi_0} & \ket{\phi_1} & \dots & \ket{\phi_{k-1}} \\ \vdots & \vdots
  &&\vdots \end{pmatrix}$. Then (remember, $\ket\phi \bra\phi = P$ - projection
of the vector on itself!): $A = \lambda_0 \ket{\phi_0}\bra{\phi_0} + \lambda_1
\ket{\phi_1}\bra{\phi_1} + \dots + \lambda_{k-1}
\ket{\phi_{k-1}}\bra{\phi_{k-1}} = \sum \limits_{j=0}^{k-1}\lambda_j P_j$

\subsection{Tensor Products}
\label{sec:7-4}

\paragraph{Tensor Product} is an operation of ``forming'' a Hilbert space of
possible states for two particles ($\ket{00}, \ket{01}, \ket{10}, \ket{11}$, 4 -
dimension space) out of two spaces for the states of two separate particles. So
if we have Hilbert spaces $\mathbf{H}_1 \in \cplx^2$ and $\mathbf{H}_2 \in
\cplx^2$ then ``common'' space is $\mathbf{H} = \mathbf{H}_1 \otimes
\mathbf{H}_2$; so the state vectors become: $\ket0 \otimes \ket0 = \ket{00} =
\ket0\ket0; \ket0 \otimes \ket1 = \ket{01} = \ket0\ket1$ etc.

So the state vectors in the relevant spaces are ``product-ed'' too (math
expressions are boring obvious).

Consider two vectors: $\ket{\phi_1} \otimes \ket{\phi_2}$ and $\ket{\psi_1}
\otimes \ket{\psi_2}$. Inner product of these will
be: $$(\braket{\phi_1}{\psi_1}) (\braket{\phi_2}{\psi_2})$$

To describe a state in the first system, you need k complex parameters; and l
complex parameters to describe in second.

Let $H_1 = \mathbf{C^k}$ and $H_2 = \mathbf{C^l}$. The tensor product will be a
space $H = H_1 \otimes H_2 \in \cplx^{k*l}$. To describe a state in the
first system, you need k complex parameters; and l complex parameters to
describe in second. The whole state will be a vector consisting of:
$$\begin{pmatrix}
  \ket0 \otimes \ket0 & \ket0 \otimes \ket1& \dots& \ket0 \otimes \ket{l-1} \\
  \dots &\dots & \dots & \dots \\
  \ket{k-1}\otimes \ket0 & \ket{k-1} \otimes \ket1 & \dots & \ket{k-1} \otimes
  \ket{l-1}
\end{pmatrix}$$

\subsection{Tensor Product of Operators (gates)}
\label{sec:7-5}

Imagine we apply $U= \begin{pmatrix} a & c \\ b & d \end{pmatrix}$ and
$V= \begin{pmatrix} e & g \\ f & h \end{pmatrix}$ to two vectors from different
spaces: $$U \otimes V = \begin{pmatrix} aV & cV \\ bV & dV \end{pmatrix} =
\begin{pmatrix} \left[ aV = \begin{pmatrix} ae & ag \\ af & ah \end{pmatrix}
  \right] & cV \\ bV & dV \end{pmatrix}$$

Example: U = H (Hadamard); V = X (bit flip). Then
$$H \otimes V =\begin{pmatrix} \frac1{\sqrt2}X & \frac1{\sqrt2}X \\
  \frac1{\sqrt2}X & -\frac1{\sqrt2}X \end{pmatrix} =
\begin{pmatrix}
  0 & \frac1{\sqrt2} & \; & 0 & \frac1{\sqrt2} \\
  \frac1{\sqrt2} & 0 & \; & \frac1{\sqrt2} & 0 \\
  \; \\
  0 & \frac1{\sqrt2} & \; & 0 & -\frac1{\sqrt2} \\
  \frac1{\sqrt2} & 0 & \; & -\frac1{\sqrt2} & 0 \\
\end{pmatrix}$$

\section{Observables and Schredinger's equation}
\label{sec:8}

\subsection{Observables}
\label{sec:7-1}

An {\bf observable } is a quantity like energy, position, momentum - something
which can be measured. We feed the quantum state to some measurement device and
get some real number

For some k-dimension system the observable A will be a $k \times k$ Hermitian
matrix: $A = A^\dag$. This means that, according to the Spectral Theorem, $A$
has k orthogonal vectors $\ket{\phi_0}, \ket{\phi_1} \dots \ket{\phi_{k-1}}$
with real eigenvalues $\lambda_0, \lambda_1 \dots, \lambda_{k-1}$ that
$$A \ket{\phi_i} = \lambda_i \ket{\phi_i}$$

\paragraph{Measurement}

Let $\ket\psi = \sum \alpha_i\ket{\phi_i}$. Measurement outcome is $\lambda_i$
with probability $|\alpha_i|^2$; new state $\ket{\psi_{new}} = \ket{\phi_i}$.

I.e. specifying matrix $A$ we really specify orthonormal basis with
corresponding eigenvalues. So, if our measurement results in $i$, the outcome
will be equal to $\lambda_i$

{\bf Example} let $\ket\psi = \alpha \ket0 + \beta\ket1$; Observable $X
= \begin{pmatrix} 0 & 1 \\ 1 & 0 \end{pmatrix}$. Eigenvectors of $X$ are
\begin{gather*} \ket{\phi_0} = \ket+; \; \lambda_0 = 1 \\ \ket{\phi_1} = \ket-;
  \; \lambda_1 = -1 \end{gather*}

When we measure, we get the value in the matrix's basis: $$\ket\psi =
\frac{\alpha + \beta}{\sqrt2} \ket+ + \frac{\alpha-\beta}{\sqrt2} \ket-$$

Outcome:
\begin{itemize}
\item +1 with probability $\left| \frac{\alpha + \beta}{\sqrt2}\right| ^2$; new
  state $\ket+$
\item -1 with probability $\left| \frac{\alpha - \beta}{\sqrt2}\right| ^2$; new
  state $\ket-$
\end{itemize}

So expected value is $= 1 \left| \frac{\alpha + \beta}{\sqrt2}\right| ^2 +
(-1)\left| \frac{\alpha - \beta}{\sqrt2}\right| ^2$

For the situation with {\bf repeated eigenvectors} (i.e. we have $\ket{\phi_1}$
and $\ket{\phi_2}$ with both $\lambda_1 = \lambda_2 = 1$). In this case it is
impossible to find out which of the state was measured: any other vector
$\ket\psi = \alpha\ket{\phi_1} + \beta\ket{\phi_2}$ will also be eigenvector
with $\lambda = 1$. Then the measure outcome will be a vector - projection of
$\ket\psi$ onto the plain formed by $\ket{\phi_1}, \ket{phi_2}$.
% video 8-2

\paragraph{Example}

Imagine we have good old hydrogen atom with k possible state of the electron -
as such, it's quantum state is a unit vector in $\psi_0. \psi_1, \dots,
\psi_{k-1}$. If we need to measure energy, our observable then will be a matrix
A with same states but eigenvalues equal to energies of corresponding states:
$$\begin{pmatrix} E_0 & 0 & \dots & 0 \\ 0 & E_1 & \dots & 0 \\ \\ \dots \\ 0 & 0 & \dots & E_{k-1} \end{pmatrix}$$

Another example: imagine we're going to have eigenvectors $\ket+, \ket-$ and
want them to have eigenvalues 2, -3. First step: find a projection for our new
basis:
\begin{itemize}
\item Vector $\ket+$: $\ket+\bra+ = \ket+ (\ket+)^\dag =
  \binom{\frac1{\sqrt2}}{\frac1{\sqrt2}} (\frac1{\sqrt2} \quad \frac1{\sqrt2})
  = \begin{pmatrix} \frac12 & \frac12 \\ \\ \frac12 & \frac12 \end{pmatrix}$
\item Vector $\ket-$: $\ket-\bra- = \ket- (\ket-)^\dag =
  \binom{\frac1{\sqrt2}}{-\frac1{\sqrt2}} (\frac1{\sqrt2} \quad -\frac1{\sqrt2})
  = \begin{pmatrix} \frac12 & -\frac12 \\ \\ -\frac12 & \frac12 \end{pmatrix}$
\end{itemize}
Summary: $A = 2 \ket+\bra+ + (-3) \ket-\bra- =
\begin{pmatrix} \frac52 & -\frac12 \\ -\frac12 & \frac52 \end{pmatrix} $ Let's
apply this to the $\ket+$ state: $(2 \ket+\bra+ + (-3) \ket-\bra-)\ket+ = 2\ket+
+ (-3) 0 = 2 \ket+$

\paragraph{In general: } given $\ket{\phi_i}, \lambda_i$, corresponding
observable is
$$A = \sum \lambda_i \ket{\phi_i}\bra{\phi_i}$$
- therefore equivalent to our previous notion of measurement: ``pick the
orthonormal basis''

\subsection{Expectation Value and Variance}
\label{sec:8-3}

\begin{itemize}
\item An observable M for a k-level quantum system is a $k\times k$ Hermitian
  matrix. Random value $X$ denotes outcome of measurement of state $\ket \psi =
  \sum \alpha_i \ket{phi_i}$.
\item Distribution of X: $Pr[X = \lambda_j] = |\alpha_j|^2$
\end{itemize}
So we have usual parameters:
\begin{gather*}
  \mu = E[X] \\ \sigma^2 = Var[X] = E[(X-\mu)^2]
\end{gather*}
Interpretation: if we'd rotate the shape around $\mu$ - axis, moment of inertia
will be equal to the one created by ...

So, if we have observable $M$ on state $\ket\psi$, then:
$$\mu=E[X]=\bra\psi M\ket\psi$$

\paragraph{Prof:} by definition $\mu = E[x] = \sum |\alpha_j|^2 \lambda_j$; and
$M$ is a diagonal matrix, so \begin{gather*}\bra\psi M\ket\psi = (\alpha_0^* \;
  \alpha_1^* \dots \alpha_{k-1}^*) \begin{pmatrix} \lambda_0 &&& \\ & \lambda_1
    && \\ && \ddots &\\ &&& \lambda_{k-a}\end{pmatrix} \begin{pmatrix} \alpha_0
    \\ \alpha_1
    \\ \vdots \\ \alpha_{k-1} \end{pmatrix} = \\
  = \sum \alpha_j^*\lambda_j\alpha_J = \sum \alpha_j^*\alpha_j\lambda_j = \sum
  |\alpha_j|^2 \lambda_j
\end{gather*}

Variance: \begin{gather*} \sigma^2 = Var[X] = E[X^2] - (E[X])^2 = E[X^2] - \mu^2\\
  = \bra\psi M^2 \ket\psi - (\bra\psi M \ket\psi)^2
\end{gather*}

\paragraph{Prof:} by definition $E[x^2] = \sum |\alpha_j|^2 \lambda_j^2$, and
\begin{gather*}\bra\psi M^2\ket\psi = (\alpha_0^* \; \alpha_1^*
  \dots \alpha_{k-1}^*) \begin{pmatrix} \lambda_0 &&& \\ & \lambda_1 && \\ &&
    \ddots &\\ &&& \lambda_{k-a}\end{pmatrix}^2 \begin{pmatrix} \alpha_0 \\
    \alpha_1
    \\ \vdots \\ \alpha_{k-1} \end{pmatrix} = \\
  = \sum \alpha_j^*\lambda_j^2\alpha_J = \sum \alpha_j^*\alpha_j\lambda_j^2 =
  \sum |\alpha_j|^2 \lambda_j^2
\end{gather*}

(when you square observable, it is eigenvalues which are being squared, the
eigenvectors remain the same).

\subsection{Schredinger's Equation}
\label{sec:8-4}

Axiom of unitary rotation:
\begin{itemize}
\item {\bf Unitary evolution axiom:} a quantum system evolves by unitary
  rotation of the Hibert space: $$U U^\dag = U^\dag U = I$$
\item But.. by which unitary rotation?
\item Energy observable $H$, called the Hamiltonian of the
  system \begin{itemize}
  \item Its eigenvectors $\ket{\phi_i}$'s are the states with definite energy
  \item The eigenvalues $\lambda_i$'s are the energy of the corresponding state
  \end{itemize}
\item Example $H = \begin{pmatrix} -\frac12 & \frac52 \\ \frac52 &
    -\frac12 \end{pmatrix}$
  \begin{itemize}
  \item $\ket+$ with energy = 2
  \item $\ket-$ with energy = -3
  \end{itemize}
\item The {\bf Schrodinger's equation }. If $\ket{\psi{t}}$ - the state of the
  system at time $t$, then:
$$i\hbar \frac\partial{\partial t}\ket{\psi(t)} = H\ket{\psi(t)}$$
- differential equation:
\begin{itemize}
\item $i = \sqrt{-1}$
\item $\hbar$ - Plank's constant
\end{itemize}
\end{itemize}

\paragraph{Solving Schrodinger's equation}

Let initial state $\ket{\psi(0)} = \ket{\phi_j}$ where $\ket{\phi_j}$ is some
eigenvector of $H$ with a corresponding eigenvalue $\ket{\lambda_j}$

Then $\ket{\psi(t)} = e^{-\frac{i\lambda_j t}{\hbar}} \ket{\phi_j}$ - at time
$t$ the state is the same eigenvector $\ket{\phi_j}$ with some phase
$e^{-\frac{i\lambda_j t}{\hbar}}$ associated with it (see slide 8-4 page 5).
I.e. the phase ($\theta$ angle) rotates at a rate proportional to energy.

Next, as $H \ket{\phi_j} = \lambda_j \ket{\phi_j} \Rightarrow $ at any time the
state vector points at the same direction, just with different amplitude:
$$\ket{\psi(t)} = a(t)\ket{\phi_j}$$
So, solving equation:
\begin{equation*}
  \begin{split}
    i\hbar\frac{\partial a(t)}{\partial t}\ket{\phi(t)} &= H(a(t) \ket{\phi_j})
    = \\
    &=a(t) \lambda_j \ket{\phi_j}\\
    \frac{\partial a(t)}{a(t)} &= \frac{\lambda_j}{i\hbar} \partial t\\
    \Rightarrow \quad a(t) &= e^{-\frac{i\lambda_j t}{\hbar}}
  \end{split}
\end{equation*}

Again, ``if the state starts with some direction, it always points at this
direction - with some phase precession over time''.

Of course, in general, the starting position is some superposition of states -
good old $\ket{\psi(0)} = \sum \alpha_j \ket{\phi_j}$. By linearity, this means
that $\ket{\psi(t)} = \sum \alpha_j e^{-\frac{i\lambda_j t}{\hbar}}
\ket{\phi_j}$

i.e. each state stays invariant in time - ``just precesses''.

In the eigenbasis, we can write
$$\ket{\psi(t)} =
\begin{pmatrix} e^{-\frac{i\lambda_1 t}{\hbar}} & 0 \\ \\ 0 &
  e^{-\frac{i\lambda_k t}{\hbar}} \end{pmatrix} \ket{\psi(0)}$$ - the state at
the moment $t$. The matrix (let's call it $U(t)$) is obvious unitary: $U(t)
U^\dag(t) = U^\dag(t) U(t) = I$. Shorthand notation:
$$U(t) = e^{-\frac{i H t}{\hbar}}$$
(in format: $B = e^A$ for matrices A and B means that B has the same
eigenvectors as A but eigenvalues of B are the exponents of eigenvalues for A)

\paragraph{Example}

Suppose Hamiltonian is $H = X$; start of the zero state: $\ket\psi(0) = \ket0$.
What is $\ket\psi(t)$?

Solution: The X's eigenvectors are: $\ket+$ with eigenvalue 1, $\ket-$ with
eigenvalue -1. So:
\begin{equation*} \begin{split}
    \ket{\psi(0)} &= \frac1{\sqrt2}\ket+ + \frac1{\sqrt2}\ket- \\
    \ket{\psi(t)} &= \frac1{\sqrt2}e^{-\frac{i\lambda_1 t}{\hbar}} +
    \frac1{\sqrt2}e^{-\frac{i\lambda_2 t}{\hbar}} \\
    &= \frac1{\sqrt2}e^{-\frac{i \; t}{\hbar}} + \frac1{\sqrt2}e^{\frac{i \;
        t}{\hbar}}
  \end{split} \end{equation*}

% video 8-5
The whole story about Schrodinger's equation is that it ties the evolution of
the system to the system's energy (as observable H is the ``energy''
measurement).

From the unitary evolution axiom we derive that evolution operator (U) should
look like $U=e^{-iMt}$ for some Hermitian operator $M$. Let's try to understand
why M should be H (Hamiltonian).

If A is arbitrary (``any'') observable this means that ($\equiv$) it conserves
physical quantity over time, then A must commute with M:
$$ A M = M A$$ (in general, matrix multiplication does not commute).

Let's we have a state $\ket\psi$, and after some infinitely small time t we have
a state $\ket{\psi'} = U \ket \psi = e^{-iMt}\ket\psi $.

``A is a conserved quantity'' means that $\bra\psi A \ket\psi = \bra{\psi'} A
\ket{\psi'} = \bra\psi U^\dag A U \ket\psi$.

As this holds for all $\psi$, then:
$$A = U^\dag A U = e^{iMt} A e^{-iMt}\approx (1+iMt) A (1-iMt)$$

(t is small so this ``$\approx$'' should work). Again:
$$ \dots \approx A + it[MA-AM] \Rightarrow it [MA-AM] = 0$$
(ignoring $O(t^2)$ terms coming from multiplication). So $MA = AM$, and A must
commute with M.

So, for now on: \begin{itemize}
\item $U = e^{-iMt}$ where M is Hermitian
\item if A is conserved $\Rightarrow AM = MA$
\end{itemize}
Th idea is that energy is always conserved (in contrast with, say, momentum
which can change if potential energy changed). There must be some ``intrinsic
reason'' why A and M commute.
% video 8-5, 10:21
The most natural reason would be that $H = \hbar M$ - i.e. A (in our case A is
H) is just linearly connected to M. At least, we expect $H = f(M)$, but, due to
some ``out of our scope physical reasons'' this function $f$ must be linear. For
the case: imagine we have a system consisting of two destroying pieces 1 and 2,
having associated operators $M_1, M_2$ and Hamiltonian $H_1, H_2$. The energy of
composed system is $H_1 + H_2$; on the other hand, the operator of the composed
system will also be $M = M_1 + M_2$, so $H = f(M_1 + M_2)$ and $f(M_1 + M_2) =
f(M_1) + f(M_2)$, so $f(M)$ must be a linear function.


\end{document}